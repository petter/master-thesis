%! Author = Petter
%! Date = 3/16/2021

\chapter*{Endringer fra forrige møte}

\section*{Struktur}

Slått sammen "Differences between PTS and PTj" og "Fulfilling requirements of PT" til "Discussion".

\section*{Introduction}

\begin{itemize}
    \item Skrevet en røff introduksjon.
    Section~\vref{ch:introduction}.
\end{itemize}

\section*{Background}

\begin{itemize}
    \item Skrive litt mer JavaScript background, og forklare hvordan språkets prototype-baserte objekt-orientering fungerer.
    Section~\vref{subsec:javascript}.
    \item Skrive litt om hva TypeScript er.
    Section~\vref{subsec:what-is-typescript}.
\end{itemize}

\section*{The Language - PTS}

\begin{itemize}
    \item Lite avsnitt som beskriver eksempel programmet.
    Section~\vref{sec:example-program}.
\end{itemize}

\section*{Planning the project}

\begin{itemize}
    \item Skrive litt om hvordan vi kan møte de samme problemene i renaming i TypeScript som i JavaScript pga \codeword{any}.
    Section~\vref{subsec:ts-vs-js-renaming}.
    \item Gå fra "implementing as a TS library" til "Implementing as an internal DSL".
    Section~\vref{subsec:implementing-pt-as-a-ts-library}.
    \item Vise til GroovyPT og se at vi kan oppnå noe lignende gjennom JS sine prototyper, som GroovyPT brukte MOPs.
    Section~\vref{subsec:implementing-pt-as-a-ts-library}.
    \item Skrive en kort konklusjon til valg av vertspråk.
    Section~\vref{subsec:langauge-choice-conclusion}.
\end{itemize}

\section*{Implementation}

\begin{itemize}
    \item Skrive om hvordan \codeword{addto} kunne blitt implementert.
\end{itemize}

\section*{Discussion}

\begin{itemize}
    \item Legge til en setning som recap av kravet, til oppfylling av "parallel extension"-kravet seksjonen.
    Section~\vref{subsubsec:pts-parallel-extension}.
    \item Skrevet om oppfylling av kravet til "class merging".
    Section~\vref{subsubsec:pts-class-merging}.
\end{itemize}

\section*{Results}

\begin{itemize}
    \item
\end{itemize}
