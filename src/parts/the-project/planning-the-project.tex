%! Author = Petter
%! Date = 9/22/2020

\chapter{Planning the project}\label{ch:planning-the-project}

%\section{Implementing PT as a TS library}\label{sec:implementing_pt_as_a_ts_library}
%
%In order to implement PT we need to be able to handle the following:
%
%\begin{itemize}
%    \item Defining templates
%    \item Instantiating templates
%    \item Renaming classes
%    \item Renaming class attributes
%    \item Merging classes
%\end{itemize}
%
%\subsection{Defining Templates}\label{sub:defining_templates}
%
%Templates could be defined as an ECMAScript class, where each member of the template is a static attribute.
%
%\subsection{Instantiating Templates}\label{sub:instantiating_templates}
%
%\subsection{Renaming Classes}\label{sub:renaming_classes}
%
%Since each class in a template is just a static member, we could create a new template where we use the new name for our class as the attribute name, and point to the class from the "old" template.
%
%TODO: Because of ES having open classes this could lead to unwanted side-effects.
%Might need to look into a different solution for this.
%
%\subsection{Renaming Class Attributes}\label{sub:renaming_class_attribtues}
%
%Maybe impossible?
%
%\subsection{Merging Classes}\label{sub:merging_classes}
%
%For merging of the types you would use the built-in decleration merging~\cite{declerationmerging}.
%Implementation merging is also possible because ECMAScript has open classes.
%For implementation merging you would create an empty class which has the type of the merged declarations, and then assign the fields and methods from the merging classes to this class.

% TODO Les over og se om dette er brukbart

\subsection{Implementing PT as a TS Library}\label{subsec:implementing-pt-as-a-ts-library}

One of the first approaches we need to check out is if we are able to achieve the functionality of PT, without having to create a compiler.
Creating a library would presumably be an easier task than having to create a compiler, and it would also make it easier to use, as the programmer would not have to install a compiler in order to get the PT functionality.
In order to implement PT we need to be able to handle the following:

\begin{itemize}
    \item Defining templates
    \item Renaming classes and class attributes
    \item Instantiating templates
    \item Merging classes
\end{itemize}

\subsubsection{Defining Templates}\label{subsubsec:defining-templates}

For defining templates we would like a construct that can wrap our template classes in a scope.
We will also need to be able to reference the template.
JavaScript has three options for this, an array, an object or a class.
It should however also be possible to inherit from classes in your own template, which rules out both arrays and objects, as there is no way of referencing other members during definition of the array/object.
Templates should therefore be defined as classes, where each member of the template is an attribute of the template class.
In listing~\vref{code:libraryimpl-template} we see an example of how this could be done.

%We are making the templates classes static in order to be able to rename them, see section~\vref{sub:renaming_classes}.

\codeinputfile{template-example-without-decorator.ts}{typescript}{Example of defining a template in a library implementation.}{code:libraryimpl-template}

\subsubsection{Renaming Classes and Class Attributes}\label{subsubsec:renaming-classes-and-class-attributes}

Renaming of classes is possible to an extent.
Since we made the classes static attributes of the template class we could easily just create a new static field on the template class and use the \codeword{delete}-op~\footnote{An operator in JavaScript for removing a property of an object. See~\url{https://developer.mozilla.org/en-US/docs/Web/JavaScript/Reference/Operators/delete}.} to remove the old field.
We can see an example of this in listing~\vref{code:libraryimpl-template-renaming}.


\codeinputfile{class-renaming-example.ts}{typescript}{Example of renaming a template class}{code:libraryimpl-template-renaming}

Even though we were able to give the class a "new name", this would still not actually rename the class.
Any reference to the old names would be left unchanged, and thus we are not able to achieve renaming in TypeScript.
Listing~\vref{lst:lib-rename-problem} shows how this can be a problem, where the function \codeword{f} of class \codeword{X} would fail at run-time due to it not being able to find class \codeword{A}.

\begin{code}{typescript}{Example showcasing the problems of renaming classes in a template in the library implementation.}{lst:lib-rename-problem}
    // Type-safe template declaration
    class T1 {
        static A = class {
            i = 0;
        }
        static X = class {
            f() {
                return new A();
            }
        }
    }

   // Renaming
    const classRef = T1.A;
    T1.B = classRef;
    delete T1.A;

    // Trying to use the template after renaming
    const x = new T1.X();
    x.f(); // ReferenceError: A is not defined
\end{code}

Attribute renaming would most likely be possible in a similar manner, where we could change the prototype~\footnote{The prototype of a class is an object which objects of the class inherit their methods from. See~\url{https://developer.mozilla.org/en-US/docs/Learn/JavaScript/Objects/Object_prototypes}.} of the class.
Seeing as we are not able to fully rename classes by doing this we will not be looking further into this.

\subsubsection{Instantiating Templates}\label{subsubsec:instantiating-templates}

As with renaming, we are also able to instantiate templates to an extent.
We are able to iterate over the attributes of the template class, and populate a package/template with references to the template.
An example of this can be seen in listing~\vref{code:libraryimpl-template-inst}.

\codeinputfile{inst-template-example.ts}{typescript}{Example of instantiating a template}{code:libraryimpl-template-inst}

The instantiation will only contain references to the instantiated templates classes, while PT instantiations make textual copies of the templates content.
Only having references to the original template could mean that if a template that has been instantiated is later renamed, then the instantiated template might lose some of its references.
We could possibly avoid circumvent this by getting the textual representation of the class, through the class' \codeword{toString}, and then use \codeword{eval} to evaluate the class declaration.

\subsubsection{Merging Classes}

For merging of types you would use the built-in declaration merging~\cite{declerationmerging}.
Implementation merging is also possible because JavaScript has open classes.
For implementation merging you would create an empty class which has the type of the merged declarations, and then assign the fields and methods from the merging classes to this class.
There are several libraries that supports class merging, such as mixin-js~\cite{mixinjs}.

\subsubsection{Conclusion}

Since we are not able to support renaming fully we will not be able to implement PT as a library for TypeScript.
Because of this we will have to find another approach for the project.


\section{What Do We Need?}\label{sec:what-do-we-need}

\begin{itemize}
    \item The ability to add custom syntax (access to the tokenizer / parser)
    \item Some semantic analysis.
\end{itemize}

\section{Syntax}\label{sec:syntax}

For the implementation of PT we need syntax for the following:

\begin{itemize}
    \item Defining packages (\codeword{package} in PTj)
    \item Defining templates (\codeword{template} in PTj)
    \item Instantiating templates (\codeword{inst} in PTj)
    \item Renaming classes (\codeword{=>} in PTj)
    \item Renaming methods (\codeword{->} in PTj)
    \item Additions to classes (\codeword{addto} in PTj)
\end{itemize}

\codeword{template} and \codeword{inst} are both not in use nor reserved in the ECMAScript standard or in TypeScript, and can therefore be used in \plname{} without any issues.

The keyword \codeword{package} in TS / ES is as of yet not in use, however the ECMAScript standard has reserved it for future use.
In order to "future proof" our implementation we should avoid using this reserved keyword, as it could have some conflicts with a potential future implementation of packages in ECMAScript.
It could also be beneficial to not share the keyword in order avoid creating confusion between the future ES packages and PT Packages.
\codeword{module} is also a keyword that could be used to describe a PT package, however this is already used in the ES standard, and should therefore also be avoided for similar reasons to \codeword{package}, to avoid confusion.
We will therefore use (\codeword{pack} eller \codeword{bundle}? Må nok se litt mer på dette) instead. % TODO: find a proper name for package.

For renaming classes PTj uses \codeword{=>}, however in ES this is used in arrow-functions\cite{arrowfunction}.
To avoid confusion and a potentially ambiguous grammar we will have to choose a different syntax for renaming classes.
PTj, for historical purposes, uses a different operator (\codeword{->}) for renaming class methods, however for keeping \plname{} simple we will stick to only having one common operator for renaming.

ECMAScript currently supports renaming of destructured fields using the \codeword{:}(colon) operator and aliasing imports using the keyword \codeword{as}.
Even though we opted to choose a different keyword for packages, we will here re-use the already existing \codeword{as} keyword for renaming as the concepts are so closely related.

\plname{}:
\begin{lstlisting}
    template T {
        class A {
            function f() : String {
                ...
            }
        }
    }

    pack P {
        inst T with A as A (f as g); // Function overloading not supported, so don't need to give signature.
        addto A {
            i : number = 0;
        }
    }
\end{lstlisting}

PTj:
\begin{lstlisting}
    template T {
        class A {
            String f() {
                ...
            }
        }
    }

    package P {
        inst T with A => A (f() -> g()); 
        addto A {
            int i = 0;
        }
    }
\end{lstlisting}



\section{TypeScript vs JavaScript / ECMAScript}

\section{Choosing the right approach}\label{sec:choosing-the-right-approach}

Before jumping into a project of this magnitude it is important to find out what approach to use. 
The goal of this project is to extend TypeScript with the Package Templates language mechanism, this can be achieved as following:

\begin{itemize}
    \item Making a fork of the TypeScript compiler
    \item Making a preprocessor for the TypeScript compiler
    \item Making a compiler plugin / transform
    \item Making a custom compiler from scratch
\end{itemize}


\subsection{Preprocessor for the TypeScript Compiler}\label{subsec:preprocessor-for-the-typescript-compiler}

More work than ex plugin / transformer.

\subsection{TypeScript Compiler Plugin / Transform}\label{subsec:typescript-compiler-plugin}

As of the time of writing this the official TypeScript compiler does not support compile time plugins. The plugins for the TypeScript compiler is, as the TypeScript compiler wiki specifies, "for changing the editing experience only"\cite{tscplugin}.
However there are alternatives that do enable compile time plugins / transformers;

\begin{itemize}
    \item ts-loader\cite{tsloadergithub}, for the webpack ecosystem
    \item Awesome Typescript Loader\cite{awesometypescriptloadergithub}, for the webpack ecosystem. Deprecated
    \item ts-node\cite{tsnodegithub}, REPL / runtime
    \item ttypescript\cite{ttypescriptgithub}, TypeScript tool TODO: Les mer på dette
\end{itemize}

Unfortunately ts-loader, Awesome Typescript Loader and ts-node does not support adding custom syntax, as it only transforms the AST produced by the TypeScript compiler.
Because of this they are not a viable option for our use-case and will therefore be discarded.

\subsection{Babel plugin}\label{subsec:babel-plugin}

Babel isn't strictly for TypeScript, but for JavaScript as a whole, however we could write our plugin to be dependent on the TypeScript transformation plugin.

Making a Babel plugin will make it very accessible as most web-projects use Babel, and the upkeep is cheap, as plugins are loosely coupled with the core.

In order for a Babel plugin to support custom syntax it has to provide a custom parser, a fork of the Babel parser.
Through this we can extend the TypeScript syntax with our syntax for PT.
This is all hidden away from the user, as this custom parser is a dependency of our Babel plugin.

Seeing as we have to make a fork of the parser in order to solve our problem, the upkeep will not be as cheap as first anticipated.
However being able to have most of the logic loosely coupled with the compiler core it will still make it easier to keep updated than through a fork of the TypeScript compiler.

TODO: Er det støttet å bruke flere plugins med forskjellige parsere?
E.g. babel-plugin-typescript + vårt babel plugin?

\subsection{TypeScript Compiler Fork}\label{subsec:typescript-compiler-fork}

Possible, however not as accessible as other alternatives and will make upkeep expensive.

The TypeScript compiler is a monolith.
It has about 2.5 million lines of code, and therefore has a quite steep learning curve to get into.
If we were to go with this route it would be quite hard to keep up with the TypeScript updates, as updates to the compiler might break our implementation.
However as we have seen, going the plugin / transform route also requires us to fork the underlying compiler and make changes to it, however with the majority of the implementation being loosely coupled it would still make it easier to keep up-to-date.
That being said it will probably be a lot easier to do semantic analysis in a fork of the TypeScript compiler vs in a plugin / transform.
