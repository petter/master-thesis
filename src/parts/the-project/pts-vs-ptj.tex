%! Author = petter
%! Date = 04.01.2021

\chapter{Difference between PTS and PTj}\label{ch:difference-between-pts-and-ptj}


\section{Nominal vs. Structural Typing}\label{sec:nominal-vs-structural-typing}

Nominal and structural are two major categories of type systems.
In nominal typing a type is

Write about pros and cons of both nominal and structural typing

nominal pros:
Trivial to check if a type is a subtype of another
Natural and intuitive recursive types (Structural typing also has recursive types, check if java and typescript support it)
Often runtime-objects are tagged with the types, which are useful for multiple things like doing runtime instanceof checks. (This can also be used in structural typing)

\subsection{What is nominal typing?}\label{subsec:what-is-nominal-typing?}



\begin{lstlisting}[label={lst:nominal-typing-example}, language=Java]
    // Given the following class definitions for A, B and C:
    class A {
        void f() {
            ...
        }
    }

    class B extends A {
        ...
    }

    class C {
        void f() {
            ...
        }
    }

    // And a consumer with the following type:
    void g(A a) { ... }

    // Would result in the following
    g(new A()); // Ok
    g(new B()); // Ok
    g(new C()); // Error, C not of type A
\end{lstlisting}

\subsection{What is structural typing?}\label{subsec:what-is-structural-typing?}

\begin{lstlisting}[label={lst:structural-typing-example2}, language=Java]
    // Given the same class definitions and the same consumer as in the example above.
    // Would result in the following
    g(new A()); // Ok
    g(new B()); // Ok
    g(new C()); // Ok, because C is structurally equal to A
\end{lstlisting}


