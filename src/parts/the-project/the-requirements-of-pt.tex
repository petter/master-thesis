% TODO: Kanskje bedre å ha denne inn i background til PT?
\section{The Requirements of PT}\label{sec:the-requirements-of-pt}

What are the requirements of PT?

As described in~\cite{jot}

\begin{itemize}
    \item Parallel extension
    \item Hierarchy preservation
    \item Renaming
    \item Multiple uses
    \item Type parameterization
    \item Class merging
    \item Collection-level type-checking
\end{itemize}

\subsection{Parallel Extension}\label{subsec:parallel-extension}
% - \textbf{Parallel  extension:}  When  using  the  collection  C  in  a  certain  setting  we  can  add  attributes  to  A  and  B.
% These  additions  should  also  have  effect  for  the  code  of  C,  e.g.  so  that  we  by  means  of  an  A-variable  defined  in  C  can  directly  (without  casting) access the attributes added to A.
%
% - - In the graph example, assume that we have added the int variable length to Edge, and  that  n  is  a  Node-reference.
% With  this  property  we  can  conveniently  specify  directly: “n.firstEdge().length = 5;”, as firstEdge is typed with the extended Edge class.

\subsection{Hierarchy Preservation}\label{subsec:hierachy-preservation}
% - \textbf{Hierarchy  preservation:}  The  mechanism  should  allow  B  to  be  a  subclass  of  A,  and  if  additional  attributes  are  given  to  A  and  to  B,  then  the  B  with  additions  should be a subclass of the A with additions.
% Note that this will not be the case if we just use the collection C with the classes A and B and then define subclasses A’  and  B’  to  A  and  B,  respectively,  with  the  additions  we  want  in  these  subclasses.
% B’ will then not be a subclass of A’.
%
% - - In  the  compiler  example  this  is  exactly  what we need in order to be able to add attributes as explained in the example.

\subsection{Renaming}\label{subsec:renaming}

% - \textbf{Renaming:}  When  C  is  used,  we  should  be  able  to  change  the  name  of  A  and  B,  and of their attributes, so that they fit with the specific use situation.
%
% - - For  the  graph  example,  the  renaming  property  makes  it  possible  to  rename  the  nodes and edges to cities and roads.

\subsection{Multiple Uses}\label{subsec:multiple-uses}
% - \textbf{Multiple  uses:}  It  should  be  possible  to  use  the  classes  of  C  for  different  and  independent  purposes  in  the  same  program,  and  so  that  each  purpose  have  different additions and renamings.
% The compiler should be able to check that each use implies a different set of classes as if they are defined in separate hierarchies.
%
% - - In  a  program  we  may  need  the  basic  graph  structure  for  different  purposes.
In  addition  to  using  it  for  cities  and  roads,  we  could  in  the  same  program  use  it  to  form the structure of pipes and joints in a water distribution system.

\subsection{Type Parameterization}\label{subsec:type-parameterization}

% - \textbf{Type parameterization:} It should be possible to write a collection of classes that assumes the existence of classes that have some required attributes, but are not yet completely defined.
% In each use of this collection, one can provide specific classes that have at least the required attributes.
%
% - - In the compiler example we assume that the front end shall always produce Java Bytecode,  and  we  use  some  readymade  mechanism  for  packing  the  code  to  a  correctly  formatted  class  file.
% However,  a  number  of  such  packers  may  be  available,  and  we  do  not  want  to  choose  which  to  use  while  implementing  the  front end class collection.

\subsection{Class Merging}\label{subsec:class-merging}


% - \textbf{Class  merging:}  Assume  we  have  the  two  collections  C  (with  classes  A  and  B)  and D (with class E).
% When they are both used in the same program, we should be able to merge e.g. the classes A and E so that the resulting class gets the union of the attributes, and so that we via an E-variable defined within D can also directly see the A attributes (and similarly for an A-variable in C).
%
% - - Assume that we in addition to the graph collection have a collection Persons with a class Person.
% In a program handling personal relations we then want to use both collections together so that we obtain a new class, say PersonNode, which has all the  attributes  of  Node  and  Person,  and  where  a  Person-variable  p  defined  in  Persons gives access directly to the Node attributes, e.g. “p.firstEdge”.

\subsection{Collection-Level Type-Checking}\label{subsec:collection-level-type-checking}

% In  addition  to  these  properties,  it  is  important  that  such  collections  of  classes  can  be  separately type-checked.
% We also prefer a mechanism that allows only single inheritance, as  the  merge  property  described  above  to  a  large  extent  will  take  care  of  the  need  for  combining  code  from  different  sources  (for  which  purpose  multiple  inheritance  is  often  used).
% Finally, the type system should be as simple and intuitive as possible.
