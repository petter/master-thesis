%! Author = petter
%! Date = 04.01.2021

\chapter{Difference between PTS and PTj}\label{ch:difference-between-pts-and-ptj}


\section{Nominal vs. Structural Typing}\label{sec:nominal-vs-structural-typing}

%Much taken from TAPL.

One of the most notable differences between PTS and PTj are the underlying languages type systems.
PTS, as an extension of TypeScript, has structural typing, while PTj on the other hand, an extension of Java, has nominal typing.

Nominal and structural are two major categories of type systems.
Nominal is defined as "being something in name only, and not in reality" in the Oxford dictionary.
Nominal types are as the name suggest, types in name only, and not in the structure of the object.
They are the norm in mainstream programming languages, such as Java, C, and C++.
A type could be \codeword{A} or \codeword{Tree}, and checking whether an object conforms to a type restriction, is to check that the type restriction is referring to the same named type, or a subtype.
Structural types on the other hand is not tied to the name of the type, but to the structure of the object.
These are not as common in mainstream programming languages, but are very prominent in research literature.
However, in more recent (mainstream) programming languages, such as Go, TypeScript and Julia (at least for implicit typing), structural typing is becoming more and more common.
A type in a structurally typed programming language is often defined as a record, and could for example be \codeword{\{ name:~string \}}.

\subsection{Advantages of Nominal Types}\label{subsec:advantages-of-nominal-types}

\subsubsection{Subtypes}\label{subsubsec:subtypes}

In nominal type systems it is trivial to check if a type is a subtype of another, as this has to be explicitly stated, while in structural type systems this has to be structurally checked, by checking that all members of the super type, are also present in the subtype.
Because of this each subtype relation only has to be checked once for each type, which makes it easier to make a more performant type checker for nominal type systems.
However, it is also possible to achieve similar performance in structurally typed languages through some clever representation techniques~\cite{tapl}. % TODO: Maybe find an article describing this
We can see an example of subtype relations in both nominal and structural type systems, in a Java-like language, in listing~\vref{lst:subtype}.
It is important to note that even though \codeword{C} is a \emph{subtype} of \codeword{A} in a structural language, it is not a \emph{subclass} of \codeword{A}.

\begin{code}{Java}{Example of subtype relations in nominal and structural typing, in a Java-like language.}{lst:subtype}
    // Given class A
    class A {
        void f() { ... }
     }

    // A subtype, B, in nominal typing
    class B extends A { ... }

    // A subtype, C, in structural typing
    class C {
        void f() { ... }
        int g() { ... }
    }
\end{code}

% TODO: Vurder om recursive types er verdt å ta med
% \subsubsection{Recursive types}\label{subsubsec:recursive-types}
%
% Recursive types are types that mention itself in its definition, and are widely used in datastructures, such as lists and trees.
% An advantage in nominal typing is how natural and intuitive recursive types are in the type system.
% Referring to itself in a type definition is as easy as referring to any other type.
% It is however just as easy to do this in structural type systems as well, however for calculi such as type safety proofs, recursive types come for free in nominal type systems, while it is a bit more cumbersome in structurally typed systems, especially with mutually recursive types~\cite{tapl}.
% Listings~\vref{lst:ts-recursive-type} and~\vref{lst:java-recursive-type} shows the use of recursive types in TypeScript (structurally typed) and Java (nominally typed), respectively.
%
% \begin{code}{TypeScript}{Usage of a recursive type, \codeword{Tree}, in TypeScript}{lst:ts-recursive-type}
%     interface Tree<T> {
%         getData(): T;
%         getChildren(): Tree<T>[];
%     }
% \end{code}
%
% \begin{code}{Java}{Usage of a recursive type, \codeword{Tree}, in Java}{lst:java-recursive-type}
%     interface Tree<T> {
%         T getData();
%         List<Tree<T>> getChildren();
%     }
% \end{code}

\subsubsection{Runtime Type Checking}\label{subsubsec:runtime-type-checking}

Often runtime-objects in nominally typed languages are tagged with the types (a pointer to the "type") of the object.
This makes it cheap and easy to do runtime type checks, like in upcasting or doing a \codeword{instanceof} check in Java.
% TODO: Skriv mere

\subsection{Advantages of Structural Types}\label{subsec:advantages-of-structural-types}

\subsubsection{Tidier and More Elegant}\label{subsubsec:tidier-and-more-elegant}

Structural types carry with it all the information needed to understand its meaning.

\subsubsection{Advanced Features}\label{subsubsec:advanced-type-features}

Type abstractions (parametric polymorphism, ADTs, user-defined type operators, functors, etc), these do not fit nice into nominal type systems.

\subsubsection{More General Functions/Classes}\label{subsubsec:more-general-functions}

See~\cite{malayeri}.

\subsection{What Difference Does This Make For PT?}\label{subsec:what-difference-does-this-make-for-pt?}

We are not going to go further into comparing nominal and structural type systems and "crown a winner", as there are a lot useful scenarios for both.
We will instead look more closely into how a structural type system fits into PT, and what differences this makes to the features, and constraints, of this language mechanism.

%Often runtime-objects are tagged with the types, which are useful for multiple things like doing runtime instanceof checks. (This can also be used in structural typing)
%It is "easier" to check sub-type relations in nominal type systems, even though you will still have to do a structural comparison, you only have to do this calculation once per type.
%    Even though it is easier to achieve better performance for nominal vs structural types, it is still possible to achieve close to if not equal performance in structural type systems, through some representation techniques(Consider finding a article on this).
%A questionable advantage for nominal typing is that they prevent "spurious subsumption", that is using a structurally equal type in a place where it should not be used.
%    I would argue that there are more cases where this is a compromise than a benefit, and there are better ways to prevent this from happening, such as single-constructor datatypes or ADTs.
%
%nominal cons:
%
%structural pros:
%Structural types are "tidier and more elegant".
%    They carry all the information to understand the meaning of the type.
%    In nominal systems we are working with some global collection of names, which can make it hard to understand some types.
% Structural types can help to make a method more general, reduce code duplication. See is structural subtyping useful.
%
%
%structural cons:
%




\begin{code}{Java}{Example of a nominally typed program in a Java-like language}{lst:nominal-typing-example}
    // Given the following class definitions for A, B and C:
    class A {
        void f() {
            ...
        }
    }

    class B extends A {
        ...
    }

    class C {
        void f() {
            ...
        }
    }

    // And a consumer with the following type:
    void g(A a) { ... }

    // Would result in the following
    g(new A()); // Ok
    g(new B()); // Ok
    g(new C()); // Error, C not of type A
\end{code}


\begin{code}{Java}{Example of a structurally typed program in a Java-like language}{lst:structural-typing-example}
    // Given the same class definitions and
    // the same consumer as in the example above.
    // Would result in the following
    g(new A()); // Ok
    g(new B()); // Ok
    g(new C()); // Ok, because C is structurally equal to A
\end{code}

\subsection{Which Better Fits PT?}\label{subsec:which-better-fits-pt?}

%In PTj(specifically the version with required types) there are elements of structural typing for the implementation of required types.
%This is because PT doesn't allow externally declared classes, interfaces, etc., so in order to be able to have constraints on generic types structural typing was needed.
%I would therefore argue that a structurally typed language is a better fit for PT, because Ockhams razor / keeping it tidier.

