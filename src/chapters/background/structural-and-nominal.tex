\section{Structural and Nominal Type Systems}\label{sec:structural-and-nominal-type-systems}

Throughout this thesis we will have a major focus on the underlying type systems of traditional PT in Java, and our implementation of PT in TypeScript.
Java has what we call a \emph{nominal} type system, while TypeScript has a \emph{structural} type system.

Nominal is defined as "being something in name only, and not in reality" in the Oxford dictionary.
Nominal types are as the name suggest, types in name only, and not in the structure of the object.
They are the norm in mainstream programming languages, such as Java, C, and C++.
A type could be \codeword{A} or \codeword{Tree}, and checking whether an object conforms to a type restriction, is to check that the type restriction is referring to the same named type, or a subtype.

Structural types on the other hand are not tied to the name of the type, but to the structure of the object.
These are not as common in mainstream programming languages, but are very prominent in research literature.
However, in more recent (mainstream) programming languages, such as Go and TypeScript, structural typing is becoming more and more common.
A type in a structurally typed programming language is often defined as a record, and could for example be \codeword{\{ name:~string \}}.

In listing~\vref{lst:nominal-typing-example} we can see an example of a nominally typed program in a Java-like language.
Here \codeword{B} is a subtype of \codeword{A}, while \codeword{C} is not.
This is due to nominally typed programs having the requirement of explicitly naming its subtype relations, through e.g.\ a subclass-relation.
Because of this we can see that at the bottom of the listing the first two statements pass, since both \codeword{A} and \codeword{B} are of type \codeword{A}, while the last statement fails (typically at compile time), as \codeword{C} is not of type \codeword{A}.

In listing~\vref{lst:structural-typing-example} we see a structurally typed program.
This program also has the exact same declarations as in listing~\vref{lst:nominal-typing-example}, that is classes \codeword{A}, \codeword{B}, and \codeword{C} and the function \codeword{g}.
In this program both type \codeword{B} and type \codeword{C} are a subtype of type \codeword{A}, since they both contain all members of type \codeword{A}.
Not necessarily the same implementation as in class \codeword{A}, but the same types as in type \codeword{A}.
This is one of the major differences between nominal and structural typing, types can conform to other types without having to explicitly state that they should.
Type \codeword{C} is an example of this, while it does not have a subclass relation to class \codeword{A}, nor implement any common nominal interface, it still conforms to the type of \codeword{A}.
The result of this is that all three usages of function \codeword{g} are valid in a structural type system, while consuming \codeword{C} was illegal in the nominal example.


\begin{code}{Java}{Example of a nominally typed program in a Java-like language}{lst:nominal-typing-example}
    // Given the following class definitions for A, B and C:
    class A {
        void f() {
            ...
        }
    }

    class B extends A {
        ...
    }

    class C {
        void f() {
            ...
        }
    }

    // And a consumer with the following type:
    void g(A a) { ... }

    // Would result in the following
    g(new A()); // Ok
    g(new B()); // Ok
    g(new C()); // Error, C not of type A
\end{code}

\begin{code}{Java}{Example of a structurally typed program in a Java-like language}{lst:structural-typing-example}
    // Given the same class definitions and
    // the same consumer as in the previous listing.
    // Would result in the following
    g(new A()); // Ok
    g(new B()); // Ok
    g(new C()); // Ok, because C is structurally equal to A
\end{code}
