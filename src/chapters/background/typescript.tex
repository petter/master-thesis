%! Author = Petter
%! Date = 3/16/2021

\section{TypeScript}\label{sec:typescript}

Before we look at what TypeScript is we first need to understand JavaScript and the JavaScript ecosystem.

\subsection{JavaScript}\label{subsec:javascript}

%TODO: Nevne hvorfor JavaScript er her/motivasjonen bak JavaScript. Snakke om verden før JavaScript med usikre Java Applets og Flash

A web page generally consists of three layers of technologies.
The first layer is HTML, which is the markup language that is used to structure the web page.
Second is CSS which gives our structured documents styling such as background colors and positioning.
The third and final layer is JavaScript which enables web pages to have dynamic content.
Whenever you visit a website that isn't just static information, but instead might have timely content updates, interactive maps, etc., then JavaScript is most likely involved~\cite{whatisjs}.

JavaScript, or more precisely ECMAScript, is the programming language of the web.
It is a language that conforms to the ECMAScript specification.
ECMAScript is simply a JavaScript standard, created by Ecma International, made to ensure interoperability across different browsers.
There is no official runtime or compiler for JavaScript as it is up to each browser to implement the languages runtime.
When we create a JavaScript program/script for a web page we don't compile it and transfer a binary or bytecode file for the web page to execute, instead the browser takes the raw source code and interprets it\footnote{On a more technical level, JavaScript is generally just-in-time compiled in the browser.}.

% TODO
JavaScript is a multi-paradigm language with dynamic typing...

\subsubsection{ECMAScript Versions}

ECMAScript versions are generally released on a yearly basis.
This release is in the form of a detailed document describing the language, ECMAScript, at the time of release.
New versions will most likely include some additions to the language, but never any breaking changes\footnote{There has been occasions where there has been minor breaking changes between ECMAScript versions, but these are both rare in occurence and encounters.}.
This is because the developer will not be able to control the environment on which the code will be executed\footnote{It is possible to ship chunks of code called \textit{polyfills} with your code in order to regain some control over the enviornment. We will go more in-depth on this in section~\vref{subsubsec:backwards-compatability}.} and you can not be sure which ECMAScript version the client browser will be using.
Because of this lack of control over the runtime environment it is crucial that any pre-existing language features don't have breaking changes between versions.

\subsubsection{Backwards Compatibility}\label{subsubsec:backwards-compatability}

With new ECMAScript versions comes new features, and it is up to each browser to implement these changes.
As we mentioned earlier, we do not transfer a binary to the client browser, we transfer the source code.
So when a JavaScript script uses a new ECMAScript feature it is not guaranteed to work with every client browser, since a lot of users might have older browsers installed, or the team behind the browser has not implemented the language feature yet.
To deal with this a common practice in JavaScript development is to first transpile the source code before using it in a production environment.
This transpilation step takes the source code and transpiles it into an older ECMAScript version.
In doing this you ensure that more client browser will be able to run the script.
This will rewrite the new language features, and often replace them with a function, called a \textit{polyfill}.
You can think of a polyfill as an implementation of a new language feature.

Some popular transpilers for JS to JS transpilation are Webpack and Babel, but you could also use the TypeScript compiler for this.

\subsubsection{Node.js}

Node.js (henceforth simply referred to as Node) is a JavaScript runtime built on the JavaScript engine, V8, used by Chrome.
While most JavaScript runtimes live in the browser, Node doesn't.
It is independent of the browser and can be run through a CLI.
One major difference from the browser runtimes is that Node also supplies some libraries for IO, such as access to the file system, HTTP, and WebSocket.
This makes Node a good choice for writing networking applications for instance.

We will be using Node for our compiler since it gives us access to the file system as well as enabling the compiler to be accessible through a CLI.

\subsection{What is TypeScript?}\label{subsec:what-is-typescript}

TypeScript is a superset of JavaScript.
The language builds on JavaScript with the additions of static type definitions~\cite{tswebsite}.

All valid JavaScript programs are also valid TypeScript programs.
