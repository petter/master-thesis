%! Author = Petter
%! Date = 3/16/2021

\section{TypeScript}\label{sec:typescript}

Before we look at what TypeScript is we first need to understand JavaScript and the JavaScript ecosystem.

\subsection{JavaScript}\label{subsec:javascript}

Back in the mid-90s web pages could only be static, however, there was a desire to remove this limitation and make the web a more interactive platform, as it became increasingly more accessible to non-technical users.
In order to remove this limitation, Netscape, with its Netscape Navigator browser, partnered up with Sun to bring the Java platform to the browser and hired Brendan Eich to create a scripting language for the web.
Eich was tasked to create a Scheme-like language with syntax similar to Java and the language was intended to be a companion language to Java.
The language when it first released was called LiveScript, however, it was later renamed to what we know it as today, JavaScript.
This has been characterized as a marketing ploy by Netscape to give the impression that it was a Java spin-off.

Microsoft, with its Internet Explorer, adopted the language and named it JScript.
During this time Microsoft and Netscape would both ship new features to the language in order to increase the popularity of their respective browsers.
Because of this war between browsers the language was later handed over to ECMA International as a starting point for a standard specification for the language.
This ensured that users would get the same experience across different browsers, making the web more accessible~\cite{jswikipedia}.

A web page generally consists of three layers of technologies.
The first layer is HTML, which is the markup language that is used to structure the web page.
Second is CSS which gives our structured documents styling such as background colors and positioning.
The third and final layer is JavaScript which enables web pages to have dynamic content.
Whenever you visit a website that isn't just static information, but instead might have timely content updates, interactive maps, etc., then JavaScript is most likely involved~\cite{whatisjs}.

JavaScript is a programming language conforming to the ECMAScript standard.
ECMAScript is a JavaScript standard, created by Ecma International, made to standardize the JavaScript language and ensure interoperability across different browsers.
There is no official runtime or compiler for JavaScript as it is up to each browser to implement the languages runtime.
When we create a JavaScript program/script for a web page we don't compile it and transfer a binary or bytecode file for the web page to execute, instead, the browser takes the raw source code and interprets it\footnote{On a more technical level, JavaScript is generally just-in-time compiled in the browser.}.

JavaScript is a multi-paradigm language, mainly consisting of object-orientation and functional programming, with a dynamic type system.
It is object-oriented in the way that most data structures are represented through objects, and functional in the way that it has first-class functions, where functions can be free from a class and are treated as values that can be assigned to variables and sent around as parameters.

Where most object-oriented programming languages are class-based, like Java, JavaScript is \textit{prototype-based}.
What this means is that the objects are not class instances, but are rather "instances" of a prototype.
What you would normally think of as a class instance in Java, is an object with a reference to a prototype object.
These instances are created through constructor functions, which create an object and sets the prototype for the object.

An object in JavaScript is a "bag" of properties containing values, which are specified in the prototype constructor, and a reference to the prototype object it is an instance of.
The prototype object is not special in any way, it is just another object that has contains values that can be commonly used by all objects with the same prototype, and can themselves have prototype objects.
This is how inheritance works in JavaScript, chains of prototype objects, where the \codeword{Object} prototype is at the end of the chain, similar to how the \codeword{Object} class is at the top of the inheritance hierarchy in Java.
The \codeword{Object} prototype has \codeword{null} as its prototype, and \codeword{null} does not have any prototype.
When trying to access a member of an object the object itself is first checked, then its prototype, and the prototype's prototype and so on, following the chain of prototypes, until a match is found~\cite{prototypechaining}, or until there are no more prototypes to follow.
Since prototypes are just objects they can as with any other object be changed at runtime or replaced by other prototypes.

In ECMAScript 2015 there was introduced a class-syntax, however, this is just syntactic sugar for creating the prototype object, and the associated constructor.
Extending a class with this syntax is as you would expect just defining the prototype for the prototype object.

\subsubsection{ECMAScript Versions}\label{subsubsec:ecmascript-versions}

ECMAScript versions are generally released on a yearly basis.
This release is in the form of a detailed document describing the language, ECMAScript, at the time of release.
New versions will most likely include some additions to the language, but never any breaking changes\footnote{There have been occasions where there have been minor breaking changes between ECMAScript versions, but these changes happen very rarely and the affected areas are often insignificant.}.
This is because the developer will not be able to control the environment on which the code will be executed since you can not be sure which ECMAScript version the client browser is using.
Because of this lack of control over the runtime environment, it is crucial that any pre-existing language features don't have breaking changes between versions.

\subsubsection{Backwards Compatibility}\label{subsubsec:backwards-compatability}

With new ECMAScript versions comes new features, and it is up to each browser to implement these changes.
As we mentioned earlier, we do not transfer a binary to the client browser, we transfer the source code.
So when a JavaScript script uses a new ECMAScript feature it is not guaranteed to work with every client browser, since a lot of users might have older browsers installed, or the team behind the browser has not implemented the language feature yet.
To deal with this a common practice in JavaScript development is to first transpile the source code before using it in a production environment.
This transpilation step takes the source code and transpiles it into an older ECMAScript version.
In doing this you ensure that more browsers will be able to run the script.
This will rewrite the new language features, and often replace them with a function, called a \textit{polyfill}.
You can think of a polyfill as an implementation of a new language feature that you ship with your code.
These polyfills help the developer regain some control over the runtime environment on which the code will be run, and ensure that the code will run on almost any browser as expected.

Some popular transpilers for JS to JS transpilation are Webpack and Babel, but you could also use the TypeScript compiler for this.

\subsubsection{Node.js}\label{subsubsec:node}

As of the time of writing, there are mainly two ways to execute JavaScript.
You can run the program in the browser, as it was intended, or you can use a JavaScript runtime that runs on the backend, outside of the browser.
Node.js (henceforth simply referred to as Node) is the mainstream solution for the latter.
Node is a JavaScript runtime built on the JavaScript engine, V8, used by Chrome.
It is independent of the browser and can be run through a \textit{CLI} (Command-Line Interface).
One major difference from the browser runtimes is that Node also supplies some libraries for IO, such as access to the file system and the possibility to listen to HTTP requests and WebSocket events.
This makes Node a good choice for writing networking applications for instance.

We will be using the Node runtime for our compiler since it gives us access to the file system, as well as enabling the compiler to be executed through a CLI, as is the norm for most compilers.
The compiler will still also be available as a library.

\subsection{What is TypeScript?}\label{subsec:what-is-typescript}

TypeScript is a superset of JavaScript.
The language builds on JavaScript with the additions of static type definitions.
TypeScripts type system is structural, which means that the type of an object is not bound to a name, such as with nominal typing, but rather the structure of the object, such as having an attribute \codeword{i}, which may be restricted to a number.
The type system also offers some more advanced type features such as union types, where you can combine types into a new type.
The new union type represents values that can be any one of the combined types.
There are also similarly intersection types.
These types combine other types into a new type, which is the intersection of the combined types.

All valid JavaScript programs are also valid TypeScript programs.
Types in TypeScript can be optional, as the type inference is powerful enough to infer most types without writing extra code.
The type-checking can be tailored to be stricter or leaner, where you can for instance disable features such as usage of \codeword{any}-types, which are a way for the programmer to bypass the type-check for certain values.
TypeScript has full interoperability with JavaScript, so you can adopt the language without refactoring.
If you are working with a JavaScript library, but you want the safety of types, there can often be found type declaration files written by the community in the DefinitelyTyped project~\cite{tswebsite}.

