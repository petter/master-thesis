\section{Package Templates}\label{sec:package-templates}

% When modelling complex concepts like a system of cities and roads or water pipes and switches,
% it would be helpful if we had a language mechanism which could gather in a module the shared
% aspects of these problems
% e.g.\ as the concept of a graph with nodes and edges, so that this module later can be used
% to form (or could make up the kernel of) an implementation of either cities/roads or
% switches/pipes. For this language mechanism to be really helpful this requires that,
% when such a module is used (or "instantiated") in a program, we must be able to add new
% declarations at any subclass levels of the module classes, and to change names on declarations.

Krogdahl proposed Generic Packages in 2001, which is a language mechanism aimed at "large scale code reuse in object oriented languages"~\cite{krogdahl:GP}.
The idea behind this mechanism is to make modules of classes, called \textit{packages}, that could later be imported and instantiated.
This would make "textual copies" of the package body, and would also allow for further expanding the classes of the packages.
Modularizing through Generic Packages made the programming more flexible as you would easily be able to write modules with a certain functionality and be able to later import it several times when there is a need for the functionality.

Generic Packages was later extended, and the mechanism is now called Package Templates (while the textual program modules themselves are simply called templates).
The system is not fully implemented and there exists a number of proposals for extending it.

\subsection{Syntax}\label{subsec:syntax}

In this section we will look at the syntax of Package Templates (further referred to as \emph{PT}) in a Java-like language as proposed in~\cite{jot}.

\subsubsection{Defining packages and package templates}

\emph{Packages} are defined by a set of classes similar to a normal Java package.
Package templates (later just templates for short), are defined in a similar manner except for using the keyword \emph{template}.
Listing~\vref{code:basicPT} shows an example of defining packages and templates.
The contents of a package can be used as you would with a normal Java package.

\begin{code}{java}{Defining a package P and a template T}{code:basicPT}
    package P {
        interface I { ... }
        class A extends I { ... }
    }

    template T {
        class B { ... }
    }
\end{code}

\subsubsection{Instantiating templates}\label{subsubsec:inst}
Instantiating is what really makes PT useful.
When defining packages and templates, PT allows for including already defined templates through instantiating.
Instantiation is done inside the body of a package (or a template) with the use of an \codeword{inst}-clause.
Instantiating a template will make "textual copies" of the  classes, interfaces and enums from the instantiated template and insert them replacing the instantiation statement at compile time.
Note that the template itself still exists and that it can be instantiated again in the same program.
Listing~\vref{code:inst} shows an example of instantiating a template inside a package.
The resulting package \codeword{P} will then have the classes \codeword{A} and \codeword{B} from template \codeword{T} and its own class \codeword{C}.

\begin{code}{java}{Instantiating template T in package P. Note that class C can reference class A and B as if they were defined in the same package, which they essentially are after the instantiation.}{code:inst}
// Before compile time instantiation of T
template T {
    class A { ... }
    class B { ... }
}

package P {
    inst T;
    class C {
        A a;
        B b() {
            ...
        }
    }
}

// After compile time instantiation of T
package P {
    class A { ... }
    class B { ... }
    class C { ... }
}
\end{code}

\subsubsection{Renaming}\label{subsubsec:renaming}

During instantiation it is possible to rename classes (as well as interfaces and enums) and class attributes, both fields and methods.
Renaming is a part of the instantiation of templates, and will only affect the copy made for this instantiation, and it is done for the copy before it replaces the \codeword{inst}-statement.
Renaming is denoted by an optional \codeword{with}-clause at the end of the \codeword{inst}-statement.
In the \codeword{with}-clause one can rename classes using the following fat arrow syntax, \codeword{A => B}, where class \codeword{A} is renamed to \codeword{B}, and rename class attributes with a similar arrow syntax, \codeword{i -> j}, where the attribute \codeword{i} is renamed to \codeword{j}.
For method renaming you have to give the signature of the method so that it is possible to distinguish between overloaded versions, i.e. \codeword{m1(int) -> m2(int)}.
On a more technical level the compiler will find the class or attribute declaration that is going to be renamed, and then find all name occurrences bound to this declaration and rename these.

Field renaming comes after the class renaming enclosed by a set of parentheses.
Renaming classes will also affect the signatures of any methods using this class.
Listing~\vref{code:rename} shows an often used example of renaming, where a graph template is renamed to better fit a domain, in this case a road map.
When renaming the class \codeword{Node} the signature of the methods in \codeword{Edge} using this \codeword{Node} was also changed to reflect this, i.e.\ the method \codeword{Node getNodeFrom()} would become \codeword{City getNodeFrom()} with the class rename, and \codeword{City getStartingCity()} with the method renaming.

\begin{code}{java}{Example of renaming classes during instantiation. This could be used to make the classes fit the domain of the project better.}{code:rename}
template Graph {
    class Node {
        ...
    }

    class Edge {
        Node getNodeFrom() { ... }
        Node getNodeTo() { ... }
    }

    class Graph {
        ...
    }
}

package RoadMap {
    ...
    inst Graph with
        Node => City,
        Edge => Road
            (getNodeFrom() -> getStartingCity(),
            getNodeTo() -> getDestinationCity()),
        Graph => RoadSystem;
    ...
}

\end{code}

Renaming makes it possible to instantiate templates with conflicting names of classes, or even instantiate the same templates multiple times.
Listing~\vref{code:renamingdoubleinst} shows an example of this where we instantiate the same template T twice without any issues.

\begin{code}{java}{Example of instantiating the same template twice solved by renaming.}{code:renamingdoubleinst}
template T {
    class A {
        void m() { ... }
    }
}

package P {
    inst T;
    inst T with A => B;
}

// package P after compile time instantiation and renaming
package P {
    class A {
        void m() { ... }
    }

    class B {
        void m() { ... }
    }
}
\end{code}

\subsubsection{Additions to a template}\label{subsubsec:additions}

When instantiating a template you can also add attributes to the classes of the template, as well as extending the classes implemented interfaces and this will only apply to the current copy.
These additions are written inside an \codeword{addto}-clause.
Extending the class with additional attributes is done in the body of the clause, like you would in a normal Java class.
If an addition has the same signature as an already existing method from the instantiated template class, then the addition will override the existing method, similarly to traditional inheritance.
Listing~\vref{lst:simple-addto} shows an example of adding attributes an instantiated class.
The resulting class \codeword{A} in package \codeword{P} would have the field \codeword{i} and the methods \codeword{someMethod} and \codeword{someOtherMethod}.

\begin{code}{java}{Adding new attributes to the instantiated class A in package P}{lst:simple-addto}
    template T {
        class A {
            void someMethod() { ... }
        }
    }

    package P {
        inst T;
        addto A {
            int i;
            void someOtherMethod() { ... }
        }
    }
\end{code}

It is also possible to extend the list of implemented interfaces of a class by suffixing the \codeword{addto}-clause with a \codeword{implements}-clause containing the list of implementing interfaces.
Having the possibility to add implementing interfaces to classes makes working with PT easier and enables the programmer to re-use template classes to a much larger degree.
This feature's use is easier explained through an example.
Say we have implemented some class that will deal with logging.
This class can log the state of a class given the class implements some interface \codeword{Loggable}.
If want to be able to log the state of our Graph implementation, from~\vref{code:rename}, then the Graph class would need to implement the \codeword{Loggable} interface.
We can't do this at the declaration of the Graph template, as we do not have access to the interface at the time of declaration.
By using \codeword{addto} however we are able to add the \codeword{Loggable} interface and the \codeword{log} method to the Graph class.
You can also achieve the same functionality through class merging, which we will look at in section~\vref{subsubsec:merging-classes}.

\begin{code}{java}{Adding the \codeword{Loggable} interface to the Graph class from listing~\vref{code:rename}, making it compatible with our logger implementation.}{lst:addto-interface}
    template Logger {
        interface Loggable {
            String log();
        }

        class Logger {
            void log(Loggable loggable) {
                ...
            }
        }
    }

    package P {
        inst Graph;
        inst Logger;
        addto Graph implements Loggable {
            String log() {
                ...
            }
        }
    }
\end{code}

The extension of classes using the \codeword{addto}-clause is done independently for each class, ignoring any inheritance-patterns.
If there is a class \codeword{A} with a subclass \codeword{B}, they can both get extensions independently of each other.
Any extensions made to class \codeword{A} would of course still be inherited to class \codeword{B}, as with normal Java inheritance.


\subsubsection{Merging classes}\label{subsubsec:merging-classes}

If two or more classes in the same or in different instantiations share the same name they will be merged into one class.
Through this mechanism PT achieves a form of static multiple inheritance.
If two classes don't share the same name, it is still possible to force a merge through renaming them to the same name.
In listing~\vref{code:renameclassmerging} we see an example of renaming class \codeword{B} to \codeword{A} in order to merge them under the class name \codeword{A}.
Merging the classes would lead to having a single class \codeword{A} with the attributes from both classes.
The two classes \codeword{A} and \codeword{B}, from templates \codeword{T1} and \codeword{T2} respectively, no longer exists in package \codeword{P}, but have formed the new class \codeword{A}, which is a union of both.
Any pointers typed with the old \codeword{A} or \codeword{B} will now be pointing to the new merged class \codeword{A}.

\begin{code}{java}{Instantiation with class merging through renaming}{code:renameclassmerging}
template T1 {
    class A {
        ...
    }
}

template T2 {
    class B {
        ...
    }
}

package P {
    inst T1 with A;
    inst T2 with B => A;
}
\end{code}

