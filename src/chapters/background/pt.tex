\section{Package Templates}\label{sec:package-templates}

% When modelling complex concepts like a system of cities and roads or water pipes and switches,
% it would be helpful if we had a language mechanism which could gather in a module the shared
% aspects of these problems
% e.g.\ as the concept of a graph with nodes and edges, so that this module later can be used
% to form (or could make up the kernel of) an implementation of either cities/roads or
% switches/pipes. For this language mechanism to be really helpful this requires that,
% when such a module is used (or "instantiated") in a program, we must be able to add new
% declarations at any subclass levels of the module classes, and to change names on declarations.

Krogdahl proposed Generic Packages in 2001, which is a language mechanism aimed at "large scale code reuse in object oriented languages"\cite{krogdahl:GP}.
The idea behind this mechanism was to make modules of classes, called \textit{packages}, that could later be imported and instantiated.
This would make "textual copies" of the package body, and would also allow for further expanding the classes of the packages.
Modularizing through Generic Packages made the programming more flexible as you would easily be able to write modules with a certain functionality and be able to later import it several times when there is a need for the functionality.

Generic Packages was later extended \red{(nevne SWAT-project?)} and the mechanism is now called Package Templates (while the textual program modules themselves are simply called templates).
Of special interest in this essay is that the system allows a special type of multiple inheritance (by "merging" classes from different templates), and templates may build upon other templates to any depth).
The system is not fully implemented and there exists a number of proposals for extending it.

\subsection{Syntax}\label{subsec:syntax}

In this section we will look at the syntax of Package Templates (further referred to as \emph{PT}) in a Java-like language as proposed in~\cite{jot}.

\subsubsection{Defining packages and package templates}

\emph{Packages} are defined by a set of classes similar to a normal Java package.
Package templates(later just templates for short), are defined similar except for using the keyword \emph{template}.
Listing~\vref{code:basicPT} shows an example of defining packages and templates.
The contents of a package can be used as you would with a normal Java package.

\begin{code}{java}{Defining a package P and a template T. \red{Merk at dette ikke er syntaksen til en vanlig pakke i Java. Dette må beskrives.}}{code:basicPT}
    package P {
        interface I { ... }
        class A extends I { ... }
    }

    template T {
        class B { ... }
    }
\end{code}

\subsubsection{Instantiating templates}\label{sec:inst}
Instantiating is what really makes PT useful.
When defining packages, PT allows for including already defined templates through instantiating.
Instantiation is done inside the body of a package with the use of a \verb|inst|-clause.
Instantiating a template will make "textual copies" of the  classes, interfaces and enums from the instantiated template and insert them replacing the instantiation statement at compile time.
Note that the template itself still exist and that it can be instantiated again in the same program.
%Instantiating a template will effectively copy classes, interfaces and enums from the template into the body of the new package or template at compile time.
Listing~\vref{code:inst} shows an example of instantiating a template inside a package.
The resulting package \verb|P| will then have the classes \verb|A| and \verb|B| from template \verb|T| and its own class \verb|C|.

\begin{code}{java}{Instantiating template T in package P}{code:inst}
// Before compile time instantiation of T
template T {
    class A { ... }
    class B { ... }
}

package P {
    inst T;
    class C { ... }
}

// After compile time instantiation of T
package P {
    class A { ... }
    class B { ... }
    class C { ... }
}
\end{code}

\subsubsection{Renaming}

During instantiations it is possible to rename classes (as well as interfaces and enums) and class methods. Renaming is a part of the instantiation of templates, and will only affect the copy made for this instantiation, and it is done for this copy before it replaces the \verb|inst|-statement.
Renaming is denoted by a \verb|with|-clause.
In the \verb|with|-clause one can rename classes using the following fat arrow syntax, \verb|A => B|, where class \verb|A| is renamed to \verb|B|, and you can rename class attributes with a similar arrow syntax, \verb|i -> j|, where the attribute \verb|i| is renamed to \verb|j|.
For method renaming you have to give the signature of the method so that it is possible to distinguish between overloaded versions, i.e. \verb|m1(int) -> m2(int)|\footnote{On a more technical level the compiler will find the class or attribute declaration that is gonna be renamed, and then find all name occurrences bound to this declaration and rename these.}.

\red{Det var vel altså egentlig noe historisk tull at vi skilte mellom "=>" og "->". Må gjerne bruke are en, f.eks. "->", så har man "=>" helt fri til andre formål. Sparer dette til masteroppgaven, da essayet fokuserer på hvordan PT er nå, mens masteroppgaven vil fokusere mer på "min versjon" av PT.}

Attribute renaming comes after the class renaming surrounded by a set of parentheses.
Renaming classes will also affect the signatures of any methods using this class.
Listing \vref{code:rename} shows an often used example of renaming, where a graph template is renamed to better fit a domain, in this case a road map.
When renaming the class Node the signature of the methods in Edge using this Node was also changed to reflect this, i.e. the method \verb|Node getNodeFrom()| would become \verb|City getNodeFrom()| with the class rename, and \verb|City getCityFrom()| with the method renaming.

\begin{code}{java}{Example of renaming classes during instantiation. This could be used to make the classes fit the domain of the project better.}{code:rename}
template Graph {
    class Node {
        ...
    }

    class Edge {
        Node getNodeFrom() { ... }
        Node getNodeTo() { ... }
    }

    class Graph {
        ...
    }
}

package RoadMap {
    ...
    inst Graph with
        Node => City,
        Edge => Road
            (getNodeFrom() -> getStartingCity(),
            getNodeTo() -> getDestinationCity()),
        Graph => RoadSystem;
    ...
}

\end{code}

Renaming makes it possible to instantiate templates with conflicting names of classes, or even instantiate the same templates multiple times.
Listing~\vref{code:renamingdoubleinst} shows an example of this where we instantiate the same template T twice without any issues.

\begin{code}{java}{Example of instantiating the same template twice solved by renaming.}{code:renamingdoubleinst}
template T {
    class A {
        void m() { ... }
    }
}

package P {
    inst T;
    inst T with A => B;
}

// package P after compile time instantiation and renaming
package P {
    class A {
        void m() { ... }
    }

    class B {
        void m() { ... }
    }
}
\end{code}

\subsubsection{Additions to a template}\label{sec:additions}

When instantiating a template you can also add attributes to the classes of the template, as well as extending the classes implemented interfaces and this will only apply to the current copy.
These additions are written inside an \verb|addto|-clause. Extending the class with additional attributes is done in the body of the clause, like you would in a normal Java class.
If an addition has the same signature as an already existing attribute from the instantiated template class, then the addition will override the already existing attribute, like in traditional inheritance.
Extending the list of implemented interfaces for a class can be done by suffixing the \verb|addto|-clause with \verb|implements| and the list of interfaces.
Listing~\vref{code:addition} shows an example of adding attributes and implemented interfaces to an instantiated class.
The resulting class \verb|A| in package \verb|P| would have the attribute \verb|i|, methods \verb|someMethod|, \verb|someOtherMethod| and \verb|run|, as well as implementing the interface \verb|Runnable|.

The extension of classes using the \verb|addto|-clause is done independently for each class, ignoring any inheritance-patterns. If there is a class \verb|A| with a subclass \verb|B|, they can both get extensions independently from each other. Any extensions made to class \verb|A| would of course still be inherited to class \verb|B|, as with normal Java inheritance.


\begin{code}{java}{Adding new attributes and extending the implemented interface for the instantiated class A in package P}{code:addition}
template T {
    class A {
        void someMethod() { ... }
    }
}

package P {
    inst T;
        addto A implements Runnable {
        int i;
        void someOtherMethod() { ... }
        void run() { ... }
    }
}
\end{code}


\subsection{Merging classes}

Since this is the mechanism of PT that gives rise to the theme of this essay, a sort of multiple inheritance (namely "static multiple inheritance"), we will be going a bit more in-depth in this section.

If two or more classes in the same or in different instantiations share the same name they will be merged into one class.
Through this mechanism PT achieves a form of static multiple inheritance.
If two classes don't share the same name, it is still possible to merge them by first using renaming.
Through renaming it is possible to force merging by giving the classes the same name.
In listing~\vref{code:renameclassmerging} we see an example of renaming class \verb|B| to \verb|A| to force these classes to be merged under the class name \verb|A|.
Merging the classes would simply lead to having a single class \verb|A| with the attributes from both classes.
The two classes \verb|A| and \verb|B|, from templates \verb|T1| and \verb|T2| respectively, no longer exists in package \verb|P|, but have formed the new class \verb|A|, which is a union of both.
Any pointers typed with the old \verb|A| or \verb|B| will now be pointing ot the new merged class \verb|A|.

\begin{code}{java}{Instantiation with class merging through renaming}{code:renameclassmerging}
template T1 {
    class A {
        ...
    }
}

template T2 {
    class B {
        ...
    }
}

package P {
    inst T1 with A;
    inst T2 with B => A;
}
\end{code}

