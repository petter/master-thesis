%! Author = Petter
%! Date = 9/22/2020

\chapter{Background}\label{ch:background}

\section{Package Templates}\label{sec:package-templates}

% When modelling complex concepts like a system of cities and roads or water pipes and switches,
% it would be helpful if we had a language mechanism which could gather in a module the shared
% aspects of these problems
% e.g.\ as the concept of a graph with nodes and edges, so that this module later can be used
% to form (or could make up the kernel of) an implementation of either cities/roads or
% switches/pipes. For this language mechanism to be really helpful this requires that,
% when such a module is used (or "instantiated") in a program, we must be able to add new
% declarations at any subclass levels of the module classes, and to change names on declarations.

Krogdahl proposed Generic Packages in 2001, which is a language mechanism aimed at "large scale code reuse in object oriented languages"~\cite{krogdahl:GP}.
The idea behind this mechanism is to make modules of classes, called \textit{packages}, that could later be imported and instantiated.
This would make "textual copies" of the package body, and would also allow for further expanding the classes of the packages.
Modularizing through Generic Packages made the programming more flexible as you would easily be able to write modules with a certain functionality and be able to later import it several times when there is a need for the functionality.

Generic Packages was later extended, and the mechanism is now called Package Templates (while the textual program modules themselves are simply called templates).
The system is not fully implemented and there exists a number of proposals for extending it.

\subsection{Syntax}\label{subsec:syntax}

In this section we will look at the syntax of Package Templates (further referred to as \emph{PT}) in a Java-like language as proposed in~\cite{jot}.

\subsubsection{Defining packages and package templates}

\emph{Packages} are defined by a set of classes similar to a normal Java package.
Package templates (later just templates for short), are defined in a similar manner except for using the keyword \emph{template}.
Listing~\vref{code:basicPT} shows an example of defining packages and templates.
The contents of a package can be used as you would with a normal Java package.

\begin{code}{java}{Defining a package P and a template T}{code:basicPT}
    package P {
        interface I { ... }
        class A extends I { ... }
    }

    template T {
        class B { ... }
    }
\end{code}

\subsubsection{Instantiating templates}\label{subsubsec:inst}
Instantiating is what really makes PT useful.
When defining packages and templates, PT allows for including already defined templates through instantiating.
Instantiation is done inside the body of a package (or a template) with the use of an \codeword{inst}-clause.
Instantiating a template will make "textual copies" of the  classes, interfaces and enums from the instantiated template and insert them replacing the instantiation statement at compile time.
Note that the template itself still exists and that it can be instantiated again in the same program.
Listing~\vref{code:inst} shows an example of instantiating a template inside a package.
The resulting package \codeword{P} will then have the classes \codeword{A} and \codeword{B} from template \codeword{T} and its own class \codeword{C}.

\begin{code}{java}{Instantiating template T in package P. Note that class C can reference class A and B as if they were defined in the same package, which they essentially are after the instantiation.}{code:inst}
// Before compile time instantiation of T
template T {
    class A { ... }
    class B { ... }
}

package P {
    inst T;
    class C {
        A a;
        B b() {
            ...
        }
    }
}

// After compile time instantiation of T
package P {
    class A { ... }
    class B { ... }
    class C { ... }
}
\end{code}

\subsubsection{Renaming}\label{subsubsec:renaming}

During instantiation it is possible to rename classes (as well as interfaces and enums) and class attributes.
Renaming is a part of the instantiation of templates, and will only affect the copy made for this instantiation, and it is done for the copy before it replaces the \codeword{inst}-statement.
Renaming is denoted by an optional \codeword{with}-clause at the end of the \codeword{inst}-statement.
In the \codeword{with}-clause one can rename classes using the following fat arrow syntax, \codeword{A => B}, where class \codeword{A} is renamed to \codeword{B}, and you can rename class attributes with a similar arrow syntax, \codeword{i -> j}, where the attribute \codeword{i} is renamed to \codeword{j}.
For method renaming you have to give the signature of the method so that it is possible to distinguish between overloaded versions, i.e. \codeword{m1(int) -> m2(int)}\footnote{On a more technical level the compiler will find the class or attribute declaration that is gonna be renamed, and then find all name occurrences bound to this declaration and rename these.}.

Field renaming comes after the class renaming enclosed by a set of parentheses.
Renaming classes will also affect the signatures of any methods using this class.
Listing~\vref{code:rename} shows an often used example of renaming, where a graph template is renamed to better fit a domain, in this case a road map.
When renaming the class \codeword{Node} the signature of the methods in \codeword{Edge} using this \codeword{Node} was also changed to reflect this, i.e.\ the method \codeword{Node getStartingNode()} would become \codeword{City getStartingNode()} with the class rename, and \codeword{City getStartingCity()} with the method renaming.

\begin{code}{java}{Example of renaming classes during instantiation. This could be used to make the classes fit the domain of the project better.}{code:rename}
template Graph {
    class Node {
        ...
    }

    class Edge {
        Node getStartingNode() { ... }
        Node getDestinationNode() { ... }
    }

    class Graph {
        ...
    }
}

package RoadMap {
    ...
    inst Graph with
        Node => City,
        Edge => Road
            (getStartingNode() -> getStartingCity(),
            getDestinationNode() -> getDestinationCity()),
        Graph => RoadSystem;
    ...
}

\end{code}

Renaming makes it possible to instantiate templates with conflicting names of classes, or even instantiate the same templates multiple times.
Listing~\vref{code:renamingdoubleinst} shows an example of this where we instantiate the same template T twice without any issues.

\begin{code}{java}{Example of instantiating the same template twice solved by renaming.}{code:renamingdoubleinst}
template T {
    class A {
        void m() { ... }
    }
}

package P {
    inst T;
    inst T with A => B;
}

// package P after compile time instantiation and renaming
package P {
    class A {
        void m() { ... }
    }

    class B {
        void m() { ... }
    }
}
\end{code}

\subsubsection{Additions to a template}\label{subsubsec:additions}

When instantiating a template you can also add attributes to the classes of the template, as well as extending the classes implemented interfaces and this will only apply to the current copy.
These additions are written inside an \codeword{addto}-clause.
Extending the class with additional attributes is done in the body of the clause, like you would in a normal Java class.
If an addition has the same signature as an already existing method from the instantiated template class, then the addition will override the existing method, similarly to traditional inheritance.
Extending the list of implemented interfaces for a class can be done by suffixing the \codeword{addto}-clause with \codeword{implements} and the list of interfaces.
\red{Snakk om hvorfor dette er veldig viktig/nyttig.}
Listing~\vref{code:addition} shows an example of adding attributes and implemented interfaces to an instantiated class.
The resulting class \codeword{A} in package \codeword{P} would have the field \codeword{i}, methods \codeword{someMethod}, \codeword{someOtherMethod} and \codeword{run}, as well as implementing the interface \codeword{Runnable}.

The extension of classes using the \codeword{addto}-clause is done independently for each class, ignoring any inheritance-patterns.
If there is a class \codeword{A} with a subclass \codeword{B}, they can both get extensions independently of each other.
Any extensions made to class \codeword{A} would of course still be inherited to class \codeword{B}, as with normal Java inheritance.


\begin{code}{java}{Adding new attribute and extending the implemented interface for the instantiated class A in package P}{code:addition}
template T {
    class A {
        void someMethod() { ... }
    }
}

package P {
    inst T;
    addto A implements Runnable {
        int i;
        void someOtherMethod() { ... }
        void run() { ... }
    }
}
\end{code}


\subsubsection{Merging classes}\label{subsubsec:merging-classes}

If two or more classes in the same or in different instantiations share the same name they will be merged into one class.
Through this mechanism PT achieves a form of static multiple inheritance.
If two classes don't share the same name, it is still possible to force a merge through renaming them to the same name.
In listing~\vref{code:renameclassmerging} we see an example of renaming class \codeword{B} to \codeword{A} in order to merge them under the class name \codeword{A}.
Merging the classes would lead to having a single class \codeword{A} with the attributes from both classes.
The two classes \codeword{A} and \codeword{B}, from templates \codeword{T1} and \codeword{T2} respectively, no longer exists in package \codeword{P}, but have formed the new class \codeword{A}, which is a union of both.
Any pointers typed with the old \codeword{A} or \codeword{B} will now be pointing to the new merged class \codeword{A}.

\begin{code}{java}{Instantiation with class merging through renaming}{code:renameclassmerging}
template T1 {
    class A {
        ...
    }
}

template T2 {
    class B {
        ...
    }
}

package P {
    inst T1 with A;
    inst T2 with B => A;
}
\end{code}



\section{TypeScript}\label{sec:typescript}

Before we look at what TypeScript is we first need to understand JavaScript and the JavaScript ecosystem.

\subsection{JavaScript}\label{subsec:javascript}

\red{Kanskje vi også skal nevne hvorfor JavaScript er her/motivasjonen bak JavaScript? Snakke om verden før JavaScript med usikre Java Applets og Flash?}

A web page generally consists of three layers of technologies.
The first layer is HTML, which is the markup language that is used to structure the web page.
Second is CSS which gives our structured documents styling such as background colors and positioning.
The third and final layer is JavaScript which enables web pages to have dynamic content.
Whenever you visit a website that isn't just static information, but instead might have timely content updates, interactive maps, etc., then JavaScript is most likely involved~\cite{whatisjs}.

JavaScript, or more precisely ECMAScript, is the programming language of the web.
It is a language that conforms to the ECMAScript specification.
ECMAScript is simply a JavaScript standard, created by Ecma International, made to ensure interoperability across different browsers.
There is no official runtime or compiler for JavaScript as it is up to each browser to implement the languages runtime.
When we create a JavaScript program/script for a web page we don't compile it and transfer a binary or bytecode file for the web page to execute, instead the browser takes the raw source code and interprets it\footnote{On a more technical level, JavaScript is generally just-in-time compiled in the browser.}.

JavaScript is a multi-paradigm language with dynamic typing...
\red{Hvor dypt skal vi gå inn i å beskrive JavaScript?}

\subsubsection{ECMAScript Versions}

ECMAScript versions are generally released on a yearly basis.
This release is in the form of a detailed document describing the language, ECMAScript, at the time of release.
New versions will most likely include some additions to the language, but never any breaking changes\footnote{There has been occasions where there has been minor breaking changes between ECMAScript versions, but these are both rare in occurence and encounters.}.
This is because the developer will not be able to control the environment on which the code will be executed\footnote{It is possible to ship chunks of code called \textit{polyfills} with your code in order to regain some control over the enviornment. We will go more in-depth on this in section~\vref{subsubsec:backwards-compatability}.} and you can not be sure which ECMAScript version the client browser will be using.
Because of this lack of control over the runtime environment it is crucial that any pre-existing language features don't have breaking changes between versions.

\subsubsection{Backwards Compatibility}\label{subsubsec:backwards-compatability}

With new ECMAScript versions comes new features, and it is up to each browser to implement these changes.
As we mentioned earlier, we do not transfer a binary to the client browser, we transfer the source code.
So when a JavaScript script uses a new ECMAScript feature it is not guaranteed to work with every client browser, since a lot of users might have older browsers installed, or the team behind the browser has not implemented the language feature yet.
To deal with this a common practice in JavaScript development is to first transpile the source code before using it in a production environment.
This transpilation step takes the source code and transpiles it into an older ECMAScript version.
In doing this you ensure that more client browser will be able to run the script.
This will rewrite the new language features, and often replace them with a function, called a \textit{polyfill}.
You can think of a polyfill as an implementation of a new language feature.

Some popular transpilers for JS to JS transpilation are Webpack and Babel, but you could also use the TypeScript compiler for this.

\subsubsection{Node.js}

Node.js (henceforth simply referred to as Node) is a JavaScript runtime built on the JavaScript engine, V8, used by Chrome.
While most JavaScript runtimes live in the browser, Node doesn't.
It is independent of the browser and can be run through a CLI.
One major difference from the browser runtimes is that Node also supplies some libraries for IO, such as access to the file system, HTTP, and WebSocket.
This makes Node a good choice for writing networking applications for instance.

We will be using Node for our compiler since it gives us access to the file system as well as enabling the compiler to be accessible through a CLI.

\subsection{What is TypeScript?}\label{subsec:what-is-typescript}

TypeScript is a superset of JavaScript.
The language builds on JavaScript with the additions of static type definitions~\cite{tswebsite}.

All valid JavaScript programs are also valid TypeScript programs.
