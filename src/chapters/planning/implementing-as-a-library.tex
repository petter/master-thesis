\subsection{Implementing PT as an internal DSL}\label{subsec:implementing-pt-as-a-ts-library}

% Tror Groovy bruker "meta-classes" til å oppnå gjenbruk og renaming av klasser. Vi kan bruke prototyper til å oppnå lignende i JS.

One of the first approaches we need to check out is if we are able to achieve the functionality of PT, without having to create a compiler.
Instead of creating a compiler, which would likely be a complicated and time-consuming task, we could potentially get away with implementing the features of PT within TypeScript itself through making a small internal DSL\footnote{While most programming languages are made to be general purpose, DSLs, or Domain-Specific Languages, are languages created to solve very specific problems within certain domains. An internal DSL (sometimes referred to as embedded DSL) is a language based on an existing programming language, but "tailoring it [...] to the domain of interest"~\cite{Hudak1997}.}.
In~\cite{groovypt}, Axelsen and Krogdahl show how they were able to implement PT in Groovy by utilizing the languages' meta-programming capabilities.
Stordahl also showed the possibility of implementing PT in Boo, through its meta-programming capabilities in~\cite{Stordahl2011}.
We will in this section therefore see if we are able to achieve something similar in TypeScript.

Both Groovy and Boo have strong meta-programming capabilities, where they can perform transformations to the syntax tree during compilation.
JavaScript and TypeScript do not have the same capabilities for meta-programming during compilation, as JavaScript is intended as an interpreted language, and TypeScript is only supposed to offer type declarations without changing any of the underlying functionality, so any compile-time transformations are not an option for our DSL\@.
Besides the compilation transformations used for some custom syntax, Groovy gives the programmer access to the \emph{meta-object protocol}, where each object has a reference to its \emph{meta-class}, where members of a class can be added or changed at run-time.
This was utilized by Axelsen and Krogdahl during the implementation of GroovyPT, and we could seemingly achieve something similar in TypeScript.
Similar to Groovy's meta-classes, objects in JavaScript have references to a prototype object.
These prototypes can have members added or removed, or the entire prototype replaced, at runtime.
Utilizing this we could potentially be able to implement PT's class merging, renaming and additions.
We will dive deeper into this in the following sections.

To implement PT we will need to handle the following:

\begin{itemize}
    \item Defining templates
    \item Renaming classes and class attributes
    \item Instantiating templates
    \item Merging classes
\end{itemize}

\subsubsection{Defining Templates}\label{subsubsec:defining-templates}

For defining templates we would like a construct that can wrap our template classes in a scope.
We will also need to be able to reference the template.
JavaScript has three options for this, an array, an object or a class.
It should however also be possible to inherit from classes within the same template, which rules out both arrays and objects, as there is no way of referencing other members during definition of the array/object (without first defining them outside the construct).
Templates could therefore be defined as classes, where each member of the template is an attribute of the template class.
In listing~\vref{code:libraryimpl-template} we see an example of how this could be done.

\codeinputfile{template-example-without-decorator.ts}{typescript}{Example of defining a template in a library implementation.}{code:libraryimpl-template}

\subsubsection{Renaming Classes and Class Attributes}\label{subsubsec:renaming-classes-and-class-attributes}

Renaming of classes is possible to an extent.
Since we made the classes static attributes of the template class we could easily just create a new static field on the template class and use the \codeword{delete}-op~\footnote{An operator in JavaScript for removing a property of an object. See~\url{https://developer.mozilla.org/en-US/docs/Web/JavaScript/Reference/Operators/delete}.} to remove the old field.
We can see an example of this in listing~\vref{code:libraryimpl-template-renaming}.


\codeinputfile{class-renaming-example.ts}{typescript}{Example of renaming a template class}{code:libraryimpl-template-renaming}

Even though we were able to give the class a "new name", this would still not actually rename the class.
Any reference to the old names would be left unchanged, and thus we are not able to achieve renaming in TypeScript.
Listing~\vref{lst:lib-rename-problem} shows how this can be a problem, where the function \codeword{f} of class \codeword{X} would fail at run-time due to it not being able to find class \codeword{A}.

\begin{code}{typescript}{Example showcasing the problems of renaming classes in a library implementation.}{lst:lib-rename-problem}
    class T1 {
        static A = class {
            i = 0;
        }
        static X = class {
            f() {
                return new A();
            }
        }
    }

   // Renaming
    const classRef = T1.A;
    T1.B = classRef;
    delete T1.A;

    // Trying to use the template after renaming
    const x = new T1.X();
    x.f(); // ReferenceError: A is not defined
\end{code}

Attribute renaming can be done similarly, where we could alter the prototype of the class to rename attributes, however this would also lead to the same problems as with class renaming, where references would not be changed.

\subsubsection{Instantiating Templates}\label{subsubsec:instantiating-templates}

As with renaming, we are also able to instantiate templates to an extent.
We are able to iterate over the attributes of the template class, and populate a package/template with references to the template.
An example of this can be seen in listing~\vref{code:libraryimpl-template-inst}.

\codeinputfile{inst-template-example.ts}{typescript}{Example of instantiating a template in a library implementation}{code:libraryimpl-template-inst}

The instantiation will only contain references to the instantiated templates classes, while PT instantiations make textual copies of the templates content.
Only having references to the original template could mean that if a template that has been instantiated is later renamed, then the instantiated template might lose some of its references.
We could possibly circumvent this by getting the textual representation of the class, through the class' \codeword{toString}, and then use \codeword{eval} to evaluate the class declaration.

\subsubsection{Merging Classes}

For merging of types we would use the built-in declaration merging~\cite{declerationmerging}.
Implementation merging is also possible because JavaScript has open classes.
For implementation merging you would create an empty class which has the type of the merged declarations, and then assign the fields and methods from the merging classes to this class.
There are several libraries that supports class merging, such as mixin-js\footnote{\url{https://www.npmjs.com/package/mixin-js}}.

