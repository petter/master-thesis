%! Author = Petter
%! Date = 9/22/2020

\chapter{Planning the Project}\label{ch:planning-the-project}

Before we start the implementation of our language we first need to do some planning.
We know we are going to be creating a programming language, a superset of TypeScript with the addition of Package Templates.
However, we might want to look at if creating a superset of TypeScript is the way to go, or if keeping it simple and extending JavaScript is a better call.
There are a lot of approaches we can take for implementing our language, so we will have map out the requirements for our approach.
Lastly we will have to look at all the different approaches we can take, and discuss the pros and cons of each approach.

This planning phase is crucial for the success of the project, as starting off on the wrong approach for the wrong language would set us back immensely.

\section{TypeScript vs. JavaScript}\label{sec:typescript-vs-javascript}

When extending TypeScript you might be asking yourself if it is truly necessary, is it better to keep it simple and just extend JavaScript instead?
This is something we need to find out before going any further with the planning of our project.

\subsection{Typechecking templates}\label{subsec:why-typescript-verifying-templates}

One of the requirements of PT is that it should be possible to type-check each template separately.
There is no easy way to type-check JavaScript code without executing it and looking for runtime errors.
Even if some JavaScript program successfully executes without throwing any errors, we can still not conclude that the program does not contain any type errors.
TypeScript on the other hand, with the language being statically typed, we can, at least to a much larger extent, verify if some piece of code is type secure.
Because of this trait TypeScript is the better candidate for our language.

Now it should be noted that due to TypeScripts type system being unsound one could argue that this requirement of PT is not met.
While this is true it still outperforms JavaScript on this remark, and we will later in section~\vref{subsec:implementation-collection-level-type-checking} discuss more in-depth to what extent this requirement is met.

\section{What Do We Need?}\label{sec:what-do-we-need}

There are a lot of approaches one can take when working with TypeScript, however due to the nature of this project there are some restrictions we have to abide by.
Our approach should allow the following:

\begin{itemize}
    \item The ability to add custom syntax (access to the tokenizer / parser)
    \item Enable us to do semantic analysis.
\end{itemize}

In addition to these we would also like to look for some other desirable traits for our implementation:

\begin{itemize}
    \item Loosely coupled implementation (So that new versions of typescript not necessarily breaks our implementation).
    %\item Mer? % TODO
\end{itemize}





\section{Choosing the right approach}\label{sec:choosing-the-right-approach}

Before jumping into a project of this magnitude it is important to find out what approach to use. 
The goal of this project is to extend TypeScript with the Package Templates language mechanism, this could be achieved by one of the following methods:

\begin{itemize}
    \item Making a fork of the TypeScript compiler
    \item Making a preprocessor for the TypeScript compiler
    \item Making a compiler plugin / transform
    \item Making a custom compiler from scratch
\end{itemize}


\subsection{Preprocessor for the TypeScript Compiler}\label{subsec:preprocessor-for-the-typescript-compiler}

More work than ex plugin / transformer.

\subsection{TypeScript Compiler Plugin / Transform}\label{subsec:typescript-compiler-plugin}

At the time of writing the official TypeScript compiler does not support compile time plugins.
The plugins for the TypeScript compiler is, as the TypeScript compiler wiki specifies, "for changing the editing experience only"~\cite{tscplugin}.
However, there are alternatives that do enable compile time plugins / transformers;

\begin{itemize}
    \item ts-loader~\cite{tsloadergithub}, for the webpack ecosystem
    \item Awesome Typescript Loader~\cite{awesometypescriptloadergithub}, for the webpack ecosystem. Deprecated
    \item ts-node~\cite{tsnodegithub}, REPL / runtime
    \item ttypescript~\cite{ttypescriptgithub}, TypeScript tool TODO: Les mer på dette
\end{itemize}

Unfortunately ts-loader, Awesome Typescript Loader and ts-node does not support adding custom syntax, as it only transforms the AST produced by the TypeScript compiler.
Because of this they are not a viable option for our use-case and will therefore be discarded.

\subsection{Babel plugin}\label{subsec:babel-plugin}

Babel isn't strictly for TypeScript, but for JavaScript as a whole, however we could write our plugin to be dependent on the TypeScript transformation plugin.

Making a Babel plugin will make it very accessible as most web-projects use Babel, and the upkeep is cheap, as plugins are loosely coupled with the core.

In order for a Babel plugin to support custom syntax it has to provide a custom parser, a fork of the Babel parser.
Through this we can extend the TypeScript syntax with our syntax for PT.
This is all hidden away from the user, as this custom parser is a dependency of our Babel plugin.

Seeing as we have to make a fork of the parser in order to solve our problem, the upkeep will not be as cheap as first anticipated.
However, being able to have most of the logic loosely coupled with the compiler core it will still make it easier to keep updated than through a fork of the TypeScript compiler.

% TODO: Er det støttet å bruke flere plugins med forskjellige parsere?
% E.g. babel-plugin-typescript + vårt babel plugin?

\subsection{TypeScript Compiler Fork}\label{subsec:typescript-compiler-fork}

Possible, however not as accessible as other alternatives and will make upkeep expensive.

The TypeScript compiler is a monolith.
It has about 2.5 million lines of code, and therefore has a quite steep learning curve to get into.
If we were to go with this route it would be quite hard to keep up with the TypeScript updates, as updates to the compiler might break our implementation.
However, as we have seen, going the plugin / transform route also requires us to fork the underlying compiler and make changes to it, however with the majority of the implementation being loosely coupled it would still make it easier to keep up-to-date.
That being said it will probably be a lot easier to do semantic analysis in a fork of the TypeScript compiler vs in a plugin / transform.
