\chapter{Concluding Remarks}\label{ch:results}

In this thesis we have discussed the approach to the project and the implementation of the compiler for the PTS language, as well as evaluated the implementation and discussed the consequences of having a structurally typed language as the host of PT\@.
We will conclude this work by returning to our initial research questions and trying to answer these with the knowledge we have accumulated through the span of this thesis.
After this we will have a look at what could have been done differently in retrospect, and finish off with proposing future works within the field.


\section{Addressing Research Questions}\label{sec:adressing-research-questions}


\subsubsection{RQ1: How does the language mechanism, Package Templates, fit into TypeScript?}

Package Templates benefits greatly from TypeScript's ecosystem.
Having access to npm makes the compiler for PTS as easily accessible as the TypeScript compiler, which makes trying out the language mechanism easy.
Publishing packages to npm is openly available for anyone.
This means that anyone could utilize npm to create and publish PT modules, which greatly enforces the overall theme of PT, namely re-use.
For instance, if someone writes a great graph template implementation, this template could then be published to npm, and others would be able to install this template and re-use it in their PTS projects (given a more complete implementation of the compiler which would allow multi-file projects).

It is worth noting, as we discussed in section~\vref{subsec:which-better-fits-pt?}, the problem of renaming in TypeScript is something that, at least for our proposal of PTS, did not fit too nicely.
Since types that a class conform to are not necessarily tied to the class, this makes what would have been explicit references to attributes in a nominally typed language, not necessarily be explicit references in a structurally typed language.
The consequence of this is that we could end up in situations where renaming an attribute of a class could break the type-safety of a program, since we would not be able to rename all "references" to the classes attributes, as they would essentially not be references.
As we touched upon in the aforementioned section, it is possible to end up in similar situations in nominally typed languages, however in these scenarios it is also possible to give an informative error to the programmer, while in structurally typed languages, we might not be able to do the same.

It could possibly be viable to change the grammar of TypeScript further to disallow inline types, so we could at least fix these issues when we stumble upon them through performing renamings on the types as well.
However, doing this would greatly take away from the overall appeal of the language.
It might just be better to let the programmer have the choice of using interfaces for these problematic inline-types if they so chooses.

\subsubsection{RQ2: Does structural typing change how the core of Package Templates works?}

Most of the functionality of Package Template stay the same in a structurally typed language as they do in a nominally typed language.
The most notable deviation is the renaming mechanism.
While we still rename all valid references, it is worth noting that what a reference is will differ in a structurally typed language.

In nominal PT, all variables that are instantiated with an object of a class, will also have some explicit relation to the class.
A variable in nominal PT will likely have either the class (or a superclass) as the type, or an interface.
For the case of the variable having the class as the type the relation to itself is obviously there.
For interface or superclass as the type the relation must explicitly be declared in the class' heritage.
If the class' attributes are renamed we will either be able to rename the variables references to these attributes if the renaming was applied to the class or any of its superclasses, or at least give a detailed error message for when the renaming was applied to the interface.

In structural PT, variables instantiated with an object of a class might not have the same explicit relation to the constructed class.
A variable in a structurally typed language might be typed with a structure that the class conforms to.
However, this conformity might break after a renaming has been applied to the class.
In these cases we will not be able to rename the variables type, nor any of the variables references to the possibly renamed attributes, since these do not have the same explicit relation to the class.

\subsubsection{RQ3: Will having PT in a structurally typed language have any notable advantages or disadvantages over having it in a nominally typed language?}

As we discussed in section~\vref{subsec:which-better-fits-pt?} there are some benefits of hosting PT in a structurally typed language.
Structural typing does fit nicely in with the theme of Package Templates, namely re-use.
A structural type system gives the programmer a flexibility that nominal type systems can not offer to the same extent.
One of the strongest points for structural typing is how it enables us to easily use third-party libraries without necessarily having to alter our classes.
Say we have implemented a graph library in PT and later on wanted to use some third-party graph utility library.
This library might take a graph as input and do some calculations such as finding the shortest path between nodes in the graph.
In structural typing, as long as our graph implementation is structurally equal to the type of the input we can pass it with no problems.
In a nominally typed language we might have to alter our implementation to explicitly declare that our classes implement some interfaces from the library.
It might at worst not even be possible if the graph utility library has typed their input with classes instead of interfaces.
This is of course one of the strengths of PT in the first place, being able to add implementing interfaces to a class without altering the original implementation or merging classes if possible, however the fact that we would not have to perform this step in a structurally typed language could arguably be seen as a benefit.

The disadvantage of having PT in a structurally typed language is as we have mentioned earlier, the problem of renaming.
However, I would argue that the advantages that structurally typed languages bring to PT, outweighs the disadvantage.
The problem of renaming is mostly the problem of occasionally getting into situations where it is harder to inform the programmer about the error that has been made.
For these scenarios it is still possible for the programmer to work their way around it through using interfaces instead of inline-types, while the advantage of greater re-use is not something that can be as easily gained in a nominally typed language.

\section{In Retrospect}\label{sec:in-retrospect}

\subsection{Approach}\label{subsec:result-approach}

While we were able to implement most of PT with our approach, I fear that further development to reach a complete implementation might be hindered by our chosen approach.
This is largely because we might have to implement much of the type system in order to properly identify references, as we briefly discussed in section~\vref{subsec:supporting-all-references}.
Instead of re-inventing the wheel, we might be better off by implementing PT in a fork of the TypeScript compiler.
While this \textit{might} make updates harder, than with our implementation, we will most often likely be able to merge the changes to the TypeScript compiler into our fork with the help of Git, conflict free.
Greater changes to the language might of course still give us some merge conflicts, however these larger changes could also make our currently used approach break.

\section{Future Work}\label{sec:future-work}

\subsection{Finishing the PTS Compiler}\label{subsec:finishing-the-pts-compiler}

As detailed in section~\vref{sec:completing-the-implementation} we have a working compiler for PTS, however the implementation is not complete.
The majority of the work of completing the implementation will presumably lie in performing more advanced semantic analysis in order to correctly identify all references.
This could as we have pointed out either be attempted in a fork of the TypeScript compiler, or as a continuation of this compiler.
If one is to continue on with this implementation it could be worth looking into if it is possible to use the TypeScript compiler API in order to identify references.

\subsection{Improve the Compilers Error Messages}\label{subsec:compiler-with-focus-on-error-messages}

As I mentioned shortly in subsection~\vref{subsec:the-ast-nodes} Tree-sitter does have support for giving position of a syntax node, and this could be utilized to produce better error messages.
In addition to giving the position of where the error occurred, it would also be helpful to give more informative error messages than we currently do.
In our implementation errors are usually first found during the type-check phase, where we might have already instantiated templates, and renamed classes and attributes.
We should try to make more effort to spot errors earlier rather than later.

\subsection{Making Syntax Highlighting for the PTS Language}\label{subsec:making-syntax-highligthing-for-the-pts-language}

By utilizing Tree-sitter as our lexer/parser we could presumably pretty easily also utilize it to get syntax highlighting.
Most modern editors and IDEs have recently been switching to Tree-sitter for syntax highlighting, opposed to the traditional method of using regex to highlight code files.
This should be as simple as writing query files for identifying keywords, operators, etc.
Similar to how we extended the TypeScript grammar in order to create our PTS grammar, it should also be possible to extend the TypeScript highlight query files to create syntax highlighting for PTS\@.
In the GitHub repository for the Tree-sitter grammar for PTS there has been done some initial work for setting this up, but has been abandoned as implementing features for the compiler was more urgent.
