\chapter{Results}\label{ch:results}

\begin{code}{typescript}{"Real world example" of PTS being used in React. In this example we create a component for fetching JSON when the component has been mounted, and then we re-use this functionality in our Pokemon-component. Since external superclasses are not allowed in PT we will have to assume that we have a template \codeword{Component}, which contains the \codeword{Component}-class.}{lst:pts-react}
    template FetchJson {
        class FetchJson {
            componentDidMount() {
                fetch(this.props.url)
                    .then(response => response.json())
                    .then(data =>
                        this.setState(state => ({...state, data})))
            }
        }
    }

    pack Pokemon {
        inst Component;
        inst FetchJson { FetchJson -> Pokemon };
        addto Pokemon extends Component {
            render() {
                if(this.state.data === undefined)
                    return 'Loading...';

                const name = this.state.data.name;
                const image = this.state.data.sprites.front_default;
                return (
                    <div>
                        <img src={image} />
                        <h1>{name}</h1>
                    </div>
                )
            }
        }

        class App extends Component {
            render() {
                <Pokemon url="https://pokeapi.co/api/v2/pokemon/ditto" />
            }
        }
    }
\end{code}

\section{Future Works}\label{sec:future-works}

\subsection{Diamond Problem}\label{subsec:diamond-problem}

essay

\subsection{Improve the Compilers Error Messages}\label{subsec:compiler-with-focus-on-error-messages}

As I mentioned shortly in subsection~\vref{subsec:the-ast-nodes} tree-sitter does have support for giving position of a syntax node, and this could be utilized to produce better error messages.
