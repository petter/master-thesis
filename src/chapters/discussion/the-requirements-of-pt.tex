% TODO: Kanskje bedre å ha denne inn i background til PT?
\subsection{The Requirements of PT}\label{subsec:the-requirements-of-pt}

In~\cite{jot} the authors discuss requirements of a desired language mechanism for re-use and adaptation through collections of classes.
They then present a proposal for Package Templates, which to a large extent fulfills all the desired requirements.
% In order for us to be satisfied to call our implementation of the language mechanism Package Templates, we will also have to meet these requirements.
These requirements can therefore be used to evaluate our implementation and determine whether our implementation can be classified as a valid implementation of Package Templates.

The requirements presented in the paper were the following:

\begin{multicols}{2}
    \begin{itemize}
        \item Parallel extension
        \item Hierarchy preservation
        \item Renaming
        \item Multiple uses
        \item Type parameterization
        \item Class merging
        \item Collection-level type-checking
    \end{itemize}
\end{multicols}

In order to get a better understanding of what these requirements entail we will have to dive a bit deeper into each requirement.

\subsubsection{Parallel Extension}\label{subsubsec:parallel-extension}
% - \textbf{Parallel  extension:}  When  using  the  collection  C  in  a  certain  setting  we  can  add  attributes  to  A  and  B.
% These  additions  should  also  have  effect  for  the  code  of  C,  e.g.  so  that  we  by  means  of  an  A-variable  defined  in  C  can  directly  (without  casting) access the attributes added to A.
%
% - - In the graph example, assume that we have added the int variable length to Edge, and  that  n  is  a  Node-reference.
% With  this  property  we  can  conveniently  specify  directly: “n.firstEdge().length = 5;”, as firstEdge is typed with the extended Edge class.

The parallel extension requirement is about making additions to classes in a package/template, and being able to make use of them in the same collection.
What this means is that if we are making additions to a class, then we should be able to reference these additions in a declaration or in an addition to a separate class within the same package/template, without needing to cast it.
We can see an example of this in listing~\vref{lst:parallel-extension}, where an addition to class \codeword{B} is referencing the added method of class \codeword{A}.
Contrast this to how an implementation with traditional extension through subclasses of A and B would work, and the casts etc. necessary to make the call \codeword{a.someMethod()} work in that context.
The order of additions does not affect the parallel extension.

\begin{code}{Java}{Example of parallel extension in PT. Here we make additions to both \codeword{A} and \codeword{B} in our instantiation in package \codeword{P}, and we are able to reference the additions done to \codeword{A} in our addition to \codeword{B}. This is done without the need to cast \codeword{A}, as if the additions were present at the time of declaration.}{lst:parallel-extension}
    template T {
        class A {
            ...
        }

        class B {
            A a = new A();
            ...
        }
    }

    package P {
        inst T;

        addto A {
            void someMethod() {
                ...
            }
        }

        addto B {
            void someOtherMethod() {
                a.someMethod();
            }
        }
    }
\end{code}

\subsubsection{Hierarchy Preservation}\label{subsubsec:hierachy-preservation}
% - \textbf{Hierarchy  preservation:}  The  mechanism  should  allow  B  to  be  a  subclass  of  A,  and  if  additional  attributes  are  given  to  A  and  to  B,  then  the  B  with  additions  should be a subclass of the A with additions.
% Note that this will not be the case if we just use the collection C with the classes A and B and then define subclasses A’  and  B’  to  A  and  B,  respectively,  with  the  additions  we  want  in  these  subclasses.
% B’ will then not be a subclass of A’.
%
% - - In  the  compiler  example  this  is  exactly  what we need in order to be able to add attributes as explained in the example.
PT should never break the inheritance hierarchy of its contents.
If we have a template with classes \codeword{A} and \codeword{B}, and class \codeword{B} is a subclass of class \codeword{A}, then this relation should not be affected by any additions or merges done to either of the classes.
That is if we make additions to class \codeword{B} it should still be a subclass of class \codeword{A}, and any additions made to class \codeword{A} should be inherited to class \codeword{B}.
Even if we make additions to both class \codeword{A} and \codeword{B}, then \codeword{B} with additions should still be a subclass of class \codeword{A} with additions.

\subsubsection{Renaming}\label{subsubsec:renaming2}

% - \textbf{Renaming:}  When  C  is  used,  we  should  be  able  to  change  the  name  of  A  and  B,  and of their attributes, so that they fit with the specific use situation.
%
% - - For  the  graph  example,  the  renaming  property  makes  it  possible  to  rename  the  nodes and edges to cities and roads.

The renaming requirement states that PT should enable us to rename the names of any class, and its attributes, so that they better fit their use case.

\subsubsection{Multiple Uses}\label{subsubsec:multiple-uses}
% - \textbf{Multiple  uses:}  It  should  be  possible  to  use  the  classes  of  C  for  different  and  independent  purposes  in  the  same  program,  and  so  that  each  purpose  have  different additions and renamings.
% The compiler should be able to check that each use implies a different set of classes as if they are defined in separate hierarchies.
%
% - - In  a  program  we  may  need  the  basic  graph  structure  for  different  purposes.
PT should be allow us to use packages/templates multiple times for different purposes in the same program, and any additions or renamings should not affect any of the other uses.
Each use should be kept independent of each other.
This is an important requirement of PT as when we create a package or a template it is often designed to be re-used.
An example of this is the graph template we created in listing~\vref{code:rename}.
Here we bundled the minimal needed classes in order to have a working implementation for graphs.
We then used this graph implementation to model a road system, however we might later also want to re-use the graph implementation for modelling the sewer systems of each city, and this should not be affected by any changes we made to the graph template for our road system.

\subsubsection{Type Parameterization}\label{subsubsec:type-parameterization}

% - \textbf{Type parameterization:} It should be possible to write a collection of classes that assumes the existence of classes that have some required attributes, but are not yet completely defined.
% In each use of this collection, one can provide specific classes that have at least the required attributes.
%
% - - In the compiler example we assume that the front end shall always produce Java Bytecode,  and  we  use  some  readymade  mechanism  for  packing  the  code  to  a  correctly  formatted  class  file.
% However,  a  number  of  such  packers  may  be  available,  and  we  do  not  want  to  choose  which  to  use  while  implementing  the  front end class collection.

%\red{Dette kan kanskje heller være en del av PT background. Skrev dette basert på JOT, men er det slik at PT nå bruker required types istedenfor? Da burde dette kanskje endres noe}

The requirement of type parameterization of templates works similar to how type parameterization for classes works in Java.
Type parameterization in Java enables the programmer to assume the existence of a type during declaration of a class, and the actual type can be given when a new object of the class is created.
Type parameters in Java can have constraints where the concrete type must extend another class or interface.
Similarly, type parameterization in PT also enables the programmer to assume the existence of a type, however here the type parameter is accessible to the whole template.
PT type parameters can also be constrained similar to Java, by giving a nominal type it should extend, however PT also allows for type constraining through structural types, by giving a structure the type should conform to.

In listing~\vref{lst:type-parameterization} we see an example of how type parameterization can be used to implement a list.
Type parameterization in PT might feel similar to how you would use it in regular Java, however having the type parameter at the template level, where Java has it at class level, does have some advantages.
One advantage of having the type parameter at template level is that you don't need to specify the actual parameter again after instantiation.
At instantiation of the \codeword{ListsOf} template we can give i.e. a \codeword{Person} class, containing some information about a persons name, date of birth, etc., as the actual parameter, and then we would not have to keep specifying the actual parameter for every reference.
Another advantage of using type parameters at the template level is that the type parameter can be used by all classes in the template.
If we wanted to implement this in Java we would either have to have type parameters for both classes, or the \codeword{AuxElem} class would need to be an inner class of the \codeword{List} class.

\begin{code}{Java}{Modified example from~\cite{jot} where type parameterization is used to create a list implementation.}{lst:type-parameterization}
    template ListsOf {
        required type E { }
        class List {
            AuxElem first, last;
            void insertAsLast(E e) { ... }
            E removeFirst() { ... }
        }
        class AuxElem {
            AuxElem next;
            E e; // Reference to the real element
        }
    }
\end{code}

\subsubsection{Class Merging}\label{subsubsec:class-merging}

PT should allow for merging two or more classes.
When merging classes the result should be a union of their attributes.
If we merge two classes \codeword{A} and \codeword{B}, it should be possible to reach all of \codeword{B}'s attributes from an \codeword{A}-variable, and vice versa.

% - \textbf{Class  merging:}  Assume  we  have  the  two  collections  C  (with  classes  A  and  B)  and D (with class E).
% When they are both used in the same program, we should be able to merge e.g. the classes A and E so that the resulting class gets the union of the attributes, and so that we via an E-variable defined within D can also directly see the A attributes (and similarly for an A-variable in C).
%
% - - Assume that we in addition to the graph collection have a collection Persons with a class Person.
% In a program handling personal relations we then want to use both collections together so that we obtain a new class, say PersonNode, which has all the  attributes  of  Node  and  Person,  and  where  a  Person-variable  p  defined  in  Persons gives access directly to the Node attributes, e.g. “p.firstEdge”.

\subsubsection{Collection-Level Type-Checking}\label{subsubsec:collection-level-type-checking}

The final requirement is collection-level type-checking.
This requirement is there to ensure that each separate package/template can be independently type-checked.
By having the possibility to type-check each pacakge/template we can also verify that the produced program is also type-safe, as long as the instantiation is conflict-free.

% In  addition  to  these  properties,  it  is  important  that  such  collections  of  classes  can  be  separately type-checked.
% We also prefer a mechanism that allows only single inheritance, as  the  merge  property  described  above  to  a  large  extent  will  take  care  of  the  need  for  combining  code  from  different  sources  (for  which  purpose  multiple  inheritance  is  often  used).
% Finally, the type system should be as simple and intuitive as possible.
