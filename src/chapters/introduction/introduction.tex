%! Author = Petter
%! Date = 9/22/2020

\chapter{Introduction}\label{ch:introduction}

In this thesis we will be looking at the language mechanism Package Templates and the language TypeScript and how this language mechanism fits into the language.

Package Templates

\section{What is PT?}\label{sec:what-is-pt?}


\section{Purpose of Implementing PT in TS}\label{sec:purpose-of-implementing-pt-in-ts}


\section{Structure of Thesis}\label{sec:structure-of-thesis}

In part one of this thesis we will focus on getting to know the language mechanism Package Templates, and get a proper understanding of the programming language TypeScript and its ecosystem in chapter~\vref{ch:background}.

With a proper understanding of these topics we will move on to the second part of the thesis.
This part will focus on the implementation of the project.
The part starts off by planning out and designing the language PTS in chapter~\vref{ch:the-language---pts}.
PTS, short for Package Template Script, will be a superset of TypeScript with a basic implementation of the language mechanism PT.
Chapter~\vref{ch:planning-the-project} will then focus on making a plan for the implementation of our programming language.
Here we will perform a requirement analysis of the project and seek out an approach for the implementation.
With a plan for approach and execution we will discuss the resulting implementation of the compiler in chapter~\vref{ch:implementation}.

In the final part of this thesis we will discuss the resulting implementation.
This discussion starts of in chapter~\vref{ch:does-pts-fulfill-the-requirements-of-pt?} where we will be examining the implementation, looking at whether it fulfills the requirements of PT.
After this we will be comparing our implementation of Package Templates to another implementation of PT in Java in chapter~\vref{ch:difference-between-pts-and-ptj}, looking at the advantages of having PT in a structurally typed language.
% TODO snakke om result kapittel?
