\chapter{Does PTS Fulfill The Requirements of PT?}\label{ch:does-pts-fulfill-the-requirements-of-pt?}

This thesis is concerned about implementing Package Templates in TypeScript.
However, in order to determine to what degree we have actually implemented PT or just created something that looks like it, we have to understand what the requirements of PT are, and if we are meeting those requirements.
We will therefore in this chapter look at the requirements as described in~\cite{jot}.
After getting an understanding of the requirements we are going to look at how our implementation holds up to them.

% TODO: Kanskje bedre å ha denne inn i background til PT?
\section{The Requirements of PT}\label{sec:the-requirements-of-pt}

In~\cite{jot} the authors discuss requirements of a desired language mechanism for re-use and adaptation through collections of classes.
They then present a proposal for Package Templates, which to a large extent fulfills all the desired requirements.
% In order for us to be satisfied to call our implementation of the language mechanism Package Templates, we will also have to meet these requirements.
These requirements can therefore be used to evaluate our implementation and determine whether our implementation can be classified as a valid implementation of Package Templates.

The requirements presented in the paper were the following:

\begin{multicols}{2}
    \begin{itemize}
        \item Parallel extension
        \item Hierarchy preservation
        \item Renaming
        \item Multiple uses
        \item Type parameterization
        \item Class merging
        \item Collection-level type-checking
    \end{itemize}
\end{multicols}

In order to get a better understanding of what these requirements entail we will have to dive a bit deeper into each requirement.

\subsection{Parallel Extension}\label{subsec:parallel-extension}
% - \textbf{Parallel  extension:}  When  using  the  collection  C  in  a  certain  setting  we  can  add  attributes  to  A  and  B.
% These  additions  should  also  have  effect  for  the  code  of  C,  e.g.  so  that  we  by  means  of  an  A-variable  defined  in  C  can  directly  (without  casting) access the attributes added to A.
%
% - - In the graph example, assume that we have added the int variable length to Edge, and  that  n  is  a  Node-reference.
% With  this  property  we  can  conveniently  specify  directly: “n.firstEdge().length = 5;”, as firstEdge is typed with the extended Edge class.

The parallel extension requirement is about making additions to classes in a package/template, and being able to make use of them in the same collection.
What this means is that if we are making additions to a class, then we should be able to reference these additions in a declaration or in an addition to a separate class within the same package/template, without needing to cast it.
We can see an example of this in listing~\vref{lst:parallel-extension}, where an addition to class \codeword{B} is referencing the added method of class \codeword{A}.
The order of additions does not affect the parallel extension.
We could just as easily switched the positions of the additions around in this example.

\begin{code}{Java}{Example of parallel extension in PTj. Here we make additions to both \codeword{A} and \codeword{B} in our instantiation in package \codeword{P}, and we are able to reference the additions done to \codeword{A} in our addition to \codeword{B}. This is done without the need to cast \codeword{A}, as if the additions were present at the time of declaration.}{lst:parallel-extension}
    template T {
        class A {
            ...
        }

        class B {
            A a = new A();
            ...
        }
    }

    package P {
        inst T;

        addto A {
            void someMethod() {
                ...
            }
        }

        addto B {
            void someOtherMethod() {
                a.someMethod();
            }
        }
    }
\end{code}

\subsection{Hierarchy Preservation}\label{subsec:hierachy-preservation}
% - \textbf{Hierarchy  preservation:}  The  mechanism  should  allow  B  to  be  a  subclass  of  A,  and  if  additional  attributes  are  given  to  A  and  to  B,  then  the  B  with  additions  should be a subclass of the A with additions.
% Note that this will not be the case if we just use the collection C with the classes A and B and then define subclasses A’  and  B’  to  A  and  B,  respectively,  with  the  additions  we  want  in  these  subclasses.
% B’ will then not be a subclass of A’.
%
% - - In  the  compiler  example  this  is  exactly  what we need in order to be able to add attributes as explained in the example.
PT should never break the inheritance hierarchy of its contents.
If we have a template with classes \codeword{A} and \codeword{B}, and class \codeword{B} is a subclass of class \codeword{A}, then this relation should not be affected by any additions or merges done to either of the classes.
That is if we make additions to class \codeword{B} it should still be a subclass of class \codeword{A}, and any additions made to class \codeword{A} should be inherited to class \codeword{B}.
Even if we make additions to both class \codeword{A} and \codeword{B}, then \codeword{B} with additions should still be a subclass of class \codeword{A} with additions.

\subsection{Renaming}\label{subsec:renaming}

% - \textbf{Renaming:}  When  C  is  used,  we  should  be  able  to  change  the  name  of  A  and  B,  and of their attributes, so that they fit with the specific use situation.
%
% - - For  the  graph  example,  the  renaming  property  makes  it  possible  to  rename  the  nodes and edges to cities and roads.

The renaming requirement states that PT should enable us to rename the names of any class, and its attributes, so that they better fit their use case.

\subsection{Multiple Uses}\label{subsec:multiple-uses}
% - \textbf{Multiple  uses:}  It  should  be  possible  to  use  the  classes  of  C  for  different  and  independent  purposes  in  the  same  program,  and  so  that  each  purpose  have  different additions and renamings.
% The compiler should be able to check that each use implies a different set of classes as if they are defined in separate hierarchies.
%
% - - In  a  program  we  may  need  the  basic  graph  structure  for  different  purposes.
PT should be allow us to use packages/templates multiple times for different purposes in the same program, and any additions or renamings should not affect any of the other uses.
Each use should be kept independent of each other.
This is an important requirement of PT as when we create a package or a template it is often designed to be reused.
An example of this is the graph template we created in listing~\vref{code:rename}.
Here we bundled the minimal needed classes in order to have a working implementation for graphs.
We then used this graph implementation to model a road systems, however we might later also want to reuse the graph implementation for modelling the sewer systems of each city, and this should not be affected by any changes we made to the graph template for our road system.

\subsection{Type Parameterization}\label{subsec:type-parameterization}

% - \textbf{Type parameterization:} It should be possible to write a collection of classes that assumes the existence of classes that have some required attributes, but are not yet completely defined.
% In each use of this collection, one can provide specific classes that have at least the required attributes.
%
% - - In the compiler example we assume that the front end shall always produce Java Bytecode,  and  we  use  some  readymade  mechanism  for  packing  the  code  to  a  correctly  formatted  class  file.
% However,  a  number  of  such  packers  may  be  available,  and  we  do  not  want  to  choose  which  to  use  while  implementing  the  front end class collection.

%\red{Dette kan kanskje heller være en del av PT background. Skrev dette basert på JOT, men er det slik at PT nå bruker required types istedenfor? Da burde dette kanskje endres noe}

The requirement of type parameterization of templates works similar to how type parameterization for classes works in Java.
Type parameterization in Java enables the programmer to assume the existence of a class during declaration, and the class can later be given at the time of instantiation.
Type parameters in Java can have a constraint that it must extend another class.
Similarly, type parameterization in PT also enables the programmer to assume the existence of a class, however here the type parameter is accessible to the whole template.
PT type parameters can also be constrained, however in PT type constraining uses structural typing instead of nominal typing.
What this means is that instead of declaring what the type parameter must extend it instead declares what structure the type parameter must conform to.

In listing~\vref{lst:type-parameterization} we see an example of how type parameterization can be used to implement a list.
This example might not look too different from what you would do with type parameterization in Java, however having the type parameter at the template level does have some advantages.
One advantage of having the type parameter at template level is that you don't need to specify the actual parameter again after the instantiation.
At instantiation of the \codeword{ListsOf} template we can give i.e. a \codeword{Person} class, containing some information about a persons name, date of birth, etc., as the actual parameter, and then we would not have to keep specifying the actual parameter for every reference.
Another advantage of using type parameters at the template level is that the type parameter can be used by all classes in the template.
If we wanted to implement this in Java we would either have to have type parameters for both classes, or the \codeword{AuxElem} class would need to be a inner class of the \codeword{List} class.

\begin{code}{Java}{Example from~\cite{jot} where type parameterization is used to create a list implementation.}{lst:type-parameterization}
    template ListsOf<E> {
        class List {
            AuxElem first, last;
            void insertAsLast(E e) { ... }
            E removeFirst() { ... }
        }
        class AuxElem {
            AuxElem next;
            E e; // Reference to the real element
        }
    }
\end{code}

\subsection{Class Merging}\label{subsec:class-merging}

PT should allow for merging two or more classes.
When merging classes the result should be a union of their attributes.
If we merge two classes \codeword{A} and \codeword{B}, it should be possible to reach all of \codeword{B}'s attributes from an \codeword{A}-variable, and vice versa.

% - \textbf{Class  merging:}  Assume  we  have  the  two  collections  C  (with  classes  A  and  B)  and D (with class E).
% When they are both used in the same program, we should be able to merge e.g. the classes A and E so that the resulting class gets the union of the attributes, and so that we via an E-variable defined within D can also directly see the A attributes (and similarly for an A-variable in C).
%
% - - Assume that we in addition to the graph collection have a collection Persons with a class Person.
% In a program handling personal relations we then want to use both collections together so that we obtain a new class, say PersonNode, which has all the  attributes  of  Node  and  Person,  and  where  a  Person-variable  p  defined  in  Persons gives access directly to the Node attributes, e.g. “p.firstEdge”.

\subsection{Collection-Level Type-Checking}\label{subsec:collection-level-type-checking}

The final requirement is collection-level type-checking.
This requirement is there to ensure that each separate package/template can be independently type-checked.
By having the possibility to type-check each pacakge/template we can also verify that the produced program is also type-safe, as long as the instantiation is conflict-free.

% In  addition  to  these  properties,  it  is  important  that  such  collections  of  classes  can  be  separately type-checked.
% We also prefer a mechanism that allows only single inheritance, as  the  merge  property  described  above  to  a  large  extent  will  take  care  of  the  need  for  combining  code  from  different  sources  (for  which  purpose  multiple  inheritance  is  often  used).
% Finally, the type system should be as simple and intuitive as possible.


\section{PTS' Implementation of the Requirements}\label{sec:pts'-implementation-of-the-requirements}

With a proper understanding of the requirements of PT we can examine our implementation and see whether our implementation fulfills these requirements.
For each requirement we will be looking at a program in PTS which showcases the requirements, and the resulting program after compilation.


\subsection{Parallel Extension}\label{subsec:pts-parallel-extension}

To understand how this requirement can be fulfilled it is important to understand how the requirement could fail to be fulfilled.
A failure to fulfill the requirement would be that making additions in parallel would fail to compile, or create an otherwise incorrect program.
Failure to compile might of course not always be a bad thing, there are certain scenarios where we do want the compiler to throw an error.
There are mainly two scenarios where we would like the compilation to fail for additions, trying to make an addition to a non-existent class, and trying to reference non-existent attributes in a class.

The first scenario where our compiler should fail is when we are making additions to a non-existent class.
This will be caught in the class merging part of our compiler.
In the class merging part of the implementation the compiler will group all class declarations and additions by the class name.
If there is a group containing only additions then it will fail, as we have no class to make additions to.

The second scenario is when we are trying to reference non-existent attributes in a class.
An example of this can be seen in listing~\vref{lst:pts-parallel-extension-non-existent-attribute}.
This example will fail during the type-checking of our pacakges/templates, as discussed in~\vref{sec:type-checking-of-templates}.
Our approach for dealing with this is pretty much by not dealing with it, and instead assume that everything is okay at this stage of the compilation.
We will then instead discover any inconsistencies in the type-checking stage of the compiler.
In the aforementioned listing it is of course pretty easy to examine class \codeword{A} to see if it contains an attribute \codeword{h}, however it might not always be this easy.
In a more complicated example where we are in the process of merging several classes and additions it might prove a tougher task to see if the addition would result in a type-safe class.
So as long as we are able to perform the addition we can instead assume that it is working as intended and instead let the TypeScript compiler check if it is type-safe, after the addition has been performed.

\begin{code}{typescript}{An example showing a program that should fail during compilation, where we are trying to reference a non-existent attribute, \codeword{h}, in an addition to class \codeword{A}.}{lst:pts-parallel-extension-non-existent-attribute}
    template T {
        class A {
            function f() {
                return 1;
            }
        }
    }

    package P {
        inst T;
        addto A {
            function g() {
                this.h();
            }
        }
    }
\end{code}

Now that we understand when we want compilation to fail let us look at where we do not want it to fail, when we have a valid parallel extension.
One such way it could fail is if we tried to check if the addition contains any invalid references or type errors.
This could commonly happen if we are trying to check the addition's references to the declared class.
However, as discussed above, checking if a reference to an attribute is valid is quite tricky, and in our implementation instead leave this up to the TypeScript compiler.
By doing this we will not incorrectly throw any false-negatives when it comes to parallel extensions.
This approach does unfortunately come with some downsides.
By not addressing the issue at the addition stage it makes it harder to give informative error messages when invalid references do occur, however this was a tradeoff that was beneficial for this project.

\subsection{Hierarchy Preservation}\label{subsec:pts-hierarchy-preservation}

In order to fulfill the hierarchy preservation requirement we have to preserve all super-/subclass relations after additions and merges have been applied.
Listing~\vref{lst:pts-hierarchy-preservation} shows a program, and the resulting TypeScript program after compilation, which fulfills the requirement of hierarchy preservation.
This one example fulfills the requirement as class \codeword{B} is still a subclass to class \codeword{A} after both a merge and an addition is made to \codeword{B}.
As we talked about briefly in section~\vref{subsec:merging-class-declarations} when we merge classes we make sure to also merge their class heritage, combining the extending classes and implementing interfaces of the different classes.
This means that we might end up with instances where we are extending multiple different classes, however this will then be picked up in the type-checking stage of the compiler.
If we had not merged class heritage, we could have ended up breaking the inheritance hierarchy in the aforementioned listing, as we could have for example ended up with class \codeword{C}'s heritage, which does not have a superclass.
Because of the heritage merging we can with confidence say that we have fulfilled the requirement of hierarchy preservation, as we always preserve all super-/subclass relations.

\begin{code}{typescript}{Example showcasing a program where the super-/subclass relation between classes \codeword{A} and \codeword{B} is preserved after additions and class merging have been applied. We can see the resulting TypeScript program at the bottom of the listing, where the semantics are as expected.}{lst:pts-hierarchy-preservation}
    // PTS
    template T1 {
        class A {
            i = 0;
        }

        class B extends A {
            f() {
                return this.i;
            }
        }
    }

    template T2 {
        class C {
            j = 0;
        }
    }

    package P {
        inst T1;
        inst T2 { C -> B };
        addto B {
            k = 0;
        }
    }

    // Resulting program
    class A {
        i = 0;
    }

    class B extends A {
        f() {
            return this.i;
        }
        j = 0;
        k = 0;
    }
\end{code}

\subsection{Renaming}\label{subsec:pts-renaming}

In order to be able to fulfill the renaming requirement our implementation should be able to rename classes and their attributes.
This renaming should result in a program where not only the declarations have been renamed, but also all references.
Listing~\vref{lst:pts-renaming} shows an example program of renaming in PTS, where we are renaming a class, \codeword{A}, and the class' attribute, \codeword{i}.
We can see in the resulting program that the identifier in the declaration of both the class and the attribute has changed, but so has the references to these in the constructor of the class and references in another class, \codeword{B}\@.
The renaming has also not wrongly renamed other references that are similar in naming, such as the parameter of the constructor of class \codeword{A}\@.
This simple example works as expected, however there are also scenarios where the renaming does not work as expected.

\begin{code}{typescript}{Example of renaming in PTS.}{lst:pts-renaming}
    // PTS
    template T {
        class A {
            i = 0;
            constructor(i: number) {
                this.i = i;
            }
        }

        class B {
            a = new A();
            function f() {
                return a.i;
            }
        }
    }

    pack P {
        inst T { A -> X (i -> j) };
    }

    // Resulting program
    class X {
        j = 0;
        constructor(i: number) {
            this.j = i;
        }
    }

    class B {
        a = new X();
        function f() {
            return a.j;
        }
    }
\end{code}

% TODO diskuter rundt strukturell typing og potensiell usikkerhet rundt hvilken deklarasjon en bruksforekomst forholder seg til.
Since TypeScript is a structurally typed language we can run into scenarios where a rename could and should result in an invalid program.
Listing~\vref{lst:pts-renaming-problem} showcases this problem.
The problem arises in the \codeword{a} attribute of class \codeword{B} in template \codeword{T}.
This has been declared to be a variable expecting an object where there exists an attribute \codeword{i}.
\codeword{a} is initialised with an object of the class \codeword{A}.
This is fine in template \codeword{T}, however when \codeword{T} is instantiated in package \codeword{P}, and \codeword{A}'s attribute \codeword{i} is renamed to \codeword{j}, this is no longer the case.
Since an object of \codeword{A} no longer contains an attribute \codeword{i}, this is no longer valid.

\begin{code}{typescript}{Example showcasing the problem of having renaming in a structural language. In class \codeword{B} we have an attribute, \codeword{a}, that expects an object that contains an attribute \codeword{i}. The attribute is initialized with an \codeword{A} object. This is fine in template \codeword{T} as \codeword{A} contains an attribute \codeword{i}, however when class \codeword{A}'s attribute is renamed in the instantiation in package \codeword{P} then an object of \codeword{A} is no longer valid as a value, since it no longer contains an attribute \codeword{i}. This is an instance where we can't just rename the references to \codeword{i}, since this reference isn't explicitly related to \codeword{A}.}{lst:pts-renaming-problem}
    // PTS
    template T {
        class A {
            i = 0;
        }

        class B {
            a : { i : number } = new A();
            i = a.i;
        }
    }

    pack P {
        inst T { A -> A (i -> j) };
    }

    // Expected result
    class A {
        j = 0;
    }

    class B {
        a : { i : number } = new A();
        i = a.i;
    }
\end{code}

The aforementioned listing shows how we would like the compilation result to look like, however this is not the result the current implementation produces.
TypeScript's type system can be quite complicated, and due to a lack of time I chose to ignore most of the type declarations.
The current implementation would have treated the attribute \codeword{a} as an \codeword{A}-variable, since it is being initialized with an object of \codeword{A}, and therefore have renamed later references to \codeword{a.i} to \codeword{a.j}.
It was more important to get a working prototype, than support all scenarios with different type signatures.
This is something I would of course have liked to take into consideration if I had more time to spend on the implementation.
Deciding the type of variables is something that possibly would have come for cheaper if I had opted for a fork of the TypeScript compiler as my approach.
This is something we will come back to in~\vref{subsec:result-approach}.

\subsection{Multiple Uses}\label{subsec:pts-multiple-uses}

In order for this requirement to be fulfilled we should be able to re-use a template several times, with different renamings and additions while the different instantiations stay independent of each other.
This was something I paid extra attention to during implementation, not just to fulfill the requirement, but to avoid bugs.
I solved this by making sure that while transforming the AST this would be done in an immutable fashion.
In order to test this we will be creating a simple program where we instantiate the same template more than once and see if the resulting program is as expected.
The program can be seen in listing~\vref{lst:pts-multiple-uses}.
The program comprises a template \codeword{T} with a single class, \codeword{A}, with an attribute \codeword{i}.
This template will then be instantiated three times, where we first will be renaming the class and field, then instantiate without renaming, and finally instantiate it with just an attribute renaming.
The expected program should have two classes, one class \codeword{B}, with an attribute \codeword{j}, and a class \codeword{A} where the two bottom instantiations should have created a merged class with attributes \codeword{i} and \codeword{x}.
We can see from the resulting program after a successful compilation that this is as expected.

\begin{code}{typescript}{A program showcasing multiple uses in PTS, and the resulting program in TypeScript at the bottom.}{lst:pts-multiple-uses}
    // PTS
    template T {
        class A {
            i = 0;
        }
    }

    pack P {
        inst T { A -> B (i -> j) };
        inst T;
        inst T { A -> A (i -> x) };
    }

    // Resulting program
    class B {
        j = 0;
    }

    class A {
        i = 0;
        x = 0;
    }
\end{code}

\subsection{Type Parameterization}\label{subsec:pts-type-parameterization}

The type parameterization requirement is something the implementation does not fulfill.
This was not implemented due to it not being prioritized.
There is only so much time available during the span of a master thesis, and I chose to look at how the core of PT would fit into a structurally language like TypeScript, rather than on making sure it would be a fully fleshed out implementation of PT\@.
Another reason for avoiding this is that much of type parameterization can be achieved through merging and renaming.
Listing~\vref{lst:type-parameterization-without-rt} shows an example of how you can use an empty class as a generic type implementation of lists, similar to the list implementation with required types in listing~\vref{lst:type-parameterization}.
Required types do of course have a lot of advantages such as making it possible to constrain the type, and forcing the programmer to give an actual parameter for the type, which we are unable to do.

\begin{code}{java}{Example of a similar list implementation as in listing~\vref{lst:type-parameterization}, without the use of required types. Instead of giving a type for the required type we will have to merge the class \codeword{E} with the "actual parameter".}{lst:type-parameterization-without-rt}
    template ListsOf {
        class E { }
        class List {
            AuxElem first, last;
            void insertAsLast(E e) { ... }
            E removeFirst() { ... }
        }
        class AuxElem {
            AuxElem next;
            E e;
        }
    }
\end{code}

\subsection{Class Merging}

\subsection{Collection-level Type-checking}\label{subsec:implementation-collection-level-type-checking}

\section{Conclusion}\label{sec:requirements-conclusion}

While we do not fulfill every requirement, we do fulfill most of them.
The current implementation might not be a full implementation of PT, but we can confidently say we have at least made an implementation of the core of PT for TypeScript.
Not having a full implementation does mean that we might not be able to examine all the differences between our implementation and PTj, however we will be able to examine the common elements, which covers the most interesting parts.
This allows us to explore how a mechanism like PT fits with the TypeScript language, and its potential utility.