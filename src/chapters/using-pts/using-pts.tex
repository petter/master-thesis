%! Author = Petter
%! Date = 5/14/2021

\chapter{Using PTS}\label{ch:using-pts}

Now that we have a working implementation of our compiler for PTS, let us look into how we could install and use it.
There are mainly two ways of using the PTS compiler:

\begin{itemize}
    \item Installing it globally, or
    \item Creating a PTS project
\end{itemize}

In the following sections we will look at how you can install and use the compiler for both approaches.

The PTS compiler requires you to have Node and npm installed on your computer.
For instructions on installing Node and npm I refer the reader to the npm documentations\footnote{\url{https://docs.npmjs.com/downloading-and-installing-node-js-and-npm}}.

\section{Installing and Using PTS Globally}\label{sec:installing-and-using-pts-globally}

Installing PTS globally will enable you to use PTS anywhere, and might be favorable if you are planning to create several smaller projects to test it out, or if you are not too experienced with the node ecosystem.
If you want to install the compiler globally you can do the following:

\begin{minted}{bash}
    $ npm install -g pts-lang
\end{minted}

This will give you access to the PTS compiler CLI through the command \codeword{pts-lang}.
By giving the \codeword{-{}-help} flag you will get some useful information for how to use the compiler.

\begin{minted}{text}
$ pts-lang --help
Options:
      --help                      Show help
      --version                   Show version number
  -i, --input                     Name of the input file
  -o, --output                    Name of the output file
  -v, --verbose                   Show extra information during
                                  transpilation
  -t, --targetLanguage, --target  Target language for
                                  transpilation
  -r, --run
\end{minted}

\section{Creating a PTS Project}\label{sec:creating-a-pts-project}

If you are using PTS for a specific project it might be better to set it up as a project dependency in npm.
When installed in an npm project the CLI is available to use through npm scripts or through accessing it directly from the \codeword{node\_modules} folder in your project.
The compiler can also be accessed through the API by importing it as with any other npm package.

Installing it inside an npm project will not require you to install it globally, as it will stay contained in the project.
This also means that any contributors of the project will not have to worry about installing PTS, as it will be installed when the project is set up.

To initialize an npm project you can do the following:

\begin{minted}{bash}
    $ mkdir <project name>
    $ cd <project name>
    $ npm init -y
\end{minted}

With a project set up you can install the PTS compiler as following:

\begin{minted}{bash}
    $ npm install pts-lang
\end{minted}

With the PTS compiler installed in the project you can then set up some scripts in your project's \codeword{package.json} to start and/or build the project.
Below you can see an example of a section of a \codeword{package.json} file with scripts for running and building a file:

\begin{minted}{json}
    {
      "scripts": {
        "start": "pts-lang -i src/index.pts --run",
        "build": "pts-lang -i src/index.pts -o build/index"
      }
    }
\end{minted}

The start script only runs the program, and does not emit any files, while the build script transpiles the \codeword{src/index.pts} file to JavaScript.
If you would rather have TypeScript output you can use the \codeword{-t} flag to specify this:

\begin{minted}{bash}
    pts-lang -i src/index.pts -o build/index -t ts
\end{minted}

\section{A "Real World" Example}\label{sec:a-real-world-example}

Now that we understand how to get PTS set up, let us look at how it could be used in a real world example, and how PT enables the programmer to modularize the code base even further giving great flexibility.
Note that the following example will not work with the current state of the compiler as it doesn't handle member expressions containing call expressions, such as \codeword{f().i}.
Properly handling these types of member expressions would require us to analyse the function for its return type.
The example serves as an example of how PTS could be useful given a more complete implementation for a real world problem.

The most common use of TypeScript is to create web applications.
Let us look at how PTS can help make this task easier for the programmer.
We will try to create a simple web application for displaying a Pokémon.
To do this we will use one of the most popular web frameworks, React\footnote{\url{https://reactjs.org/}}.
We could of course just display some information about a predetermined Pokémon, however, we would like to make something re-usable.
We will utilize React to create something re-usable, which we can use to display information about any Pokémon.
We do not want to have to write down information about all Pokémon, so we will fetch this information from an API, more specifically the PokéAPI\footnote{\url{https://pokeapi.co/}}.
This API lets us fetch data about all Pokémon.

\subsection{Short Introduction to React}\label{subsec:short-introduction-to-react}

React is a web framework developed by Facebook\footnote{\url{https://www.facebook.com/}}.
It aims to make creating scalable web projects easier to handle, through enabling the programmer to modularize collections of elements into \emph{components}.
These components are often created to make re-use of common elements easier, such as creating a styled button with certain features, or we could create a component to represent the entire web application.

Components can be made either through creating a function that returns some JSX\footnote{JSX is a syntax extension to JavaScript which resembles HTML, but in reality is just syntactic sugar for creating React elements. See~\url{https://reactjs.org/docs/introducing-jsx.html}.}, which we call functional components, or through creating a class that extends the \codeword{Component} class, which we call class components.
We will in this example create class components, as PT has a lot of useful functionality for adapting classes.
Class components most important method is the \codeword{render} method.
The \codeword{render} method works essentially the same way as a functional components, where you return some piece of JSX, which is what will be displayed when the component is used.

A component essentially has two sources of data, its \emph{state} and its \emph{props}.
Props, short for properties, is data passed to a component from its parents.
An example of this can be \codeword{<SomeComponent text="some text" />}, where the component \codeword{SomeComponent} got some text from its parents.
This can be accessed by the component through the \codeword{props} attribute.
The other source of data for components is the component's state.
State is a piece of data connected to the component, which similar to props can be accessed through the \codeword{state} attribute.
Unlike props, state is entirely controlled by the component itself.
The state can be updated through using the \codeword{setState} method of the \codeword{Component} class, which will also trigger a re-render of the component.

Except for the \codeword{render} method, class components also has methods for lifecycle events, such as \codeword{componentDidMount}, which is called after the initial render of the component has finished, and \codeword{componentDidUpdate}, which is called after the state of the component is updated.
These lifecycle methods are very useful for reacting to state changes, or to perform some asynchronous actions.
For a more thorough introduction to React I refer the reader to the React documentation\footnote{\url{https://reactjs.org/docs/getting-started.html}}.

\subsection{The \codeword{FetchJSON} Template}\label{subsec:the-fetchjson-template}

We will start off with the task of fetching data.
As this is something you commonly want to do in web applications it might be a good idea to separate this logic into a separate template.
Fetching data is commonly done after a component has been mounted, so we use the \codeword{componentDidMount} lifecycle method for this.
For fetching the data we use the \codeword{fetch} function from the WebAPI\footnote{\url{https://developer.mozilla.org/en-US/docs/Web/API/Fetch_API}}.

\begin{minted}{typescript}
    template FetchJSON {
        class FetchJSON extends Component {
            componentDidMount() {
                fetch(this.props.url)
                    .then(response => response.json())
                    .then(data =>
                        this.setState(state => ({...state, data}))
                    ).catch(error =>
                        this.setState(state => ({...state, error}))
                    );
            }
        }
    }
\end{minted}

The \codeword{FetchJSON} component we will fetch whatever URL we pass to it in its props and update the state with the results of the fetch.
If we for some reason should fail to fetch the data we will instead update the state with the error message we got.

\subsection{The \codeword{StateLogger} Template}\label{subsec:the-statelogger-template}

In addition to fetching data, it might be useful to have a logger, which will log all state changes to the console.
This is often useful when working with React components as we are able to see when they update, and what the state was at the time of the update.
Such a logger could then also be separated into its own template, like the following:

\begin{minted}{typescript}
    template StateLogger {
        class StateLogger extends Component {
            componentDidUpdate() {
                console.log("State updated!", this.state);
            }
        }
    }
\end{minted}

\subsection{Creating the \codeword{Pokemon} Component}\label{subsec:creating-the-pokemon-component}

Finally we would like to combine these templates into our Pokémon component, and add some logic for displaying the information.
We will do this inside of a package, so that this will produce an output:

\begin{minted}{typescript}
    pack Pokemon {
        inst FetchJSON { FetchJSON -> Pokemon };
        inst StateLogger { StateLogger -> Pokemon };
        addto Pokemon {
            render() {
                if(this.state.error) {
                    return (
                        <div>
                            <h1>An error occurred</h1>
                            <p>{this.state.error.message}</p>
                        </div>
                    );
                }

                if(this.state.data === undefined) {
                    return 'Loading...';
                }

                const name = this.state.data.name;
                const pokemonTypes = this.state.data.types;
                const image = this.state.data.sprites.front_default;
                return (
                    <div>
                        <img src={image} />
                        <h1>{name}</h1>

                        <h2>Types</h2>
                        <ul>
                            {pokemonTypes.map(pokemonType => (
                                <li>{pokemonType.name}</li>
                            ))}
                        </ul>
                    </div>
                )
            }
        }
    }
\end{minted}

We could then use our Pokémon component in our application by supplying a URL for the Pokémon to display, as seen below:

\begin{minted}{typescript}
    class App extends Component {
        render() {
            <Pokemon
                url="https://pokeapi.co/api/v2/pokemon/ditto" />
        }
    }
\end{minted}

\subsection{How the Example Benefited From PT}\label{subsec:how-the-example-benefited-from-pt}

In this example PT helped us split up our component into several smaller modules.
This enables us to later re-use these common pieces of functionality, fetching data and logging, in other components, which will make our project more scalable, and aids in shortening development time.
Modularizing these concepts also helped make the implementation of our Pokémon component less cluttered, which makes the readability of the code better.
