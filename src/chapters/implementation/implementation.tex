%! Author = petter
%! Date = 04.01.2021

\chapter{Implementation}\label{ch:implementation}

In this chapter we are going to look at the implementation for our compiler for PTS, as described in chapter~\vref{ch:the-language---pts}.

\section{Compiler Architecture}\label{sec:architecture}

Our compiler consists of the following parts:

\begin{itemize}
    \item Lexing and parsing
    \item Parse tree transformation
    \item Type checking packages/templates
    \item Code generation
\end{itemize}

An overview of our architecture can be seen in figure~\vref{fig:compiler-overview}.
The first part of the compiler, namely the lexing and parsing will take a source file and transform it into a parse tree.
Our compiler will then take this parse tree and transform it into a simpler AST\@.
This simple AST will then be used to perform the PT transformations.
We will then use this AST to close any open packages and templates, and finally use the transformed tree for code generation.
After all packages and templates have been closed we can type-check each individual package/template to validate the type-safety of our program.
Given a valid type-safe program we can then move on to code generation.
The target language for our code generation will mainly be TypeScript, however we will also offer to transpile the TypeScript into JavaScript.

\begin{figure}
   \centering
   \includegraphics[scale=.75]{images/Compiler overview.png}
   \caption{Overview of the compiler}
   \label{fig:compiler-overview}
\end{figure}

\section{Lexer and Parser}\label{sec:lexer-and-parser}


\subsection{Parser Generator}\label{subsec:parser-generator}

There are a lot of parser generators out there, but there is no one-size-fits-all solution.
In order to navigate through the sea of options we need to set some requirements in terms of functionality, so that we can more easily find the right tool for the task.

As we talked about in section~\vref{sec:what-do-we-need}, we set ourselves the goal to find an approach that would allow us to create an implementation that was loosely coupled with TypeScript.
TypeScript is a large language that is constantly updated, and is getting new features fairly often.
Because of this one of the requirements for our choice of parser generator is the possibility for extending grammars.
This is important because we want to keep our grammar loosely coupled with the TypeScript grammar, and don't want to be forced to rewrite the entire TypeScript grammar, as well as keeping it up-to-date.

Because our language will be extending TypeScript we would like to utilize the TypeScript compiler as much as we can.
The TypeScript compiler will help us perform the type-checking for our compiler, as well as producing JavaScript output.
Therefore we need to be able to interact with the TypeScript compiler somehow.
The TypeScript compiler has two main interfaces for interaction, through the command-line interface or using the compiler API\@.
Optimally we would like to use the compiler API as this is the easiest way for us to perform type-checking and compilation.
The catch is however that the only supported languages for the compiler API is JavaScript and TypeScript.
Therefore a desired attribute for our choice of parser generator is that it offers a runtime library in either JavaScript or TypeScript, so that all of our implementation can be done in the same language, and not have to work with command-line interfaces programmatically.

\subsubsection{ANTLR4}\label{subsubsec:antlr}

ANTLR, ANother Tool for Language Recognition, is a very powerful and versatile tool, used by many, such as Twitter for query parsing in their search engine~\cite{Terence2012}.

ANTLR supports extending grammars, or more specifically importing them.
Importing a grammar works much like a "smart include".
It will include all rules that are not already defined in the grammar.
Through this you can extend a grammar with new rules or replacing them.
It does not however support extending rules, as in referencing the imported rule while overriding~\cite{Terence2012}.
This isn't a major issue however as you could easily rewrite the rule with the additions.

The only supported runtime library in ANTLR is in Java.
This does not mean that you won't be able to use it in any other language, as you could simply invoke the runtime library through command line, however it is worth keeping in mind.

Overall ANTLR seems like a good option for our project, but the lack of a runtime library in TypeScript is a hurdle we would rather get a round if we can.

\subsubsection{Bison}\label{subsubsec:bison}

Bison is a general-purpose parser generator.
It is one of many successors to Yacc, and is upwards compatible with Yacc~\cite{bison}.

Bison does not support extending grammars.
The tool works on a single grammar file and produces a C/C++ program.
There is a possibility to include files, like with any other C/C++ program, in the grammar files prologue, however this will not allow us to include another grammar, as it only inserts the prologue into the generated parser.
In order to extend a grammar we would have to change the produced parser to include some extra rules.
Although this could possibly be automated by a script, it seems too hacky of a solution to consider.

On top of this Bison does not have a runtime library in JavaScript/TypeScript.
There do exist some ports/clones of Bison for JavaScript, such as \href{http://zaa.ch/jison/}{Jison} and \href{http://canna71.github.io/Jacob/}{Jacob}, however to my knowledge these also lack the functionality of extending grammars.

\subsubsection{Tree-sitter}\label{subsubsec:tree-sitter}

\href{https://tree-sitter.github.io/tree-sitter/}{Tree-sitter} is a fairly new parser generator tool, compared to the others in this list.
It aims to be general, fast, robust and dependency-free~\cite{tree-sitter}.
The tool has been garnering a lot of traction the last couple of years, and is being used by Github, VS Code and Atom to name a few.
It has mainly been used in language servers and syntax highlighting, however it should still work fine for our compiler since it does produce a parse tree.

Although it isn't a documented feature, Tree-sitter does allow for extending grammars.
Extending a grammar works much like in ANTLR, where you get almost a superclass relation to the grammar.
One difference from ANTLR though is that it does allow for referencing the grammar we are extending during rule overriding.
This makes it easier and more robust to extend rules than in ANTLR.

Tree-sitter also has a runtime library for TypeScript, which makes it easier for us to use it in our implementation than the previous candidates.

Another cherry on top is that Tree-sitter is becoming one of the mainstream ways of syntax highlighting in modern editors and IDEs, which means that we could utilize the same grammar to get syntax highlighting for our language.

All this makes Tree-sitter stand out as the best candidate for our project.

\subsubsection{Implementing Our Grammar in Tree-sitter}

Tree-sitter uses the term rule instead of production, and I will therefore also refer to productions as rules here.

Extending a grammar in Tree-sitter works much like extending a class in an object-oriented language.
A "sub grammar" inherits all the rules from the "super grammar", so an empty ruleset would effectively work the same as the super grammar.
Just like most object-oriented languages have access to the super class, we also have access to the super grammar in Tree-sitter.
All of this enables us to add, override, and extend rules in an existing grammar, all while staying loosely coupled with the super grammar.
By extending the grammar, and not forking it, we are able to simply update our dependency on the TypeScript grammar, minimizing the possibility for conflicts.

As mentioned, Tree-sitter allows for referencing the super grammar during rule overriding, effectively making it possible to combine the old rule and the new.
A good example of overriding and combining rules can be found in the grammar of PTS, see listing~\vref{lst:overriding-combining-rule}, where we override the \codeword{\_declaration} rule from the TypeScript grammar, to include the possibility for package and template declarations.

\begin{code}{javascript}{Snippet from the PTS grammar, where we override the \codeword{\_declaration} rule from the TypeScript grammar, and adding two additional declarations.}{lst:overriding-combining-rule}
    _declaration: ($, previous) =>
        choice(
            previous,
            $.template_declaration,
            $.package_declaration
        )
\end{code}




\section{Transforming Parse Tree to AST}\label{sec:transforming-parse-tree-to-ast}
%! Author = Petter
%! Date = 2/26/2021

\subsection{What Does the Parse Tree Look Like}\label{subsec:what-does-the-parse-tree-look-like}

\subsection{The AST Nodes}\label{subsec:the-ast-nodes}

Not class for each node type.
    Too much work, and not loosely coupled

\subsection{Transforming}\label{subsec:transforming}

Walk DFS

\section{Closing Templates}\label{sec:closing-templates}
A closed template is a template that comprises only class, interface or enum declarations.
An open template on the other hand is a template that contains at least one instantiation statement.

The task of closing open packages and templates is what most of the implementation is focused around.
It is the task of performing the instantiations declared in the body of a package/template.

For instantiations without renaming the task is fairly simple.
We merely have to find the referenced template, and replace the instantiation statement with the body of said template.
Renaming on the other hand requires a bit more work.

In order to perform renaming on an instantiation we will have to perform the following tasks.
\begin{enumerate}
    \item Create a scope for the template body, and give all nodes of the body access to this scope
    \item Find all identifiers, member expressions, class declarations, etc., and replace them with a reference node
    \item Perform the rename on the scope
    \item Transform the scoped AST with reference nodes back to the original AST.
\end{enumerate}

If the template body is also open we would have to close it as well.
We want to close the nested templates before closing the upper templates, as renaming at the top level should affect all members from the nested instantiations.

Finally, once all templates have been closed we will have to perform class merging and apply any additions to classes.

In order to get a better understanding of this we will go through each step of closing a template in more detail.

%\begin{code}{}{Pseudocode for closing a template. The same code could also be used to close a package, as they are essentially equal in this regard.}{lst:pseudo-close-pacakge}
%    templates <- map from template name to body
%    FUNCTION close_template(template_body)
%        new_program <-
%            MAP syntax_node IN template_body
%                IF syntax_node IS instantiation statement THEN
%                    instantiated_template_body <- LOOKUP referenced template from instantiation statement IN templates
%                    closed_template_body <- close_template(instantiated_template_body)
%                    scoped_body <- add_scope(closed_template_body)
%
%                    renamings <- get renamings from instantiation statement
%                    FOR renaming IN
%                ELSE
%                    syntax_node
%                END IF
%            END MAP
%        RETURN new_program
%    END FUNCTION
%
%    FUNCTION
%\end{code}

\subsection{Scoping}\label{subsec:inst-scoping}

For creating scopes I chose the following node types for "making new scopes".

\begin{itemize}
    \item \codeword{class\_body}
    \item \codeword{statement\_block}
    \item \codeword{enum\_body}
    \item \codeword{if\_statement}
    \item \codeword{else\_statement}
    \item \codeword{for\_statement}
    \item \codeword{for\_in\_statement}
    \item \codeword{while\_statement}
    \item \codeword{do\_statement}
    \item \codeword{try\_statement}
    \item \codeword{with\_statement}
\end{itemize}

\subsection{Transforming Nodes to References}\label{subsec:transforming-nodes-to-references}


\section{Verification of Templates}\label{sec:verification-of-templates}

ts api

\section{Code Generation}\label{sec:code-generation}

After performing these steps we can finally produce the output.
Producing TypeScript output is a pretty simple task.
By traversing the AST we can concatenate the text from each leaf node and insert some whitespace between them.
This will produce quite ugly, unformatted code, but as long as the closed packages and templates are valid typescript programs this will also produce a valid typescript program.
In order to make it a bit more clean we perform an extra step before writing the output to the specified file, a formatting step.
There are a lot of TypeScript formatters out there, but we will be using \textit{Prettier} for our implementation.
Running our produced source code through the prettier formatter produces a nicely formatted, readable output.

Since browsers only support JavaScript we will also be implementing this as a target for code generation.
This is fortunately also a simple task, as we already dependent on the TypeScript compiler, and we are able to produce TypeScript source code, we can use this to produce JavaScript output.
% TODO: Undersøk  hvilken ES versjon vi outputter til og om dette kan være lavere.

\section{Notes on Performance}\label{sec:notes-on-performance}

Very slow compiler/PP because of the chosen implementation, with tree traverser for every step.


\section{Testing}\label{sec:testing}

\subsection{Lexer and Parser}\label{subsec:testing-lexer-and-parser}

Tree-sitter tests are simple \codeword{.txt} files split up into three sections, the name of the test, the code that should be parsed, and the expected parse tree in \textit{S-expressions}\footnote{S-expressions are textual representations for tree-structured data. See~\cite{sexprs} for additional information and examples}.

\begin{code}{typescript}{Example of tree-sitter grammar test}{lst:tree-sitter-grammar-test}
    ===========================
    Closed template declaration
    ===========================

    template T {
        class A {
            i = 0;
        }
    }

    ---
    (program
        (template_declaration
            name: (identifier)
            body: (package_template_body
                    (class_declaration
                        name: (type_identifier)
                        body: (class_body
                            (public_field_definition
                                name: (property_identifier)
                                value: (number)))))))

\end{code}

\subsection{Transpiler}\label{subsec:testing-transpiler}

Started with jest, and used some time to get it to work with typescript files, however had to switch because jest doesn't handle native libraries (tree-sitter) too well.
It \codeword{require}s the same native library several times, making the wrapping around the native program to break.