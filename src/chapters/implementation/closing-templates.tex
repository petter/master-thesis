The task of closing open packages and templates is what most of the implementation is focused around.
It is the task of performing the declared instantiations and altering the declared classes through \codeword{addto}-statements.
This step is cruicial as it will make each package/template a valid TypeScript program and make the program ready for code generation.

For instantiations without renaming the task is fairly simple.
We merely have to find the referenced template, and replace the instantiation statement with the body of said template.
Renaming on the other hand requires a bit more work.

In order to perform renaming on an instantiation we will have to perform the following tasks.
\begin{enumerate}
    \item Create a correctly scoped AST.
    \item Find all identifiers, member expressions, class declarations, etc., and replace them with \textit{reference nodes}\footnote{Reference nodes are AST nodes that contain a pointer to the class or attribute they are supposed to represent. This makes the task of renaming easier as we only have to worry about changing the name in one place. We will go into more detail about this in section~\vref{subsec:transforming-nodes-to-references}.}
    \item Perform the rename.
    \item Transform the scoped AST with reference nodes back to the original AST\@.
    \item Merge class declarations and apply \codeword{addto} statements.
\end{enumerate}

If the template body is also open we would have to close it as well.
We want to close the nested templates before closing the upper templates, as renaming at the top level should affect all members from the nested instantiations.

Finally, once all templates have been closed we will have to perform class merging and apply any additions to classes.

In order to get a better understanding of this we will go through each step of closing a template in more detail.

%\begin{code}{}{Pseudocode for closing a template. The same code could also be used to close a package, as they are essentially equal in this regard.}{lst:pseudo-close-pacakge}
%    templates <- map from template name to body
%    FUNCTION close_template(template_body)
%        new_program <-
%            MAP syntax_node IN template_body
%                IF syntax_node IS instantiation statement THEN
%                    instantiated_template_body <- LOOKUP referenced template from instantiation statement IN templates
%                    closed_template_body <- close_template(instantiated_template_body)
%                    scoped_body <- add_scope(closed_template_body)
%
%                    renamings <- get renamings from instantiation statement
%                    FOR renaming IN
%                ELSE
%                    syntax_node
%                END IF
%            END MAP
%        RETURN new_program
%    END FUNCTION
%
%    FUNCTION
%\end{code}

\subsection{Scoping}\label{subsec:inst-scoping}

This step of closing templates works on the body of a copy of the instantiated template.
In order to be able to rename classes and class attributes we first need to create correct scopes in which the renaming can be applied.
We start off with a list of normal AST nodes and will transform these nodes into nodes that has a reference to the scope they are part of.

A scope is represented through the \codeword{Scope} class.
The \codeword{Scope} class is essentially a symbol table that optionally extends a parent scope.
A scope without a reference to a parent scope is the root scope.
The symbol table is implemented as a map from the original attribute or class name, to a reference to either a variable (this covers both class attributes and other variables used throughout the program) or a class.
Looking up symbols in the symbol table will always start in the called scope, looking for any references matching the given name.
If we don't find any references with the given name we propagate the lookup to the parent scope.
Given further misses in the symbol table means that we will eventually reach the root scope, and if the root scope also doesn't contain any references then we fail the compilation and inform the programmer that there is a reference error in the program.

Having several layers of scope enables us to correctly handle shadowed variables, or parallel declarations with equal naming.

For creating scopes I chose the following node types for "making new scopes".

\begin{multicols}{2}
\begin{itemize}
    \item \codeword{class\_body}
    \item \codeword{statement\_block}
    \item \codeword{enum\_body}
    \item \codeword{if\_statement}
    \item \codeword{else\_statement}
    \item \codeword{for\_statement}
    \item \codeword{for\_in\_statement}
    \item \codeword{while\_statement}
    \item \codeword{do\_statement}
    \item \codeword{try\_statement}
    \item \codeword{with\_statement}
\end{itemize}
\end{multicols}

We start of with a root scope which is given to the root node of the template body.
We then traverse the tree and give every node a reference to the scope.
When we reach one of the aforementioned nodes that should have its own scope, we create a new scope, with the current scope as the parent scope.
This new scope is then given to all nodes beneath this node, until we reach another node in the list above.

At the end of this traversal we have an AST where every node has a reference to the scope its in.

\subsection{Transforming Nodes to References}\label{subsec:transforming-nodes-to-references}

In order to rename a class or an attribute we need to find all references to the class or attribute.
That is what this step is supposed to do, find all references and group these together, so that a rename can be performed.
A reference in our program can either be a variable or a class reference.

A variable reference can be a class attribute or any other variable declaration.
Even though we call it a variable reference this does also cover functions, however both variables and functions share the same namespace, so there is essentially no difference between them, unlike Java where methods and variables can have the same name.
These references are represented by a \codeword{Variable} class.
The class contains the name of the attribute, and optionally what type it is.
The type of a variable in our implementation is set through a very simple type-inference (relative to TypeScripts type system), where we only look at if there is a \codeword{new}-expression assigned to the variable at declaration.
If a variable is initialized with a class instantiation then the type of the variable is an object of that class.
However, this simple type inference might not always correctly type variables, as variables might have explicitly declared types which are not typed with the class' type, however for simplicity, and in order to not reimplement TypeScript's entire type system, we currently ignore these in the implementation.

Class references are all references made to a class.
These references are references such as class declarations, instantiations of a class, usage of a class as a type, etc.
The \codeword{Class} class is a representation for class references.
The class is an extension of the \codeword{Variable} class, so it can store the name of the class, however they also have references to all the variables that are instances of the class, and optionally a superclass, which is a reference to another \codeword{Class} instance.

With an understanding of what a reference is, let us look at how we can transform the AST nodes that are references.
Transforming nodes into references mainly consists of two steps.

\begin{itemize}
    \item Transforming declarations
    \item Transforming references
\end{itemize}

\subsubsection{Transforming Declarations}

When transforming declarations we both create the \codeword{Variable} or \codeword{Class} instance, register them in the scope they were found, and transform the identifier in the declaration to a reference.
This reference is a special AST node.
It is represented by the \codeword{RefNode} class.
This class contains the same fields as other scoped AST nodes, \codeword{type} (which is always \codeword{"variable"} for reference nodes), \codeword{text}, \codeword{children} and \codeword{scope}, as well as two extra fields.
The first field is a reference to the \codeword{Variable} or \codeword{Class} instance that the \codeword{RefNode} is a reference to.
The second is a field containing the original type of the node, which is used when transforming the \codeword{RefNode}s back to the original AST\@.

The task of transforming declarations requires several passes through the AST\@.
In total we pass through the AST three times when transforming declarations, once for class declarations, once for class heritage, and once for class attribute declarations.
This is because in order to transform some declarations we need to have access to others.
To understand this we can look at the difference between the first and second pass.

During the first pass through the AST we register and transform all class declarations, creating the \codeword{Class} instances.
Listing~\vref{lst:transform-reference-class-decl} shows an example of how a class declaration will be transformed in this step.
In the listing we can see that the \codeword{"type\_identifier"} node has been transformed to a \codeword{RefNode} instance.

For the second pass we register class heritage.
The second pass updates the \codeword{Class} instances' superclasses, by looking them up in the scope.
If we had done these two steps in the same pass through the AST we might have ended up in situations where we could not find the superclass instance, as we might not have reached the declaration yet.

In addition to transforming class declarations, and their heritage, we also transform attribute declarations.
This could have been done in the same pass as with class heritage, however it was separated to make the code more readable and the flow of the transformations more understandable.
During this transformation we also check if there is a \codeword{new}-expression in the assignment, in order to possibly register the type of the variable.
The current implementation only supports public field definitions.
Other declaration types could be supported in the same manner as public field definitions, however due to a lack of time these were not implemented.

\begin{code}{javascript}{AST of a class declaration of class A before and after transforming the references. The values surrounded by angle brackets are references to \codeword{Scope}/\codeword{Class} instances.}{lst:transform-reference-class-decl}
    // Before transformation
    {
        type: 'class_declaration',
        text: 'class A ...',
        scope: <Scope 1>,
        children: [
            {
                type: 'class',
                text: 'class',
                scope: <Scope 1>,
                children: []
            }, {
                type: 'type_identifier',
                text: 'A',
                scope: <Scope 1>,
                children: [],
            }
            ...
        ]
    }

    // After transformation
    {
        type: 'class_declaration',
        text: 'class A ...'
        scope: <Scope 1>,
        children: [
            {
                type: 'class',
                text: 'class',
                scope: <Scope 1>,
                children: []
            }, {
                type: 'variable',
                text: '',
                scope: <Scope 1>,
                children: [],
                origType: 'type_identifier',
                ref: <Class 1>
            },
            ...
        ]
    }
\end{code}

\subsubsection{Transform References}

Transforming references also has several passes, more precisely two passes.
The first pass is to transform the \codeword{this} keyword.
This does not have to be done in a separate pass, however it does simplify the task since we do not have to have extra special cases in the second pass.
The \codeword{this}-nodes are transformed to a \codeword{RefNode} like other references, however here we can set the type of the variable without working out the type of it, as \codeword{this} will always have the type of the class it is contained in.

In the second pass the rest of the references are transformed.
Currently the implementation will transform \codeword{new} expressions, member expressions, and type annotations.

\codeword{new} expressions are pretty simple to transform, similar to class declarations, we can just find the identifier node and replace this by the referenced class.
We find the class by looking it up in the scope.
It is worth noting that this will not rename any generic type parameters.

Member expressions are a bit more troublesome.
A member expression is an expression such as \codeword{a.i} or \codeword{this.b.x}.
This is the main reason for why we transformed \codeword{this}-expressions separately, since they can, and often do, occur in member expressions, and this saves us from checking another special case.
A member expression consists of the object and the property.
The property is always an identifier, while the object on the other hand can be one of three things, a \codeword{RefNode}, a member of an identifier, or another member expression.

The most common member expression we can transform is a member of a \codeword{RefNode}.
The \codeword{RefNode} here could be something like \codeword{this} or a variable, \codeword{a}.
Here we use the type of the \codeword{Variable} instance in order to find the attribute that the member expression is referencing.
If the attribute does not exist we throw an error, informing the programmer that there is a reference error.
Else we can create a \codeword{RefNode} for the property part of the member expression.

The second scenario for member expressions are when there is a nested member expression.
When this is the case we can recursively call the function on the nested member expression.
After the nested member expression has been transformed we can get the property of the nested member expression, and use this to transform the property of the initial member expression, similar to how we did it for members of \codeword{RefNode} above.

The final member expression is a member expression of an identifier.
This can happen if we are dealing with something that can't be renamed, such as \codeword{console.log}, or if we stumble upon an identifier that has not been picked up during previous transformations.
We will in both cases try to see if we are able to find the object by looking it up in the scope.
If we can't find it we know that its something that can't be renamed.
If we do find it we treat it like a member of a \codeword{RefNode}.

Type annotations consists of the following:

\begin{itemize}
    \item Type identifiers
    \item Generic types
    \item Predefiend types
    \item Type predicates
\end{itemize}

\red{TODO: Skriv om transformation av type annotations}

\subsection{Performing the Rename}\label{subsec:performing-the-rename}

With all declarations and references transformed into reference nodes we can now perform the rename.
For each class rename we can simply lookup the old class name in the root scope and change the reference to the new class name.
For class attribute renames we do the same just for the enclosing classes scope.

\subsection{Going back to the original AST}\label{subsec:going-back-to-the-original-ast}

reference -> original with new names
remove scope

\subsection{Merging Class Declarations}\label{subsec:merging-class-declarations}

Check if can be merged, non overlapping members.
Override and merge addto
Preserve hierarchy