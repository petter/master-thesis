The task of closing open packages and templates is what most of the implementation is focused around.
Closing packages/templates can be summarized up into these tasks:

% templates & program -> close template/package -> close instantiation -> scope instantiated template body -> rename classes and fields
%                                                                       <- closed template
%                                                   <- replace inst with closed template
%                       <- closed template/package

\begin{enumerate}
    \item Gather template declarations
    \item Replace instantiation statements with the body of the referenced template
    \item 
\end{enumerate}

Step 2 however

\begin{code}{}{Pseudocode for closing a template. The same code could also be used to close a package, as they are essentially equal in this regard.}{lst:pseudo-close-pacakge}
    templates <- map from template name to body
    FUNCTION close_template(template_body)
        new_program <-
            MAP syntax_node IN template_body
                IF syntax_node IS instantiation statement THEN
                    instantiated_template_body <- LOOKUP referenced template from instantiation statement IN templates
                    closed_template_body <- close_template(instantiated_template_body)
                    scoped_body <- add_scope(closed_template_body)

                    renamings <- get renamings from instantiation statement
                    FOR renaming IN
                ELSE
                    syntax_node
                END IF
            END MAP
        RETURN new_program
    END FUNCTION

    FUNCTION
\end{code}

\subsection{Scoping}\label{subsec:inst-scoping}

For creating scopes I chose the following node types for "making new scopes".

\begin{itemize}
    \item \codeword{class\_body}
    \item \codeword{statement\_block}
    \item \codeword{enum\_body}
    \item \codeword{if\_statement}
    \item \codeword{else\_statement}
    \item \codeword{for\_statement}
    \item \codeword{for\_in\_statement}
    \item \codeword{while\_statement}
    \item \codeword{do\_statement}
    \item \codeword{try\_statement}
    \item \codeword{with\_statement}
\end{itemize}

\subsection{Transforming Nodes to References}\label{subsec:transforming-nodes-to-references}
