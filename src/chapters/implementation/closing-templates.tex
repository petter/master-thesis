The task of closing open packages and templates is what most of the implementation is focused around.
It is the task of performing the declared instantiations and altering the declared classes through \codeword{addto}-statements.
This step is crucial as it will make each package/template a valid TypeScript program and make the program ready for code generation.

For the most basic instantiations that don't lead to any renaming, performing additions to classes, or merging of classes, the task is fairly simple.
We merely have to find the referenced template, and replace the instantiation statement with the body of said template.
However, to support proper instantiation, the task becomes a bit more advanced.

In order to support instantiation of templates, with all the concepts related to it, we will have to do some additional steps.
For each instantiation statement we will have to perform the following steps on the body of the instantiated template.

\begin{enumerate}
    \item Create a correctly scoped AST
    \item Transform nodes to \textit{reference nodes}\footnote{Reference nodes are AST nodes that contain a pointer to the class or attribute they are supposed to represent. This makes the task of renaming easier as we only have to worry about changing the name in one place. We will go into more detail about this in section~\vref{subsec:transforming-nodes-to-references}.}
    \item Perform the rename
    \item Go back to the original AST
    \item Merge classes
\end{enumerate}

If the template that is being instantiated is not closed, we will perform the same steps to all the nested instantiations in a depth-first manner.
We want to close the nested templates before closing the upper templates, as renaming at the top level should affect all members from the nested instantiations.

Finally, once all templates have been closed we will have to perform class merging and apply any additions to classes.

In order to get a better understanding of this we will go through each step of closing a template in more detail.

%\begin{code}{}{Pseudocode for closing a template. The same code could also be used to close a package, as they are essentially equal in this regard.}{lst:pseudo-close-pacakge}
%    templates <- map from template name to body
%    FUNCTION close_template(template_body)
%        new_program <-
%            MAP syntax_node IN template_body
%                IF syntax_node IS instantiation statement THEN
%                    instantiated_template_body <- LOOKUP referenced template from instantiation statement IN templates
%                    closed_template_body <- close_template(instantiated_template_body)
%                    scoped_body <- add_scope(closed_template_body)
%
%                    renamings <- get renamings from instantiation statement
%                    FOR renaming IN
%                ELSE
%                    syntax_node
%                END IF
%            END MAP
%        RETURN new_program
%    END FUNCTION
%
%    FUNCTION
%\end{code}

\subsection{Create a correctly scoped AST}\label{subsec:inst-scoping}

This step of closing templates works on the body of a copy of the instantiated template.
In order to be able to rename classes and class attributes we first need to create correct scopes in which the renaming can be applied.
We start off with a list of normal AST nodes and will transform these nodes into nodes that has a reference to the scope they are part of.

A scope is represented through the \codeword{Scope} class.
The \codeword{Scope} class is essentially a symbol table that optionally extends a parent scope.
A scope without a reference to a parent scope is the root scope.
The symbol table is implemented as a map from the original attribute or class name, to a reference to either a variable (this covers both class attributes and other variables used throughout the program) or a class.
Looking up symbols in the symbol table will always start in the called scope, looking for any references matching the given name.
If we don't find any references with the given name we propagate the lookup to the parent scope.
Given further misses in the symbol table means that we will eventually reach the root scope, and if the root scope also doesn't contain any references then we fail the compilation and inform the programmer that there is a reference error in the program.

Having several layers of scope enables us to correctly handle shadowed variables, or parallel declarations with equal naming.

For creating scopes I chose the following node types for "making new scopes".

\begin{multicols}{2}
\begin{itemize}
    \item \codeword{class\_body}
    \item \codeword{statement\_block}
    \item \codeword{enum\_body}
    \item \codeword{if\_statement}
    \item \codeword{else\_statement}
    \item \codeword{for\_statement}
    \item \codeword{for\_in\_statement}
    \item \codeword{while\_statement}
    \item \codeword{do\_statement}
    \item \codeword{try\_statement}
    \item \codeword{with\_statement}
\end{itemize}
\end{multicols}

We start of with a root scope which is given to the root node of the template body.
We then traverse the tree and give every node a reference to the scope.
When we reach one of the aforementioned nodes that should have its own scope, we create a new scope, with the current scope as the parent scope.
This new scope is then given to all nodes beneath this node, until we reach another node in the list above.

At the end of this traversal we have an AST where every node has a reference to the scope its in.

\subsection{Transforming Nodes to Reference Nodes}\label{subsec:transforming-nodes-to-references}

In order to rename a class or an attribute we need to find all references to the class or attribute.
That is what this step is supposed to do, find all references and group these together, so that a rename can be performed.
A reference in our program can either be a variable or a class reference.

A variable reference can be a class attribute or any other variable declaration.
Even though we call it a variable reference this does also cover functions.
We don't need a different representation for functions, since functions in JavaScript are values which are assigned to a variable, and therefore share the same namespace, unlike Java where methods and variables can have the same name.
These references are represented by a \codeword{Variable} class.
The class contains the name of the attribute, and optionally what type it is.
The type of a variable in our implementation is set through a very simple type-inference (relative to TypeScript's type system), where we only look at if there is a \codeword{new}-expression assigned to the variable at declaration.
If a variable is initialized with a class instantiation then the type of the variable is an object of that class.
However, this simple type inference might not always correctly type variables, as variables might have explicitly declared types which are not typed with the class' type, however for simplicity, and in order to not reimplement TypeScript's entire type system, we currently ignore these in the implementation (though we will still perform a full type-check at a later stage in the compiler).

Class references are all references made to a class.
These references are references such as class declarations, instantiations of a class, usage of a class as a type, etc.
The \codeword{Class} class is a representation for class references.
The class is an extension of the \codeword{Variable} class, so it can store the name of the class, however they also have references to all the variables that are instances of the class, and optionally a superclass, which is a reference to another \codeword{Class} instance.

With an understanding of what a reference is, let us look at how we can transform the AST nodes that are references.
Transforming nodes into references mainly consists of two steps:

\begin{enumerate}
    \item Transforming declarations
    \item Transforming references
\end{enumerate}

\subsubsection{Transforming Declarations}

When transforming declarations we both create the \codeword{Variable} or \codeword{Class} instance, register them in the scope they were found, and transform the identifier in the declaration to a reference.
This reference is a special AST node.
It is represented by the \codeword{RefNode} class.
This class contains the same fields as other scoped AST nodes, \codeword{type} (which is always \codeword{"variable"} for reference nodes), \codeword{text}, \codeword{children} and \codeword{scope}.
In addition to these fields, the \codeword{RefNode} class contains two additional fields for dealing with references.
The first field is a reference to the \codeword{Variable} or \codeword{Class} instance that the \codeword{RefNode} is a reference to.
The second is a field containing the original type of the node, which is used when transforming the \codeword{RefNode}s back to the original AST\@.

When we are transforming both the declarations and the references we need to pass through the AST multiple times.
A pass through the AST will visit every node and perform a transformation.
The task of transforming declarations requires us to do three passes through the AST:

\begin{enumerate}
    \item Class declarations
    \item Class heritage
    \item Class attribute declarations
\end{enumerate}

During the first pass through the AST we register and transform all class declarations, creating the \codeword{Class} instances.
While creating the \codeword{Class} instances we also register the associated \codeword{this} as a \codeword{Variable} in the class' body scope.
Listing~\vref{lst:transform-reference-class-decl} shows an example of how a class declaration will be transformed in this step.
In the listing we can see that the \codeword{"type\_identifier"} node has been transformed to a \codeword{RefNode} instance.

%For the second pass we register class heritage.
%The second pass will not create new \codeword{Class} instances when stumbling upon the identifiers for the superclasses, but will rather look them up in the AST nodes' attached scope.
%For each class declaration we visit we will look for a class heritage node containing an extends clause.
%The extends clauses will be transformed similarly to the previous pass, but instead of creating \codeword{Class} instances we will instead look the mup in the attached scope.
%For these class declarations with contained extends clauses we will update the class declaration's \codeword{Class} instance's superclasses with the \codeword{Class} instance found from the
For the second pass we register class heritage.
Here we will visit all the class heritages and update the identifiers contained in the extends clause with \codeword{RefNode}s containing references to the appropriate \codeword{Class} instances found in the scope.
For class declarations in which we found the class heritage nodes we will update the associated \codeword{Class} instance's superclasses.
Setting these superclasses will ensure that later references to inherited attributes can be referenced and renamed properly.

The final step takes care of any class attribute declarations.
This could have been done in the same pass as with class heritage, however it was separated to make the code more readable and the flow of the transformations more understandable.
During this transformation we also perform a very simple form of type inference by checking if there is a \codeword{new}-expression in the assignment.
For the cases where there is a \codeword{new}-expression we assign the type of the variable to be an instance of the constructed class.
It is worth noting that the current implementation only transforms public field definitions.
Other declaration types could be supported in the same manner as with public field definitions, however due to a lack of time these were not implemented.

\begin{code}{javascript}{AST of a class declaration of class A before and after transforming the references. The values surrounded by angle brackets are references to \codeword{Scope}/\codeword{Class} instances.}{lst:transform-reference-class-decl}
    // Before transformation
    {
        type: 'class_declaration',
        text: 'class A ...',
        scope: <Scope 1>,
        children: [
            {
                type: 'class',
                text: 'class',
                scope: <Scope 1>,
                children: []
            }, {
                type: 'type_identifier',
                text: 'A',
                scope: <Scope 1>,
                children: [],
            }
            ...
        ]
    }

    // After transformation
    {
        type: 'class_declaration',
        text: 'class A ...'
        scope: <Scope 1>,
        children: [
            {
                type: 'class',
                text: 'class',
                scope: <Scope 1>,
                children: []
            }, {                                //
                type: 'variable',               // This node
                text: '',                       // has been
                scope: <Scope 1>,               // transformed
                children: [],                   // to a RefNode
                origType: 'type_identifier',    //
                ref: <Class 1>                  //
            },
            ...
        ]
    }
\end{code}

\subsubsection{Transforming References}

Transforming references also has several passes, more precisely two passes.

\begin{enumerate}
    \item Transforming \codeword{this} keyword
    \item Transforming the rest
\end{enumerate}

While transforming \codeword{this} references does not need to be done in a separate pass, it does simplify the process a little for next pass, as we do not have to worry about the \codeword{this} keyword as a special case for the other transformations.
The \codeword{this}-nodes are transformed to a \codeword{RefNode} like all other references, and are simply a reference to the \codeword{Variable} instances for \codeword{this}, which we created in the class declaration pass.

In the second pass the rest of the references are transformed.
Currently, the implementation will transform \codeword{new} expressions, member expressions, and type annotations.
These are all transformed in a pretty similar manner, where the identifier is found and looked up in the attached scope.
If we can't find it in the scope this usually means that it is something we can't rename, and is therefore not of interest, such as \codeword{console.log}, \codeword{number}, etc.

\subsection{Perform the Rename}\label{subsec:performing-the-rename}

With all declarations and references transformed into reference nodes we can now perform the rename.
For each class rename we can simply look up the old class name in the root scope and change the reference to the new class name.
For class attribute renames we do a lookup on the specified class' scope.
If the attribute can't be found in the class' scope we do a lookup in the optional superclass' scope.
If either the superclass is missing or we can't find it in its scope either we throw an error, informing the programmer that the attempted renaming can't be performed.

\subsection{Go back to the Original AST}\label{subsec:going-back-to-the-original-ast}

After we have performed the renaming we can go back to the original, simpler, AST\@.
For most nodes this is as simple as removing the scope property.
For \codeword{RefNode}s we have to create new AST nodes.
Fortunately we stored the original type of the \codeword{RefNode}, which means we only need to fill in the \codeword{children} and \codeword{text} properties.
The \codeword{children} property will always be an empty array, since all \codeword{RefNode}s are also leaf nodes.
The \codeword{text} property will be filled with the name of the variable or class it is a reference to.
This name can be found through the reference stored in the \codeword{RefNode} to either a \codeword{Variable} or \codeword{Class} instance.

Once we have traversed the tree and done the transformations as described we will end up with an AST tree similar to what we had initially, where all instantiation statements have been replaced by the renamed bodies of the templates they were declared to instantiate.

\subsection{Merge Classes}\label{subsec:merging-class-declarations}

After we have performed the instantiation and have returned to the original AST format we can finish off the last task necessary to close a template, namely class merging.
At this stage we will work on a package/template body which will mostly consist of a (potentially) valid TypeScript program, with the exception of possibly \codeword{addto}-statements and duplicate class declarations.
These \codeword{addto}-statements and duplicate class declarations will in this step be merged to form new classes, and remove the final construct of PT before the body is a TypeScript program.
Merging of classes mainly consists of performing the following tasks:

\begin{itemize}
    \item Grouping class declarations and \codeword{addto}-statements
    \item Verify the validity of the groups
    \item Merging the bodies of the classes
    \item Merging the class declarations
    \item Replace the old classes with the new
\end{itemize}

\subsubsection{Grouping Class Declarations and \codeword{addto}-statements}

Before we can start merging the classes we need to first group the classes and \codeword{addto}-statements which are going to be merged together.
We will do this through collecting all classes and \codeword{addto}-statements in the first layer of the package/template body.
These will then be split up into groups based on the class name.

\subsubsection{Verify the Validity of the Groups}

In order to verify that we can indeed merge these groups together we need to validate that doing so is valid.
This is mainly checking that there is at least one class declaration, as we can not perform additions to a non-existant class.

\subsubsection{Merging the Bodies of the Classes}

Now that we have groups of classes that should be merged, and these groups are valid, we can then proceed to composing new classes from these.
The first step to this is to create the new bodies of the new classes.
This task is relatively simple as we just combine the AST nodes contained within each class declaration's or \codeword{addto}-statement's body.


\subsubsection{Merging the Class Declarations}

Merging of class declarations is similar to merging the bodies, however we here instead merge the class heritage.
That is if there are classes with class heritage we will merge the optional \codeword{extends} and \codeword{implements} clauses.
As well as combining the extending classes and implemented interfaces we also insert comma nodes between each member.
This step might produce class declarations with multiple superclass, however this will be picked up by the type-checking step later on.

\subsubsection{Replace the Old Classes With the New}

Now that we have composed new classes from the merging of the old classes we can insert these back into the AST.
In order to somewhat resemble the original program we replace the first class declaration within the group with the new class, while all other members of the group are removed from the AST\@.
