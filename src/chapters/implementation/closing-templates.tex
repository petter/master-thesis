A closed template is a template that comprises only class, interface or enum declarations.
An open template on the other hand is a template that contains at least one instantiation statement.

The task of closing open packages and templates is what most of the implementation is focused around.
It is the task of performing the instantiations declared in the body of a package/template.

For instantiations without renaming the task is fairly simple.
We merely have to find the referenced template, and replace the instantiation statement with the body of said template.
Renaming on the other hand requires a bit more work.

In order to perform renaming on an instantiation we will have to perform the following tasks.
\begin{enumerate}
    \item Create a scope for the template body, and give all nodes of the body access to this scope
    \item Find all identifiers, member expressions, class declarations, etc., and replace them with a reference node
    \item Perform the rename on the scope
    \item Transform the scoped AST with reference nodes back to the original AST.
\end{enumerate}

If the template body is also open we would have to close it as well.
We want to close the nested templates before closing the upper templates, as renaming at the top level should affect all members from the nested instantiations.

Finally, once all templates have been closed we will have to perform class merging and apply any additions to classes.

In order to get a better understanding of this we will go through each step of closing a template in more detail.

%\begin{code}{}{Pseudocode for closing a template. The same code could also be used to close a package, as they are essentially equal in this regard.}{lst:pseudo-close-pacakge}
%    templates <- map from template name to body
%    FUNCTION close_template(template_body)
%        new_program <-
%            MAP syntax_node IN template_body
%                IF syntax_node IS instantiation statement THEN
%                    instantiated_template_body <- LOOKUP referenced template from instantiation statement IN templates
%                    closed_template_body <- close_template(instantiated_template_body)
%                    scoped_body <- add_scope(closed_template_body)
%
%                    renamings <- get renamings from instantiation statement
%                    FOR renaming IN
%                ELSE
%                    syntax_node
%                END IF
%            END MAP
%        RETURN new_program
%    END FUNCTION
%
%    FUNCTION
%\end{code}

\subsection{Scoping}\label{subsec:inst-scoping}

For creating scopes I chose the following node types for "making new scopes".

\begin{itemize}
    \item \codeword{class\_body}
    \item \codeword{statement\_block}
    \item \codeword{enum\_body}
    \item \codeword{if\_statement}
    \item \codeword{else\_statement}
    \item \codeword{for\_statement}
    \item \codeword{for\_in\_statement}
    \item \codeword{while\_statement}
    \item \codeword{do\_statement}
    \item \codeword{try\_statement}
    \item \codeword{with\_statement}
\end{itemize}

\subsection{Transforming Nodes to References}\label{subsec:transforming-nodes-to-references}
