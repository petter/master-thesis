%TODO sjekk at alle listings er formatert likt, aka ikke noe punktum på slutten osv.

\documentclass[UKenglish]{ifimaster}
\usepackage[utf8]{inputenc}
\usepackage[T1]{fontenc,url}
\urlstyle{sf}
\usepackage{babel,textcomp,csquotes,varioref,graphicx}
\usepackage{listings}
\usepackage{xcolor}
\usepackage[backend=biber,style=numeric-comp]{biblatex}
\usepackage{color}
\usepackage{hyperref}
\usepackage[chapter]{minted}
\usepackage{backnaur}
\usepackage{multicol}
\usepackage{duomasterforside}
\usepackage{parskip}
%\usepackage[IFI, master, LongTitle]{mnfrontpage}

\hypersetup{
    colorlinks=true,
    linktoc=all,
    linkcolor=blue
}

\title{Package Template Script}
\subtitle{An Implementation of Package Templates in TypeScript}
\author{Petter Sæther Moen}

\addbibresource{library.bib}


% Custom macros
\newcommand{\codeword}[1]{\texttt{#1}}
\newcommand{\plname}{Package Template Script}
\newcommand{\red}[1]{\textcolor{red}{#1}}

% Custom environments
% \begin{code}{programming language}{Caption}{label}
\newenvironment{code}[4][]
{\VerbatimEnvironment
    \begin{listing}
        \caption{#3}\label{#4}
    \begin{minted}[
        frame=lines,
        #1]{#2}}
 {\end{minted}\end{listing}}

\AtBeginEnvironment{minted}{%
    \renewcommand{\fcolorbox}[4][]{#4}}

\newcommand{\codeinputfile}[4]{
    \begin{listing}
        \inputminted[frame=lines]{#2}{#1}
        \caption{#3}
        \label{#4}
    \end{listing}}


\begin{document}

%%! Author = Petter
%! Date = 3/16/2021

\chapter*{Endringer fra forrige møte}

\begin{itemize}
    \item Skrive seksjon "Avoiding Indirect Multiple Inheritance".
    Section~\vref{subsubsec:avoiding-indirect-multiple-inheritance}.
\end{itemize}




%\mnfrontpage
\duoforside[dept={Department of Informatics},
  program={Informatics: Programming and System Architecture},
  long]

\frontmatter{}

\clearpage
\section*{Abstract}
\addcontentsline{toc}{section}{Abstract}

In this thesis we will explore how TypeScript can be extended with an additional language mechanism for re-use and adaptation, namely Package Templates.
We will look at how Package Templates, which was initially designed for a nominally typed language, will work in a structurally typed language like TypeScript, and what differences this makes for its usage.

Package Templates, or as it was originally called, Generic Packages, is a language mechanism first proposed by Krogdahl in 2001.
The language mechanism gives the programmer the opportunity to create collections of classes, interfaces and enums which can later be re-used and adapted.
These collections can be instantiated inside new collections, where the mechanism allows for renaming classes and its attributes, as well as merging members of the instantiated collections.
This enables the programmer to write general collections for concepts such as graphs and lists, and later adapt these to new domains with additional concepts forming collections for domains such as road systems between cities.

% TODO skrive noe mer om hva som er resultatet av arbediet, noe ala dette:
The result of the work done in this thesis is the Package Templates Script programming language, or just PTS for short, and an easily accessible compiler for the language.
This contribution will hopefully make the language mechanism, Package Templates, more accessible for newcomers, and potentially spark further research in the field.

\cleardoublepage

\section*{Acknowledgements}
\addcontentsline{toc}{section}{Acknowledgments}

I would like to thank my supervisor, Associate Professor Eyvind Wærsted Axelsen, who has made me a more critical thinker through his thorough and pedagogic feedback and has helped me gain insights I would likely have lacked without his help.

I would also like to thank my co-supervisor, Professor Stein Krogdahl, who unfortunately fell ill and passed away.
It was a true inspiration to work with someone with such vast knowledge and experience in the field of programming languages.

Finally, I would like to thank my parents who have supported and encouraged me throughout my education, and my friends who have motivated me and brightened my days in these rather challenging times.\\

\begin{flushright}
    \textbf{Petter Sæther Moen}\\*
    Oslo, 2021
\end{flushright}

\hypersetup{linkcolor=black}

\tableofcontents{}
%\listoffigures{}
%\listoftables{}
\listoflistings{}



\cleardoublepage

\hypersetup{linkcolor=blue}

\mainmatter{}


\part{Introduction and Background}\label{part:introduction}

%! Author = Petter
%! Date = 9/22/2020

\chapter{Introduction}\label{ch:introduction}

In this thesis we will be looking at the language mechanism Package Templates and the language TypeScript and how this language mechanism fits into the language.

Package Templates

\section{What is PT?}\label{sec:what-is-pt?}


\section{Purpose of Implementing PT in TS}\label{sec:purpose-of-implementing-pt-in-ts}


\section{Structure of Thesis}\label{sec:structure-of-thesis}

In part one of this thesis we will focus on getting to know the language mechanism Package Templates, and get a proper understanding of the programming language TypeScript and its ecosystem in chapter~\vref{ch:background}.

With a proper understanding of these topics we will move on to the second part of the thesis.
This part will focus on the implementation of the project.
The part starts off by planning out and designing the language PTS in chapter~\vref{ch:the-language---pts}.
PTS, short for Package Template Script, will be a superset of TypeScript with a basic implementation of the language mechanism PT\@.
Chapter~\vref{ch:planning-the-project} will then focus on making a plan for the implementation of our programming language.
Here we will perform a requirement analysis of the project and seek out an approach for the implementation.
With a plan for approach and execution we will discuss the resulting implementation of the compiler in chapter~\vref{ch:implementation}.

In the final part of this thesis we will discuss the resulting implementation.
Chapter~\vref{ch:discussion} will discuss the implementation, where we will look at whether our implementation fulfills the requirements of PT, and then look at how PT fits into a structurally typed langauge.
Finally, in chapter~\vref{ch:results} we will talk about the resulting implementation and what it can be used for.
We will also evaluate our chosen approach, and look at some potential future works within this field.

% TODO litt mer


%! Author = Petter
%! Date = 9/22/2020

\chapter{Background}\label{ch:background}

\section{Package Templates}\label{sec:package-templates}

% When modelling complex concepts like a system of cities and roads or water pipes and switches,
% it would be helpful if we had a language mechanism which could gather in a module the shared
% aspects of these problems
% e.g.\ as the concept of a graph with nodes and edges, so that this module later can be used
% to form (or could make up the kernel of) an implementation of either cities/roads or
% switches/pipes. For this language mechanism to be really helpful this requires that,
% when such a module is used (or "instantiated") in a program, we must be able to add new
% declarations at any subclass levels of the module classes, and to change names on declarations.

Krogdahl proposed Generic Packages in 2001, which is a language mechanism aimed at "large scale code reuse in object oriented languages"\cite{krogdahl:GP}.
The idea behind this mechanism was to make modules of classes, called \textit{packages}, that could later be imported and instantiated.
This would make "textual copies" of the package body, and would also allow for further expanding the classes of the packages.
Modularizing through Generic Packages made the programming more flexible as you would easily be able to write modules with a certain functionality and be able to later import it several times when there is a need for the functionality.

Generic Packages was later extended \red{(nevne SWAT-project?)} and the mechanism is now called Package Templates (while the textual program modules themselves are simply called templates).
Of special interest in this essay is that the system allows a special type of multiple inheritance (by "merging" classes from different templates), and templates may build upon other templates to any depth).
The system is not fully implemented and there exists a number of proposals for extending it.

\subsection{Syntax}\label{subsec:syntax}

In this section we will look at the syntax of Package Templates (further referred to as \emph{PT}) in a Java-like language as proposed in~\cite{jot}.

\subsubsection{Defining packages and package templates}

\emph{Packages} are defined by a set of classes similar to a normal Java package.
Package templates(later just templates for short), are defined similar except for using the keyword \emph{template}.
Listing~\vref{code:basicPT} shows an example of defining packages and templates.
The contents of a package can be used as you would with a normal Java package.

\begin{code}{java}{Defining a package P and a template T. \red{Merk at dette ikke er syntaksen til en vanlig pakke i Java. Dette må beskrives.}}{code:basicPT}
    package P {
        interface I { ... }
        class A extends I { ... }
    }

    template T {
        class B { ... }
    }
\end{code}

\subsubsection{Instantiating templates}\label{sec:inst}
Instantiating is what really makes PT useful.
When defining packages, PT allows for including already defined templates through instantiating.
Instantiation is done inside the body of a package with the use of a \verb|inst|-clause.
Instantiating a template will make "textual copies" of the  classes, interfaces and enums from the instantiated template and insert them replacing the instantiation statement at compile time.
Note that the template itself still exist and that it can be instantiated again in the same program.
%Instantiating a template will effectively copy classes, interfaces and enums from the template into the body of the new package or template at compile time.
Listing~\vref{code:inst} shows an example of instantiating a template inside a package.
The resulting package \verb|P| will then have the classes \verb|A| and \verb|B| from template \verb|T| and its own class \verb|C|.

\begin{code}{java}{Instantiating template T in package P}{code:inst}
// Before compile time instantiation of T
template T {
    class A { ... }
    class B { ... }
}

package P {
    inst T;
    class C { ... }
}

// After compile time instantiation of T
package P {
    class A { ... }
    class B { ... }
    class C { ... }
}
\end{code}

\subsubsection{Renaming}

During instantiations it is possible to rename classes (as well as interfaces and enums) and class methods. Renaming is a part of the instantiation of templates, and will only affect the copy made for this instantiation, and it is done for this copy before it replaces the \verb|inst|-statement.
Renaming is denoted by a \verb|with|-clause.
In the \verb|with|-clause one can rename classes using the following fat arrow syntax, \verb|A => B|, where class \verb|A| is renamed to \verb|B|, and you can rename class attributes with a similar arrow syntax, \verb|i -> j|, where the attribute \verb|i| is renamed to \verb|j|.
For method renaming you have to give the signature of the method so that it is possible to distinguish between overloaded versions, i.e. \verb|m1(int) -> m2(int)|\footnote{On a more technical level the compiler will find the class or attribute declaration that is gonna be renamed, and then find all name occurrences bound to this declaration and rename these.}.

\red{Det var vel altså egentlig noe historisk tull at vi skilte mellom "=>" og "->". Må gjerne bruke are en, f.eks. "->", så har man "=>" helt fri til andre formål. Sparer dette til masteroppgaven, da essayet fokuserer på hvordan PT er nå, mens masteroppgaven vil fokusere mer på "min versjon" av PT.}

Attribute renaming comes after the class renaming surrounded by a set of parentheses.
Renaming classes will also affect the signatures of any methods using this class.
Listing \vref{code:rename} shows an often used example of renaming, where a graph template is renamed to better fit a domain, in this case a road map.
When renaming the class Node the signature of the methods in Edge using this Node was also changed to reflect this, i.e. the method \verb|Node getNodeFrom()| would become \verb|City getNodeFrom()| with the class rename, and \verb|City getCityFrom()| with the method renaming.

\begin{code}{java}{Example of renaming classes during instantiation. This could be used to make the classes fit the domain of the project better.}{code:rename}
template Graph {
    class Node {
        ...
    }

    class Edge {
        Node getNodeFrom() { ... }
        Node getNodeTo() { ... }
    }

    class Graph {
        ...
    }
}

package RoadMap {
    ...
    inst Graph with
        Node => City,
        Edge => Road
            (getNodeFrom() -> getStartingCity(),
            getNodeTo() -> getDestinationCity()),
        Graph => RoadSystem;
    ...
}

\end{code}

Renaming makes it possible to instantiate templates with conflicting names of classes, or even instantiate the same templates multiple times.
Listing~\vref{code:renamingdoubleinst} shows an example of this where we instantiate the same template T twice without any issues.

\begin{code}{java}{Example of instantiating the same template twice solved by renaming.}{code:renamingdoubleinst}
template T {
    class A {
        void m() { ... }
    }
}

package P {
    inst T;
    inst T with A => B;
}

// package P after compile time instantiation and renaming
package P {
    class A {
        void m() { ... }
    }

    class B {
        void m() { ... }
    }
}
\end{code}

\subsubsection{Additions to a template}\label{sec:additions}

When instantiating a template you can also add attributes to the classes of the template, as well as extending the classes implemented interfaces and this will only apply to the current copy.
These additions are written inside an \verb|addto|-clause. Extending the class with additional attributes is done in the body of the clause, like you would in a normal Java class.
If an addition has the same signature as an already existing attribute from the instantiated template class, then the addition will override the already existing attribute, like in traditional inheritance.
Extending the list of implemented interfaces for a class can be done by suffixing the \verb|addto|-clause with \verb|implements| and the list of interfaces.
Listing~\vref{code:addition} shows an example of adding attributes and implemented interfaces to an instantiated class.
The resulting class \verb|A| in package \verb|P| would have the attribute \verb|i|, methods \verb|someMethod|, \verb|someOtherMethod| and \verb|run|, as well as implementing the interface \verb|Runnable|.

The extension of classes using the \verb|addto|-clause is done independently for each class, ignoring any inheritance-patterns. If there is a class \verb|A| with a subclass \verb|B|, they can both get extensions independently from each other. Any extensions made to class \verb|A| would of course still be inherited to class \verb|B|, as with normal Java inheritance.


\begin{code}{java}{Adding new attributes and extending the implemented interface for the instantiated class A in package P}{code:addition}
template T {
    class A {
        void someMethod() { ... }
    }
}

package P {
    inst T;
        addto A implements Runnable {
        int i;
        void someOtherMethod() { ... }
        void run() { ... }
    }
}
\end{code}


\subsection{Merging classes}

Since this is the mechanism of PT that gives rise to the theme of this essay, a sort of multiple inheritance (namely "static multiple inheritance"), we will be going a bit more in-depth in this section.

If two or more classes in the same or in different instantiations share the same name they will be merged into one class.
Through this mechanism PT achieves a form of static multiple inheritance.
If two classes don't share the same name, it is still possible to merge them by first using renaming.
Through renaming it is possible to force merging by giving the classes the same name.
In listing~\vref{code:renameclassmerging} we see an example of renaming class \verb|B| to \verb|A| to force these classes to be merged under the class name \verb|A|.
Merging the classes would simply lead to having a single class \verb|A| with the attributes from both classes.
The two classes \verb|A| and \verb|B|, from templates \verb|T1| and \verb|T2| respectively, no longer exists in package \verb|P|, but have formed the new class \verb|A|, which is a union of both.
Any pointers typed with the old \verb|A| or \verb|B| will now be pointing ot the new merged class \verb|A|.

\begin{code}{java}{Instantiation with class merging through renaming}{code:renameclassmerging}
template T1 {
    class A {
        ...
    }
}

template T2 {
    class B {
        ...
    }
}

package P {
    inst T1 with A;
    inst T2 with B => A;
}
\end{code}



%! Author = Petter
%! Date = 3/16/2021

\section{TypeScript}\label{sec:typescript}

Before we look at what TypeScript is we first need to understand JavaScript and the JavaScript ecosystem.

\subsection{JavaScript}\label{subsec:javascript}

Back in the mid 90s web pages could only be static, however there was a desire to remove this limitation and make the web a more interactive platform, as it became increasingly more accessible to non-technical users.
In order to remove this limitation, Netscape, with its Netscape Navigator browser, partnered up with Sun in order to bring the Java platform to the browser, and hired Brendan Eich to create a scripting language for the web.
Eich was tasked to create a Scheme like language with syntax similar to Java, and the language was intended to be a companion language to Java.
The language when it first released was called LiveScript, however it was later renamed to JavaScript.
This has been characterized as a marketing ploy by Netscape in order to give the impression that it was a Java spin-off.

Microsoft, with its Internet Explorer, adopted the language and named it JScript.
During this time Microsoft and Netscape would both ship new features to the language in order to increase the popularity for their respective browsers.
Because of this war between browsers the language was later handed over to ECMA International as a starting point for a standard specification for the language.
This ensured that users would get the same experience across different browsers, making the web more accessible~\cite{jswikipedia}.

A web page generally consists of three layers of technologies.
The first layer is HTML, which is the markup language that is used to structure the web page.
Second is CSS which gives our structured documents styling such as background colors and positioning.
The third and final layer is JavaScript which enables web pages to have dynamic content.
Whenever you visit a website that isn't just static information, but instead might have timely content updates, interactive maps, etc., then JavaScript is most likely involved~\cite{whatisjs}.

JavaScript, or more precisely ECMAScript, is the programming language of the web.
It is a language that conforms to the ECMAScript specification.
ECMAScript is simply a JavaScript standard, created by Ecma International, made to ensure interoperability across different browsers.
There is no official runtime or compiler for JavaScript as it is up to each browser to implement the languages runtime.
When we create a JavaScript program/script for a web page we don't compile it and transfer a binary or bytecode file for the web page to execute, instead the browser takes the raw source code and interprets it\footnote{On a more technical level, JavaScript is generally just-in-time compiled in the browser.}.

% TODO
JavaScript is a multi-paradigm language with dynamic typing...

\subsubsection{ECMAScript Versions}\label{subsubsec:ecmascript-versions}

ECMAScript versions are generally released on a yearly basis.
This release is in the form of a detailed document describing the language, ECMAScript, at the time of release.
New versions will most likely include some additions to the language, but never any breaking changes\footnote{There has been occasions where there has been minor breaking changes between ECMAScript versions, but these changes happen very rarely and the affected areas are often insignificant.}.
This is because the developer will not be able to control the environment on which the code will be executed since you can not be sure which ECMAScript version the client browser is using.
Because of this lack of control over the runtime environment it is crucial that any pre-existing language features don't have breaking changes between versions.

\subsubsection{Backwards Compatibility}\label{subsubsec:backwards-compatability}

With new ECMAScript versions comes new features, and it is up to each browser to implement these changes.
As we mentioned earlier, we do not transfer a binary to the client browser, we transfer the source code.
So when a JavaScript script uses a new ECMAScript feature it is not guaranteed to work with every client browser, since a lot of users might have older browsers installed, or the team behind the browser has not implemented the language feature yet.
To deal with this a common practice in JavaScript development is to first transpile the source code before using it in a production environment.
This transpilation step takes the source code and transpiles it into an older ECMAScript version.
In doing this you ensure that more client browser will be able to run the script.
This will rewrite the new language features, and often replace them with a function, called a \textit{polyfill}.
You can think of a polyfill as an implementation of a new language feature that you ship with your code.
These polyfills help the developer regain some control over the runtime environment on which the code will be run on, and ensures that the code will run on almost any browser as expected.

Some popular transpilers for JS to JS transpilation are Webpack and Babel, but you could also use the TypeScript compiler for this.

\subsubsection{Node.js}\label{subsubsec:node}

As of the time of writing there are mainly two ways to execute JavaScript.
You can run the program in the browser, as it was intended, or you can use a JavaScript runtime that runs on the backend, outside of the browser.
Node.js (henceforth simply referred to as Node) is the mainstream solution for the latter.
Node is a JavaScript runtime built on the JavaScript engine, V8, used by Chrome.
It is independent of the browser and can be run through a \textit{CLI} (Command-Line Interface).
One major difference from the browser runtimes is that Node also supplies some libraries for IO, such as access to the file system and the possibility to listen to HTTP requests and WebSocket events.
This makes Node a good choice for writing networking applications for instance.

We will be using the Node runtime for our compiler since it gives us access to the file system, as well as enabling the compiler to be executed through a CLI, as is the norm for most compilers.
The compiler will still also be available as a library.

\subsection{What is TypeScript?}\label{subsec:what-is-typescript}

TypeScript is a superset of JavaScript.
The language builds on JavaScript with the additions of static type definitions~\cite{tswebsite}.

All valid JavaScript programs are also valid TypeScript programs.


\part{The project}\label{part:the-project}

\chapter{The Langauge - PTS}\label{ch:the-langauge---pts}

\section{The PTS Grammar}\label{sec:the-pts-grammar}

PTS is an extension of TypeScript, and the grammar is also therefore an extension of the TypeScript grammar.
There is no published official TypeScript grammar (other than interpreting it from the implementation of the TypeScript compiler), however up until recently there used to be a TypeScript specification\cite{tsspec}.
This TypeScript specification was deprecated as it proved a to great a task to keep updated with the ever-changing nature of the language.
However, most of the essential parts are still the same.
The PTS grammar is therefore based on the TypeScript specification, and on the ESTree Specification\cite{estreespec}.

In figure~\vref{fig:pts-grammar} we can see the PTS BNF grammar.
This is not the full grammar for PTS, as I have only included any additions or changes to the original TypeScript/JavaScript grammars.
More specifically the non-terminal $\bnfpn{declaration}$ is an extension of the original grammar, where we also include package and template declarations as legal declarations.
The productions for non-terminals $\bnfpn{id}$, $\bnfpn{class declaration}$, $\bnfpn{interface declaration}$, and $\bnfpn{class body}$ are also from the original grammar.

\begin{figure}
    \begin{bnf*}
        \bnfprod{declaration}
        { \bnfsk \bnfor \bnfpn{package declaration} \bnfor \bnfpn{template declaration} }\\
        \bnfprod{package declaration}
        { \bnfts{pack} \bnfsp \bnfpn{id} \bnfsp \bnfpn{PT body} }\\
        \bnfprod{template declaration}
        { \bnfts{template} \bnfsp \bnfpn{id} \bnfsp \bnfpn{PT body} }\\
        \bnfprod{PT body}
        { \bnfts{\{} \bnfsp \bnfpn{PT body decls} \bnfsp \bnfts{\}} }\\
        \bnfprod{PT body decls}
        { \bnfpn{PT body decls} \bnfsp \bnfpn{PT body decl} \bnfor \bnfes}\\
        \bnfprod{PT body decl}
        { \bnfpn{inst statement} \bnfor \bnfpn{addto statement} \bnfor }\\
        \bnfmore{ \bnfpn{class declaration} \bnfor \bnfpn{interface declaration} }\\
        \bnfprod{inst statement}
        { \bnfts{inst} \bnfsp \bnfpn{id} \bnfsp \bnfpn{inst rename block} }\\
        \bnfprod{inst rename block}
        { \bnfts{\{} \bnfsp \bnfpn{class renamings} \bnfsp \bnfts{\}} \bnfor \bnfes }\\
        \bnfprod{class renamings}
        { \bnfpn{class rename} \bnfor \bnfpn{class rename} \bnfts{,} \bnfsp \bnfpn{class renamings} }\\
        \bnfprod{class rename}
        { \bnfpn{rename} \bnfsp \bnfpn{field rename block} }\\
        \bnfprod{field rename block}
        { \bnfts{(} \bnfsp \bnfpn{field renamings} \bnfsp \bnfts{)} \bnfor \bnfes }\\
        \bnfprod{field renamings}
        { \bnfpn{rename} \bnfor \bnfpn{rename} \bnfts{,} \bnfsp \bnfpn{field renamings} }\\
        \bnfprod{rename}
        { \bnfpn{id} \bnfsp \bnfts{->} \bnfsp \bnfpn{id} }\\
        \bnfprod{addto statement}
        { \bnfts{addto} \bnfsp \bnfpn{id} \bnfsp \bnfpn{addto heritage} \bnfsp \bnfpn{class body} }\\
        \bnfprod{addto heritage}
        { \bnfpn{class heritage} \bnfor \bnfes }\\
    \end{bnf*}
    \caption{BNF grammar for PTS. The non-terminals $\bnfpn{declaration}$, $\bnfpn{id}$, $\bnfpn{class declaration}$, $\bnfpn{interface declaration}$, and $\bnfpn{class body}$ are productions from the TypeScript grammar.
    The ellipsis in the declaration production means that we extend the TypeScript production with some extra choices.}

    \textit{Legend:} Non-terminals are surrounded by $\bnfpn{angle brackets}$.
    Terminals are in $\bnfts{typewriter font}$.
    Meta-symbols are in regular font.
    \label{fig:pts-grammar}
\end{figure}


%! Author = Petter
%! Date = 9/22/2020

\chapter{Planning the project}\label{ch:planning-the-project}

\section{What Do We Need?}\label{sec:what-do-we-need}

\begin{itemize}
    \item The ability to add custom syntax (access to the tokenizer / parser)
    \item Some semantic analysis.
\end{itemize}

\section{Syntax}\label{sec:syntax}

For the implementation of PT we need some syntax for the following:

\begin{itemize}
    \item Defining packages (\codeword{package} in PTj)
    \item Defining templates (\codeword{template} in PTj)
    \item Instantiating templates (\codeword{inst} in PTj)
    \item Renaming classes (\codeword{=>} in PTj)
    \item Renaming methods (\codeword{->} in PTj)
\end{itemize}

\codeword{template} and \codeword{inst} are both not used or reserved in the ECMAScript standard or in TypeScript, and can therefore be used in \languagename without any issues.

The keyword \codeword{package} in TS / ES is as of yet not in use, however the ECMAScript standard has reserved it for future use.
In order to "future proof" our implementation we should avoid using this reserved keyword, as it could have some conflicts with a potential future implementation of packages in ECMAScript.
It could also be beneficial to not share the keyword in order to not create confusion between ES packages and PT Packages.
\codeword{module} is also a keyword that could be used to describe a PT package, however this is also reserved in the ES standard, and should therefore also be avoided for similar reasons to \codeword{package}, to avoid confusion.
We will therefore use (\codeword{pack} or \codeword{bundle}?) instead. % TODO: find a proper name for package.

For renaming classes PTj uses \codeword{=>}, however in ES this is used in arrow-functions\cite{arrowfunction}.
To avoid confusion and a potentially ambiguous grammar we will have to choose a different syntax for renaming classes.
PTj, for historical purposes, uses a different operator (\codeword{->}) for renaming class methods, however for keeping \languagename simple we will stick to only having one common operator for renaming.

ECMAScript currently supports renaming of destructured fields using the \codeword{:}(colon) operator and aliasing imports using the keyword \codeword{as}.
Even though we opted to choose a different keyword for packages, we will here re-use the already existing \codeword{as} keyword for renaming 

%! Author = petter
%! Date = 09.10.2020

\section{Why TypeScript}\label{sec:why-typescript}



\section{Choosing the right approach}\label{sec:choosing-the-right-approach}

Before jumping into a project of this magnitude it is important to find out what approach to use. 
The end goal of this project is to extend TypeScript with the Package Templates language mechanism,
this can be achieved as following:

\begin{itemize}
    \item Making a fork of the TypeScript compiler
    \item Making a preprocessor for the TypeScript compiler
    \item Making a compiler plugin / transform
    \item Making a custom compiler from scratch
\end{itemize}

\subsection{TypeScript Compiler Fork}\label{subsec:typescript-compiler-fork}

Possible, however not as accessible as other alternatives and will make upkeep expensive.

\subsection{Preprocessor for the TypeScript Compiler}\label{subsec:preprocessor-for-the-typescript-compiler}

A lot more work than ex plugin / transformer.

\subsection{TypeScript Compiler Plugin / Transform}\label{subsec:typescript-compiler-plugin}

%Not possible at the time as the TypeScript compiler wiki specifies "TypeScript Language Service Plugins ("plugins") are for changing the editing experience only."\cite{tscplugin}.
%This might be possible in the future...

As of the time of writing this the official TypeScript compiler does not support compile time plugins. The plugins for the TypeScript compiler is, as the TypeScript compiler wiki specifies, "for changing the editing experience only"\cite{tscplugin}.
However there are alternatives that do enable compile time plugins / transformers;

\begin{itemize}
    \item ts-loader\cite{tsloadergithub}, for the webpack ecosystem
    \item Awesome Typescript Loader\cite{awesometypescriptloadergithub}, for the webpack ecosystem. Deprecated
    \item ts-node\cite{tsnodegithub}, REPL / runtime
    \item ttypescript\cite{ttypescriptgithub}, TypeScript tool TODO: Les mer på dette
    \item A compiler wrapper
\end{itemize}

Unfortunately most of these don't support custom syntax, which is one of the requirements we have in order to properly implement PT.


\subsection{Babel plugin}\label{subsec:babel-plugin}

Babel isn't strictly for TypeScript, but for JavaScript as a whole.
Will however make it very accessible as most web-projects use Babel, and the upkeep is cheap, as plugins are pretty independent from the core.

Saying that this is strictly a Babel plugin wouldn't be entirely true, as we would have to fork the Babel parser in order to include our custom syntax\cite{babelparserdocs}.
However this is all hidden away from the user, as this custom parser is a dependency of our Babel plugin.

Babel Plugin is not as nice as I first thought, it seems to be pretty hard to write third-party plugins as you have to make a parser fork for custom syntax, and there is a severe lack of documentation.
Most examples of Babel plugins mostly use internal helpers and utils, which are hard to use for third-party plugins.

TODO: Does it support having multiple custom parser?
E.g. babel-plugin-typescript + our custom babel plugin?


%! Author = petter
%! Date = 04.01.2021

\chapter{Implementation}\label{ch:implementation}

In this chapter we are going to look at the implementation of PT in TypeScript.

\section{Architecture / Parts of the compiler / PP}\label{sec:architecture}



\subsection{Lexer and Parser}\label{subsec:lexer-and-parser}

tree-sitter grammar extending tree-sitter-typescript

\subsection{Instantiation and Renaming}\label{subsec:instantiation-and-renaming}


\subsection{Verification of Templates}\label{subsec:verification-of-templates}

ts api

\subsection{Code Generation}\label{subsec:code-generation}

generate ts and compile ts to js through ts api.

\section{Notes on Performance}\label{sec:notes-on-performance}

Very slow compiler/PP because of the chosen implementation, with tree traverser for every step.


\section{Testing}\label{sec:testing}

\subsection{Lexer and Parser}\label{subsec:testing-lexer-and-parser}

Tree-sitter tests are simple \codeword{.txt} files split up into three sections, the name of the test, the code that should be parsed, and the expected parse tree in S-expressions\cite{sexprs}.

\begin{code}{typescript}{Example of tree-sitter grammar test}{lst:tree-sitter-grammar-test}
    ===========================
    Closed template declaration
    ===========================

    template T {
        class A {
            i = 0;
        }
    }

    ---
    (program
        (template_declaration
            name: (identifier)
            body: (package_template_body
                    (class_declaration
                        name: (type_identifier)
                        body: (class_body
                            (public_field_definition
                                name: (property_identifier)
                                value: (number)))))))

\end{code}

\subsection{Transpiler}\label{subsec:testing-transpiler}

jest

%! Author = Petter
%! Date = 5/14/2021

\chapter{Using PTS}\label{ch:using-pts}

Now that we have a working implementation of our compiler for PTS, let us look into how we could install and use it.
There are mainly two ways of using the PTS compiler:

\begin{itemize}
    \item Installing it globally, or
    \item Creating a PTS project
\end{itemize}

In the following sections we will look at how you can install and use the compiler for both approaches.

The PTS compiler requires you to have Node and npm installed on your computer.
For instructions on installing Node and npm I refer the reader to the npm documentations\footnote{\url{https://docs.npmjs.com/downloading-and-installing-node-js-and-npm}}.

\section{Installing and Using PTS Globally}\label{sec:installing-and-using-pts-globally}

Installing PTS globally will enable you to use PTS anywhere, and might be favorable if you are planning to create several smaller projects to test it out, or if you are not too experienced with the node ecosystem.
If you want to install the compiler globally you can do the following:

\begin{minted}{bash}
    $ npm install -g pts-lang
\end{minted}

This will give you access to the PTS compiler CLI through the command \codeword{pts-lang}.
By giving the \codeword{-{}-help} flag you will get some useful information for how to use the compiler.

\begin{minted}{text}
$ pts-lang --help
Options:
      --help                      Show help
      --version                   Show version number
  -i, --input                     Name of the input file
  -o, --output                    Name of the output file
  -v, --verbose                   Show extra information during
                                  transpilation
  -t, --targetLanguage, --target  Target language for
                                  transpilation
  -r, --run
\end{minted}

\section{Creating a PTS Project}\label{sec:creating-a-pts-project}

If you are using PTS for a specific project it might be better to set it up as a project dependency in npm.
When installed in an npm project the CLI is available to use through npm scripts or through accessing it directly from the \codeword{node\_modules} folder in your project.
The compiler can also be accessed through the API by importing it as with any other npm package.

Installing it inside an npm project will not require you to install it globally, as it will stay contained in the project.
This also means that any contributors of the project will not have to worry about installing PTS, as it will be installed when the project is set up.

To initialize an npm project you can do the following:

\begin{minted}{bash}
    $ mkdir <project name>
    $ cd <project name>
    $ npm init -y
\end{minted}

With a project set up you can install the PTS compiler as following:

\begin{minted}{bash}
    $ npm install pts-lang
\end{minted}

With the PTS compiler installed in the project you can then set up some scripts in your project's \codeword{package.json} to start and/or build the project.
Below you can see an example of a section of a \codeword{package.json} file with scripts for running and building a file:

\begin{minted}{json}
    {
      "scripts": {
        "start": "pts-lang -i src/index.pts --run",
        "build": "pts-lang -i src/index.pts -o build/index"
      }
    }
\end{minted}

The start script only runs the program, and does not emit any files, while the build script transpiles the \codeword{src/index.pts} file to JavaScript.
If you would rather have TypeScript output you can use the \codeword{-t} flag to specify this:

\begin{minted}{bash}
    pts-lang -i src/index.pts -o build/index -t ts
\end{minted}

\section{A "Real World" Example}\label{sec:a-real-world-example}

Now that we understand how to get PTS set up, let us look at how it could be used in a real world example, and how PT enables the programmer to modularize the code base even further giving great flexibility.
Note that the following example will not work with the current state of the compiler as it doesn't handle member expressions containing call expressions, such as \codeword{f().i}.
Properly handling these types of member expressions would require us to analyse the function for its return type.
The example serves as an example of how PTS could be useful given a more complete implementation for a real world problem.

The most common use of TypeScript is to create web applications.
Let us look at how PTS can help make this task easier for the programmer.
We will try to create a simple web application for displaying a Pokémon.
To do this we will use one of the most popular web frameworks, React\footnote{\url{https://reactjs.org/}}.
We could of course just display some information about a predetermined Pokémon, however, we would like to make something re-usable.
We will utilize React to create something re-usable, which we can use to display information about any Pokémon.
We do not want to have to write down information about all Pokémon, so we will fetch this information from an API, more specifically the PokéAPI\footnote{\url{https://pokeapi.co/}}.
This API lets us fetch data about all Pokémon.

\subsection{Short Introduction to React}\label{subsec:short-introduction-to-react}

React is a web framework developed by Facebook\footnote{\url{https://www.facebook.com/}}.
It aims to make creating scalable web projects easier to handle, through enabling the programmer to modularize collections of elements into \emph{components}.
These components are often created to make re-use of common elements easier, such as creating a styled button with certain features, or we could create a component to represent the entire web application.

Components can be made either through creating a function that returns some \emph{JSX}, which we call functional components, or through creating a class that extends the \codeword{Component} class, which we call class components.
JSX is a syntax extension to JavaScript which resembles HTML, but in reality is just syntactic sugar for creating React elements.
We will in this example create class components, as PT has a lot of useful functionality for adapting classes.
Class components most important method is the \codeword{render} method.
The \codeword{render} method works essentially the same way as a functional components, where you return some piece of JSX, which is what will be displayed when the component is used.

A component essentially has two sources of data, its \emph{state} and its \emph{props}.
Props, short for properties, is data passed to a component from its parents.
An example of this can be \codeword{<SomeComponent text="some text" />}, where the component \codeword{SomeComponent} got some text from its parents.
This can be accessed by the component through the \codeword{props} attribute.
The other source of data for components is the component's state.
State is a piece of data connected to the component, which similar to props can be accessed through the \codeword{state} attribute.
Unlike props, state is entirely controlled by the component itself.
The state can be updated through using the \codeword{setState} method of the \codeword{Component} class, which will also trigger a re-render of the component.

Except for the \codeword{render} method, class components also has methods for lifecycle events, such as \codeword{componentDidMount}, which is called after the initial render of the component has finished, and \codeword{componentDidUpdate}, which is called after the state of the component is updated.
These lifecycle methods are very useful for reacting to state changes, or to perform some asynchronous actions.
For a more thorough introduction to React I refer the reader to the React documentation\footnote{\url{https://reactjs.org/docs/getting-started.html}}.

\subsection{The \codeword{FetchJSON} Template}\label{subsec:the-fetchjson-template}

We will start off with the task of fetching data.
As this is something you commonly want to do in web applications it might be a good idea to separate this logic into a separate template.
Fetching data is commonly done after a component has been mounted, so we use the \codeword{componentDidMount} lifecycle method for this.
For fetching the data we use the \codeword{fetch} function from the WebAPI\footnote{\url{https://developer.mozilla.org/en-US/docs/Web/API/Fetch_API}}.

\begin{minted}{typescript}
    template FetchJSON {
        class FetchJSON extends Component {
            componentDidMount() {
                fetch(this.props.url)
                    .then(response => response.json())
                    .then(data =>
                        this.setState(state => ({...state, data}))
                    ).catch(error =>
                        this.setState(state => ({...state, error}))
                    );
            }
        }
    }
\end{minted}

The \codeword{FetchJSON} component we will fetch whatever URL we pass to it in its props and update the state with the results of the fetch.
If we for some reason should fail to fetch the data we will instead update the state with the error message we got.

\subsection{The \codeword{StateLogger} Template}\label{subsec:the-statelogger-template}

In addition to fetching data, it might be useful to have a logger, which will log all state changes to the console.
This is often useful when working with React components as we are able to see when they update, and what the state was at the time of the update.
Such a logger could then also be separated into its own template, like the following:

\begin{minted}{typescript}
    template StateLogger {
        class StateLogger extends Component {
            componentDidUpdate() {
                console.log("State updated!", this.state);
            }
        }
    }
\end{minted}

\subsection{Creating the \codeword{Pokemon} Component}\label{subsec:creating-the-pokemon-component}

Finally we would like to combine these templates into our Pokémon component, and add some logic for displaying the information.
We will do this inside of a package, so that this will produce an output:

\begin{minted}{typescript}
    pack Pokemon {
        inst FetchJSON { FetchJSON -> Pokemon };
        inst StateLogger { StateLogger -> Pokemon };
        addto Pokemon {
            render() {
                if(this.state.error) {
                    return (
                        <div>
                            <h1>An error occurred</h1>
                            <p>{this.state.error.message}</p>
                        </div>
                    );
                }

                if(this.state.data === undefined) {
                    return 'Loading...';
                }

                const name = this.state.data.name;
                const pokemonTypes = this.state.data.types;
                const image = this.state.data.sprites.front_default;
                return (
                    <div>
                        <img src={image} />
                        <h1>{name}</h1>

                        <h2>Types</h2>
                        <ul>
                            {pokemonTypes.map(pokemonType => (
                                <li>{pokemonType.name}</li>
                            ))}
                        </ul>
                    </div>
                )
            }
        }
    }
\end{minted}

We could then use our Pokémon component in our application by supplying a URL for the Pokémon to display, as seen below:

\begin{minted}{typescript}
    class App extends Component {
        render() {
            <Pokemon
                url="https://pokeapi.co/api/v2/pokemon/ditto" />
        }
    }
\end{minted}

\subsection{How the Example Benefited From PT}\label{subsec:how-the-example-benefited-from-pt}

In this example PT helped us split up our component into several smaller modules.
This enables us to later re-use these common pieces of functionality, fetching data and logging, in other components, which will make our project more scalable, and aids in shortening development time.
Modularizing these concepts also helped make the implementation of our Pokémon component less cluttered, which makes the readability of the code better.


\part{Results}\label{part:Results}

%! Author = petter
%! Date = 02.05.2021

\chapter{Discussion}\label{ch:discussion}

\chapter{Does PTS Fulfill The Requirements of PT?}\label{ch:does-pts-fulfill-the-requirements-of-pt?}

% TODO: Kanskje bedre å ha denne inn i background til PT?
\section{The Requirements of PT}\label{sec:the-requirements-of-pt}

What are the requirements of PT?

As described in~\cite{jot}

\begin{itemize}
    \item Parallel extension
    \item Hierarchy preservation
    \item Renaming
    \item Multiple uses
    \item Type parameterization
    \item Class merging
    \item Collection-level type-checking
\end{itemize}

\subsection{Parallel Extension}\label{subsec:parallel-extension}
% - \textbf{Parallel  extension:}  When  using  the  collection  C  in  a  certain  setting  we  can  add  attributes  to  A  and  B.
% These  additions  should  also  have  effect  for  the  code  of  C,  e.g.  so  that  we  by  means  of  an  A-variable  defined  in  C  can  directly  (without  casting) access the attributes added to A.
%
% - - In the graph example, assume that we have added the int variable length to Edge, and  that  n  is  a  Node-reference.
% With  this  property  we  can  conveniently  specify  directly: “n.firstEdge().length = 5;”, as firstEdge is typed with the extended Edge class.

\subsection{Hierarchy Preservation}\label{subsec:hierachy-preservation}
% - \textbf{Hierarchy  preservation:}  The  mechanism  should  allow  B  to  be  a  subclass  of  A,  and  if  additional  attributes  are  given  to  A  and  to  B,  then  the  B  with  additions  should be a subclass of the A with additions.
% Note that this will not be the case if we just use the collection C with the classes A and B and then define subclasses A’  and  B’  to  A  and  B,  respectively,  with  the  additions  we  want  in  these  subclasses.
% B’ will then not be a subclass of A’.
%
% - - In  the  compiler  example  this  is  exactly  what we need in order to be able to add attributes as explained in the example.

\begin{code}{Java}{Example of hierarchy preservation}{lst:hierarchy-preservation}
    class A {
        ...
    }
\end{code}

\subsection{Renaming}\label{subsec:renaming}

% - \textbf{Renaming:}  When  C  is  used,  we  should  be  able  to  change  the  name  of  A  and  B,  and of their attributes, so that they fit with the specific use situation.
%
% - - For  the  graph  example,  the  renaming  property  makes  it  possible  to  rename  the  nodes and edges to cities and roads.

\subsection{Multiple Uses}\label{subsec:multiple-uses}
% - \textbf{Multiple  uses:}  It  should  be  possible  to  use  the  classes  of  C  for  different  and  independent  purposes  in  the  same  program,  and  so  that  each  purpose  have  different additions and renamings.
% The compiler should be able to check that each use implies a different set of classes as if they are defined in separate hierarchies.
%
% - - In  a  program  we  may  need  the  basic  graph  structure  for  different  purposes.
In  addition  to  using  it  for  cities  and  roads,  we  could  in  the  same  program  use  it  to  form the structure of pipes and joints in a water distribution system.

\subsection{Type Parameterization}\label{subsec:type-parameterization}

% - \textbf{Type parameterization:} It should be possible to write a collection of classes that assumes the existence of classes that have some required attributes, but are not yet completely defined.
% In each use of this collection, one can provide specific classes that have at least the required attributes.
%
% - - In the compiler example we assume that the front end shall always produce Java Bytecode,  and  we  use  some  readymade  mechanism  for  packing  the  code  to  a  correctly  formatted  class  file.
% However,  a  number  of  such  packers  may  be  available,  and  we  do  not  want  to  choose  which  to  use  while  implementing  the  front end class collection.

\subsection{Class Merging}\label{subsec:class-merging}


% - \textbf{Class  merging:}  Assume  we  have  the  two  collections  C  (with  classes  A  and  B)  and D (with class E).
% When they are both used in the same program, we should be able to merge e.g. the classes A and E so that the resulting class gets the union of the attributes, and so that we via an E-variable defined within D can also directly see the A attributes (and similarly for an A-variable in C).
%
% - - Assume that we in addition to the graph collection have a collection Persons with a class Person.
% In a program handling personal relations we then want to use both collections together so that we obtain a new class, say PersonNode, which has all the  attributes  of  Node  and  Person,  and  where  a  Person-variable  p  defined  in  Persons gives access directly to the Node attributes, e.g. “p.firstEdge”.

\subsection{Collection-Level Type-Checking}\label{subsec:collection-level-type-checking}

% In  addition  to  these  properties,  it  is  important  that  such  collections  of  classes  can  be  separately type-checked.
% We also prefer a mechanism that allows only single inheritance, as  the  merge  property  described  above  to  a  large  extent  will  take  care  of  the  need  for  combining  code  from  different  sources  (for  which  purpose  multiple  inheritance  is  often  used).
% Finally, the type system should be as simple and intuitive as possible.


\section{Verifying Templates}\label{sec:pt-requirements-verifying-templates}

Talk about unsoudness of TypeScript.
Talk about unsoudness of Java~\cite{java-unsound}
Talk about since the requirement is met with Java we assume it is adequately met with TypeScript as well.



%! Author = petter
%! Date = 04.01.2021

\chapter{Difference between PTS and PTj}\label{ch:difference-between-pts-and-ptj}


\section{Nominal vs. Structural Typing}\label{sec:nominal-vs-structural-typing}

%Much taken from TAPL.

One of the most notable differences between PTS and PTj are the underlying languages type systems.
PTS, as an extension of TypeScript, has structural typing, while PTj on the other hand, an extension of Java, has nominal typing.

Nominal and structural are two major categories of type systems.
Nominal is defined as "being something in name only, and not in reality" in the Oxford dictionary.
Nominal types area as the name suggest, types in name only, and not in the structure of the object.
They are the norm in mainstream programming languages, such as Java, C, and C++.
A type could be \codeword{A} or \codeword{BinTree}, and checking whether an object conforms to a type restriction, is to check that the type restriction is referring to the same named type, or a subtype.
Structural types on the other hand is not tied to the name of the type, but to the structure of the object.
These are not as common in mainstream programming languages, but are very prominent in research literature.
However, in more recent (mainstream) programming languages, such as Go, TypeScript and Julia(at least for implicit typing), structural typing is becoming more and more common.
A type in a structurally typed programming language, are often defined as records, and could for example be \codeword{{ name: string }}.

\subsection{Advantages of Nominal Types}\label{subsec:advantages-of-nominal-types}

\subsubsection{Subtypes}\label{subsubsec:subtypes}

In nominal type systems it is trivial to check if a type is a subtype of another, as this has to be explicitly stated, while in structural type systems this has to be structurally checked, by checking that all members of the super type, are also present in the subtype.
Because of this each subtype relation only has to be checked once for each type, which makes it easier to make a more performant type checker for nominal type systems.
However, it is also possible to achieve similar performance in structurally typed languages through some clever representation techniques. % TODO: Maybe find an article describing this
We can see an example of subtype relations in both nominal and structural type systems, in a Java-like language, in listing~\vref{lst:subtype}.
It is important to note that even though \codeword{C} is a \emph{subtype} of \codeword{A} in a structural language, it is not a \emph{subclass} of \codeword{A}.

\begin{code}{Java}[label={lst:subtype}, caption={Example of subtype relations in nominal and structural typing, with a Java-like language.}, language=Java]
    // Given the class A
    class A {
        void f() { ... }
     }

    // A subtype, B, in nominal typing
    class B extends A { ... }

    // A subtype, C, in structural typing
    class C {
        void f() { ... }
        int g() { ... }
    }
\end{code}

\subsubsection{Recursive types}\label{subsubsec:recursive-types}

Recursive types are types that mention itself in its definition, and are widely used in datastructures, such as lists and trees.
Another advantage in nominal typing is how natural and intuitive recursive types are in the type system.
Referring to itself in a type definition is as easy as referring to any other type.
It is however just as easy to do this in structural type systems as well, however for calculi such as type safety proofs, recursive types come for free in nominal type systems, while it is a bit more cumbersome in structurally typed systems, especially with mutually recursive types\cite{tapl}.
Listings~\vref{lst:ts-recursive-type} and~\vref{lst:java-recursive-type} shows the use of recursive types in TypeScript(structurally typed) and Java(nominally typed), respectively.

\begin{lstlisting}[label={lst:ts-recursive-type}, language=TypeScript]
    interface BinTree<T> {
        getData(): T;
        getChildren(): [BinTree<T> | null, BinTree<T> | null];
    }
\end{lstlisting}

\begin{lstlisting}[label={lst:java-recursive-type}, language=Java]
    interface BinTree<T> {
        T getData();
        Pair<BinTree<T>, BinTree<T>> getChildren();
    }
\end{lstlisting}

\subsubsection{Runtime Type Checking}\label{subsubsec:runtime-type-checking}

Often runtime-objects in nominally typed languages are tagged with the types(a pointer to the "type") of the object.
This makes it cheap and easy to do runtime type checks, like in upcasting or doing a \codeword{instanceof} check in Java.
% TODO: Skriv mere

\subsection{Advantages of Structural Types}\label{subsec:advantages-of-structural-types}

\subsubsection{Tidier and More Elegant}\label{subsubsec:tidier-and-more-elegant}

Structural types carry with it all the information needed to understand its meaning.

\subsubsection{Advanced Features}\label{subsubsec:advanced-type-features}

Type abstractions(parametric polymorphism, ADTs, user-defined type operators, functors, etc), these do not fit nice into nominal type systems.

\subsubsection{More General Functions/Classes}\label{subsubsec:more-general-functions}

See~\cite{malayeri}.

\subsection{What Difference Does This Make For PT?}\label{subsec:what-difference-does-this-make-for-pt?}

We are not going to go further into comparing nominal and structural type systems and "crown a winner", as there are a lot useful scenarios for both.
We will instead look more closely into how a structural type system fits into PT, and what differences this makes to the features, and constraints, of this language mechanism.

%Often runtime-objects are tagged with the types, which are useful for multiple things like doing runtime instanceof checks. (This can also be used in structural typing)
%It is "easier" to check sub-type relations in nominal type systems, even though you will still have to do a structural comparison, you only have to do this calculation once per type.
%    Even though it is easier to achieve better performance for nominal vs structural types, it is still possible to achieve close to if not equal performance in structural type systems, through some representation techniques(Consider finding a article on this).
%A questionable advantage for nominal typing is that they prevent "spurious subsumption", that is using a structurally equal type in a place where it should not be used.
%    I would argue that there are more cases where this is a compromise than a benefit, and there are better ways to prevent this from happening, such as single-constructor datatypes or ADTs.
%
%nominal cons:
%
%structural pros:
%Structural types are "tidier and more elegant".
%    They carry all the information to understand the meaning of the type.
%    In nominal systems we are working with some global collection of names, which can make it hard to understand some types.
% Structural types can help to make a method more general, reduce code duplication. See is structural subtyping useful.
%
%
%structural cons:
%




\begin{lstlisting}[label={lst:nominal-typing-example}, language=Java]
    // Given the following class definitions for A, B and C:
    class A {
        void f() {
            ...
        }
    }

    class B extends A {
        ...
    }

    class C {
        void f() {
            ...
        }
    }

    // And a consumer with the following type:
    void g(A a) { ... }

    // Would result in the following
    g(new A()); // Ok
    g(new B()); // Ok
    g(new C()); // Error, C not of type A
\end{lstlisting}


\begin{lstlisting}[label={lst:structural-typing-example2}, language=Java]
    // Given the same class definitions and the same consumer as in the example above.
    // Would result in the following
    g(new A()); // Ok
    g(new B()); // Ok
    g(new C()); // Ok, because C is structurally equal to A
\end{lstlisting}

\subsection{Which Better Fits PT?}\label{subsec:which-better-fits-pt?}

%In PTj(specifically the version with required types) there are elements of structural typing for the implementation of required types.
%This is because PT doesn't allow externally declared classes, interfaces, etc., so in order to be able to have constraints on generic types structural typing was needed.
%I would therefore argue that a structurally typed language is a better fit for PT, because Ockhams razor / keeping it tidier.



\chapter{Results}\label{ch:results}

\section{Future Works}\label{sec:future-works}



\backmatter{}
\printbibliography
\end{document}
