%TODO sjekk at alle listings er formatert likt, aka ikke noe punktum på slutten osv.

\documentclass[UKenglish]{ifimaster}
\usepackage[utf8]{inputenc}
\usepackage[T1]{fontenc,url}
\urlstyle{sf}
\usepackage{babel,textcomp,csquotes,varioref,graphicx}
\usepackage{listings}
\usepackage{xcolor}
\usepackage[backend=biber,style=numeric-comp]{biblatex}
\usepackage{color}
\usepackage{hyperref}
\usepackage[chapter]{minted}
\usepackage{backnaur}
\usepackage{multicol}
\usepackage{duomasterforside}
\usepackage{parskip}
%\usepackage[IFI, master, LongTitle]{mnfrontpage}

\hypersetup{
    colorlinks=true,
    linktoc=all,
    linkcolor=blue
}

\title{Package Template Script}
\subtitle{An Implementation of Package Templates in TypeScript}
\author{Petter Sæther Moen}

\addbibresource{library.bib}


% Custom macros
\newcommand{\codeword}[1]{\texttt{#1}}
\newcommand{\plname}{Package Template Script}
\newcommand{\red}[1]{\textcolor{red}{#1}}

% Custom environments
% \begin{code}{programming language}{Caption}{label}
\newenvironment{code}[4][]
{\VerbatimEnvironment
    \begin{listing}
        \caption{#3}\label{#4}
    \begin{minted}[
        frame=lines,
        #1]{#2}}
 {\end{minted}\end{listing}}

\AtBeginEnvironment{minted}{%
    \renewcommand{\fcolorbox}[4][]{#4}}

\newcommand{\codeinputfile}[4]{
    \begin{listing}
        \inputminted[frame=lines]{#2}{#1}
        \caption{#3}
        \label{#4}
    \end{listing}}


\begin{document}

%! Author = Petter
%! Date = 3/16/2021

\chapter*{Endringer fra forrige møte}

\section*{Background}

\begin{itemize}
    \item Skrive litt mer JavaScript background, og forklare hvordan språkets prototype-baserte objekt-orientering fungerer.
    Section~\vref{subsec:javascript}.
    \item Skrive litt om hva TypeScript er.
    Section~\vref{subsec:what-is-typescript}.
\end{itemize}

\section*{Planning the project}

\begin{itemize}
\end{itemize}

\section*{Implementation}

\begin{itemize}
\end{itemize}

\section*{Does PTS Fulfill the Requirements of PT?}

\begin{itemize}
\end{itemize}



%\mnfrontpage
\duoforside[dept={Department of Informatics},
  program={Informatics: Programming and System Architecture},
  long]

\frontmatter{}

\clearpage
\section*{Abstract}
\addcontentsline{toc}{section}{Abstract}

In this thesis we will explore how TypeScript can be extended with an additional language mechanism for re-use and adaptation, namely Package Templates.
We will look at how Package Templates, which was initially designed for a nominally typed language, will work in a structurally typed language like TypeScript, and what differences this makes for its usage.

Package Templates, or as it was originally called, Generic Packages, is a language mechanism first proposed by Krogdahl in 2001.
The language mechanism gives the programmer the opportunity to create collections of classes, interfaces and enums which can later be re-used and adapted.
These collections can be instantiated inside new collections, where the mechanism allows for renaming classes and its attributes, as well as merging members of the instantiated collections.
This enables the programmer to write general collections for concepts such as graphs and lists, and later adapt these to new domains with additional concepts forming collections for domains such as road systems between cities.

\cleardoublepage

\section*{Acknowledgements}
\addcontentsline{toc}{section}{Acknowledgments}

I would like to thank my supervisor, Professor Eyvind Wærsted Axelsen, who has made me a more critical thinker through his thorough and pedagogic feedback and has helped me gain insights I would likely have lacked without his help.

I would also like to thank my co-supervisor, Professor Stein Krogdahl, who unfortunately fell ill and passed away.
It was a true inspiration to work with someone with such vast knowledge and experience in the field of programming languages.

Finally, I would like to thank my parents who have supported and encouraged me throughout my education, and my friends who have motivated me and brightened my days in these rather challenging times.

\hypersetup{linkcolor=black}

\tableofcontents{}
%\listoffigures{}
%\listoftables{}
\listoflistings{}



\cleardoublepage

\hypersetup{linkcolor=blue}

\mainmatter{}


\part{Introduction and Background}\label{part:introduction}

%! Author = Petter
%! Date = 9/22/2020

\chapter{Introduction}\label{ch:introduction}

In this thesis we will be looking at the language mechanism Package Templates and the language TypeScript and how this language mechanism fits into TypeScript.

Package Templates is a language mechanism created at the University of Oslo, at the Institute for Informatics.
The language mechanism is a mechanism for reuse and adaptation, where you are able to define collections of classes, interfaces and enums.
These collections can then be instantiated at a later time, in a different context, where we can tailor the collections' content to fit its use.
Package Templates was first proposed by Krogdahl in 2001~\cite{krogdahl:GP}, and was at the time called Generic Packages.
Since then further proposals have been made to the language mechanism, and it is now known as Package Templates, or PT for short.
There have been several implementations for PT in many languages.
One of these implementations is PTj, which implemented PT in Java.

TypeScript is a superset of JavaScript, the programming language of the web.
It extends JavaScript with the addition of static type definitions.
These type definitions serve as documentation and helps you validate that your program is working correctly~\cite{tswebsite}.

We will in this thesis look at how Package Templates can be implemented in a language like TypeScript.
Here we will discuss the different approaches that can be taken when working with a project such as this, and how an implementation like this can be carried out.


The purpose of implementing PT in TypeScript is to look at how a language mechanism like PT would suit into a language like TypeScript.
Most interesting is probably TypeScript's structural type system, and how this mechanism will work in this context, where other implementations so far has only been conducted in nominally typed languages.
It will also be interesting to see how PT can be used in the context of the web, with its vast variety of frameworks and libraries.

\section{Research Questions}\label{sec:research-questions}

Implementing Package Templates into a language like TypeScript give rise to some interesting research questions which we will try to answer in this thesis:

\begin{itemize}
    \item \textbf{RQ1}: How does the language mechanism Package Templates fit into TypeScript?
    \item \textbf{RQ2}: Does structural typing change how the core of Package Templates works?
\end{itemize}

\section{Contributions}\label{sec:contributions}

This thesis' main contribution is the PTS compiler.
It is easily accessible through the Node package manager.
This makes it easier to try out the PTS language, but more importantly the PT language mechanism.
Having easy access to a language with PT might make adoption of the language mechanism greater, and spark new research within the field.

By making the parser for the language separated from the compiler this also contributes to making creations of tools for the language more accessible.
While we have in this project used the parser solely for producing a parse tree for our compiler, this parser could also be used to make other tooling, such as syntax highlighting or a language server.

The final contribution this thesis makes is conveying how to approach related projects.
We show in this thesis how someone can approach extending a language by utilizing the grammar extending capabilities of tree-sitter, and how tree-sitter can be used to create a compiler.

\section{Chapter Overview}\label{sec:chapter-overview}

% TODO fyll inn

\textbf{Chapter 2} will give the reader an introduction to the Package Templates language mechanism and the programming language TypeScript.
We will also look into TypeScript's underlying language JavaScript, and its ecosystem.

\textbf{Chapter 3} will present the programming language PTS, which is a superset of TypeScript with the addition of Package Templates.
Here we will look at the grammar of the language as well as an example program.

\textbf{Chapter 4} is focused around the planning phase of the project.
This planning will make a decision to whether or not we will need to go for TypeScript as our host language, or if we could opt for the simpler underlying language, JavaScript.
We will look at the requirements for our project and look at the different approaches we could use to implement PTS, as well as making a decision for which approach is the most suitable for our project.

\textbf{Chapter 5} is all about the implementation of our compiler for PTS\@.
Here we will look at the methodology used during the implementation phase, as well as going into detail about how the compiler was implemented.
As the compiler is not fully implemented we will also talk about what remains to be implemented, and how this could be implemented to complete the implementation.
% TODO kanskje noe om limitations?

\textbf{Chapter 6} is the first chapter of the results part of the thesis.
Here we will discuss and evaluate the PTS programming language, whether it is a "true" implementation of PT, and how PT is affected by a structurally typed language.

\textbf{Chapter 7} will look at our results.
We will look at how our contribution, the PTS language, can be installed and used.
We will finish the chapter by looking at some proposals for future works within this field.

\section{Project Source Code}\label{sec:project-source-code}

The source code for the implementation of the PTS language is split up into two GitHub repositories, one for the parser of the project, and one for the compiler.
The parser's source code can be found at \url{https://github.com/petter/tree-sitter-pts}.
Source code for the compiler can be found at \url{https://github.com/petter/pts-lang}.


%In part one of this thesis we will focus on getting to know the language mechanism Package Templates, and get a proper understanding of the programming language TypeScript and its ecosystem in chapter~\vref{ch:background}.
%
%With a proper understanding of these topics we will move on to the second part of the thesis.
%This part will focus on the implementation of the project.
%The part starts off by planning out and designing the language PTS in chapter~\vref{ch:the-language---pts}.
%PTS, short for Package Template Script, will be a superset of TypeScript with a basic implementation of the language mechanism PT\@.
%Chapter~\vref{ch:planning-the-project} will then focus on making a plan for the implementation of our programming language.
%Here we will perform a requirement analysis of the project and seek out an approach for the implementation.
%With a plan for approach and execution we will discuss the resulting implementation of the compiler in chapter~\vref{ch:implementation}.
%
%In the final part of this thesis we will discuss the resulting implementation.
%Chapter~\vref{ch:discussion} will discuss the implementation, where we will look at whether our implementation fulfills the requirements of PT, and then look at how PT fits into a structurally typed langauge.
%Finally, in chapter~\vref{ch:results} we will talk about the resulting implementation and what it can be used for.
%We will also evaluate our chosen approach, and propose some potential future works within this field.

% TODO litt mer


%! Author = Petter
%! Date = 9/22/2020

\chapter{Background}

\section{Package Templates}

\section{TypeScript}


\part{The project}\label{part:the-project}

\chapter{The Language - PTS}\label{ch:the-language---pts}

In this chapter we will introduce the programming language Package Template Script, henceforth just referred to as PTS\@.
Here we will make decisions about the syntax of the language, whether we can keep most of the syntax of the original PT proposal, or if we will have to make some adjustments to avoid concept confusion and an ambiguous grammar.

\section{Syntax}\label{sec:syntax}

For the implementation of PT we need a way to express the following language constructs:

\begin{itemize}
    \item Defining packages (\codeword{package} in PT)
    \item Defining templates (\codeword{template} in PT)
    \item Instantiating templates (\codeword{inst} in PT)
    \item Specifying renaming for an instantiation (\codeword{with} in PT)
    \item Renaming classes (\codeword{=>} in PT)
    \item Renaming class attributes (\codeword{->} in PT)
    \item Additions to classes (\codeword{addto} in PT)
\end{itemize}

\codeword{template}, \codeword{addto}, and \codeword{inst} are all not in use nor reserved in the ECMAScript standard or in TypeScript, and can therefore be used in \plname{} without any issues.

The keyword \codeword{package} in TS/JS is, as of yet, not in use, however the ECMAScript standard has reserved it for future use.
In order to "future proof" our implementation we should avoid using this reserved keyword, as it could have some conflicts with a potential future implementation of packages in ECMAScript.
It could also be beneficial to not share the keyword in order to avoid creating confusion between the future ES packages and PT Packages.
\codeword{module} is also a keyword that could be used to describe a PT package, however this is already used in the ES standard, and should therefore also be avoided in order to avoid confusion.
We will therefore use \codeword{pack} instead.

Renaming in PT uses \codeword{=>}(fat-arrow) for renaming classes, and \codeword{->}(thin-arrow) for renaming class attributes.
PT, for historical purposes, used two different operators for renaming classes and methods, however in more recent PT implementations, such as~\cite{Isene2018}, a single common operator is used for both.
We will do as the latter, and only use a single common operator for renaming.
Another reason for rethinking the renaming syntax is the fact that the \codeword{=>}(fat-arrow) operator is already in use in arrow functions~\cite{arrowfunction}, and reusing it for renaming could potentially produce an ambiguous grammar, or the very least be confusing to the programmer.
JavaScript currently supports renaming of destructured attributes using the \codeword{:}(colon) operator and aliasing imports using the keyword \codeword{as}.
We could opt to choose one of these for renaming in PTS as well, however in order to keep the concepts separated, as well as making the syntax more familiar for Package Template users, we will go for the \codeword{->}(thin-arrow) operator.

The \codeword{with} keyword is currently in use in JavaScript for \codeword{with}-statements~\cite{with-statement}.
With it being a statement, we could still use it and not end up with an ambiguous grammar, however as with previous keywords, we will avoid using it in order to minimize concept confusion.
Instead of this we will contain our instantiation renamings inside a block-scope (\codeword{\{ \}}).
Field renamings for a class will remain the same as in PT, being enclosed in a set of parentheses (\codeword{( )}).

Another change we will make to renaming is to remove the requirement of having to specify the signature of the method being renamed.
This was necessary in PT as Java supports overloading, which means that several methods could have the same name, or a method and a field.
Method overloading is not supported in JavaScript/TypeScript, and we do therefore not need this constraint.

\section{The PTS Grammar}\label{sec:the-pts-grammar}

Now that we have made our choices for keywords and operators we can look at the grammar of the language.

PTS is an extension of TypeScript, and the grammar is therefore also an extension of the TypeScript grammar.
There is no published official TypeScript grammar (other than interpreting it from the implementation of the TypeScript compiler), however up until recently there used to be a TypeScript specification~\cite{tsspec}.
This TypeScript specification was deprecated as it proved a too great a task to keep updated with the ever-changing nature of the language.
However, most of the essential parts are still the same.
The PTS grammar is therefore based on the TypeScript specification, and on the ESTree Specification~\cite{estreespec}.

In figure~\vref{fig:pts-grammar} we can see the BNF grammar for our language.
This is not the full grammar for PTS, as I have only included any additions or changes to the original TypeScript/JavaScript grammars.
More specifically the non-terminal $\bnfpn{declaration}$ is an extension of the original grammar, where we also include package and template declarations as legal declarations.
The non-terminals $\bnfpn{id}$, $\bnfpn{class declaration}$, $\bnfpn{interface declaration}$, and $\bnfpn{class body}$ are also from the original grammar.

\begin{figure}
    \begin{bnf*}
        \bnfprod{declaration}
        { \bnfsk \bnfor \bnfpn{package declaration} \bnfor \bnfpn{template declaration} }\\
        \bnfprod{package declaration}
        { \bnfts{pack} \bnfsp \bnfpn{id} \bnfsp \bnfpn{PT body} }\\
        \bnfprod{template declaration}
        { \bnfts{template} \bnfsp \bnfpn{id} \bnfsp \bnfpn{PT body} }\\
        \bnfprod{PT body}
        { \bnfts{\{} \bnfsp \bnfpn{PT body decls} \bnfsp \bnfts{\}} }\\
        \bnfprod{PT body decls}
        { \bnfpn{PT body decls} \bnfsp \bnfpn{PT body decl} \bnfor \bnfes}\\
        \bnfprod{PT body decl}
        { \bnfpn{inst statement} \bnfor \bnfpn{addto statement} \bnfor }\\
        \bnfmore{ \bnfpn{class declaration} \bnfor \bnfpn{interface declaration} }\\
        \bnfprod{inst statement}
        { \bnfts{inst} \bnfsp \bnfpn{id} \bnfsp \bnfpn{inst rename block} }\\
        \bnfprod{inst rename block}
        { \bnfts{\{} \bnfsp \bnfpn{class renamings} \bnfsp \bnfts{\}} \bnfor \bnfes }\\
        \bnfprod{class renamings}
        { \bnfpn{class rename} \bnfor \bnfpn{class rename} \bnfts{,} \bnfsp \bnfpn{class renamings} }\\
        \bnfprod{class rename}
        { \bnfpn{rename} \bnfsp \bnfpn{attribute rename block} }\\
        \bnfprod{attribute rename block}
        { \bnfts{(} \bnfsp \bnfpn{attribute renamings} \bnfsp \bnfts{)} \bnfor \bnfes }\\
        \bnfprod{attribute renamings}
        { \bnfpn{rename} \bnfor \bnfpn{rename} \bnfts{,} \bnfsp \bnfpn{attribute renamings} }\\
        \bnfprod{rename}
        { \bnfpn{id} \bnfsp \bnfts{->} \bnfsp \bnfpn{id} }\\
        \bnfprod{addto statement}
        { \bnfts{addto} \bnfsp \bnfpn{id} \bnfsp \bnfpn{addto heritage} \bnfsp \bnfpn{class body} }\\
        \bnfprod{addto heritage}
        { \bnfpn{class heritage} \bnfor \bnfes }\\
    \end{bnf*}
    \caption{BNF grammar for PTS. The non-terminals $\bnfpn{declaration}$, $\bnfpn{id}$, $\bnfpn{class declaration}$, $\bnfpn{interface declaration}$, and $\bnfpn{class body}$ are productions from the TypeScript grammar.
    The ellipsis in the declaration production means that we extend the TypeScript production with some extra choices.}

    \textit{Legend:} Non-terminals are surrounded by $\bnfpn{angle brackets}$.
    Terminals are in $\bnfts{typewriter font}$.
    Meta-symbols are in regular font.
    \label{fig:pts-grammar}
\end{figure}

\section{Example Program}\label{sec:example-program}

Listing~\vref{lst:pts-example-inst-renaming-addto} shows an example of a program in PTS\@.
This program showcases the basics of defining packages and templates, and how instantiation, renaming and additions can be applied in the language.
We also have a similar program at the bottom, showing how this is done in PT\@.
While both the basic instantiation and additions stay pretty much the same, renaming does have some interesting differences.
We can see that in the PT example we have to specify the signature of methods we are renaming, while in the PTS example it is enough to just specify the names of the methods.

\begin{code}{typescript}{An example program with instantiation, renaming, and addition-classes in PTS vs. PT}{lst:pts-example-inst-renaming-addto}
    // PTS
    template T {
        class A {
            function f() : string {
                ...
            }
        }
    }

    pack P {
        inst T { A -> A (f -> g) };
        addto A {
            i : number = 0;
        }
    }

    // PT
    template T {
        class A {
            String f() {
                ...
            }
        }
    }

    package P {
        inst T with A => A (f() -> g());
        addto A {
            int i = 0;
        }
    }
\end{code}


%! Author = Petter
%! Date = 9/22/2020

\chapter{Planning the Project}\label{ch:planning-the-project}

Before we start the implementation of our language we first need to do some planning.
We know we are going to be creating a programming language, a superset of TypeScript with the addition of Package Templates.
However, we might want to look at if creating a superset of TypeScript is the way to go, or if keeping it simple and extending JavaScript is a better call.
We might also want to see if it is needed to create a language at all, or if we are able to create a TypeScript library which can achieve the functionality of PT instead.
There are a lot of approaches we can take for implementing our language, so we will have to map out the requirements for our desired approach.
Lastly we will have to look at all the different approaches we can take, and see which approach is right for this project.

This planning phase is crucial for the success of the project, as starting off on the wrong approach for the wrong language would set us back immensely.

\section{TypeScript vs. JavaScript}\label{sec:typescript-vs-javascript}

When extending TypeScript you might be asking yourself if it is truly necessary, is it better to keep it simple and just extend JavaScript instead?
This is something we need to find out before going any further with the planning of our project.

\subsection{Type-checking Templates}\label{subsec:why-typescript-verifying-templates}

One of the requirements of PT is that it should be possible to type-check each template separately.
There is no easy way to type-check JavaScript code without executing it and looking for runtime errors.
Even if some JavaScript program successfully executes without throwing any errors, we can still not conclude that the program does not contain any type errors.
TypeScript on the other hand, with the language being statically typed, we can, at least to a much larger extent, verify if some piece of code is type safe.
Because of this trait TypeScript is the better candidate for our language.

Now it should be noted that due to TypeScripts type system being unsound one could argue that this requirement of PT is not met.
While this is true it still outperforms JavaScript on this remark, and we will later in section~\vref{subsubsec:implementation-collection-level-type-checking} discuss more in-depth to what extent this requirement is met.

\subsection{Renaming}\label{subsec:ts-vs-js-renaming}

Renaming is a hard task.
In order to perform a (safe) renaming we will need to find the declaration and all references to this declaration and rename these.
Doing this at compile time would mean that we will have to implement a type system of sorts, since this will help us identify references.
This is also one of the reasons for why TypeScript is a better candidate than JavaScript, as TypeScript is statically typed, meaning the type of a variable is known at compile-time, while JavaScript is dynamically typed, where the type of a variable is first known at run-time.
While TypeScript generally allows us to determine the types of variables at compile time, this is not always the case, since it is possible for the programmer to explicitly type a variable with \codeword{any}, a catch-all type which effectively bypasses type-checking.
This means that we can still run into the same issues as we would in a JavaScript program, and not be able to perform a safe renaming, however in cases such as these where the programmer has explicitly chosen to bypass the type-check, it might then also be acceptable to not offer renaming of \codeword{any}-typed variables.

\subsection{Language Choice Conclusion}\label{subsec:langauge-choice-conclusion}

Since TypeScript would enable us to perform type-checks on templates, as well as making it possible to perform safe renamings we will in this project opt to go for TypeScript as our choice of language.
% TODO kanskje litt mer her

\section{What Do We Need?}\label{sec:what-do-we-need}

There are a lot of approaches one can take when working with TypeScript, however due to the nature of this project there are some restrictions we have to abide by.
Our approach should allow the following:

\begin{itemize}
    \item The ability to add custom syntax (access to the tokenizer / parser)
    \item Enable us to do semantic analysis.
\end{itemize}

In addition to these we would also like to look for some other desirable traits for our implementation:

\begin{itemize}
    \item Loosely coupled implementation (So that new versions of typescript not necessarily breaks our implementation).
    %\item Mer? % TODO
\end{itemize}


\section{Approach}\label{sec:choosing-the-right-approach}

Before jumping into a project of this magnitude it is important to find out what approach to use. 
The goal of this project is to extend TypeScript with the Package Templates language mechanism, this could be achieved by one of the following methods:

\begin{itemize}
    \item Implementing as a library
    \item Making a preprocessor for the TypeScript compiler
    \item Making a compiler plugin/transform
    \item Making a fork of the TypeScript compiler
    \item Making a custom compiler
\end{itemize}

%\section{Implementing PT as a TS library}\label{sec:implementing_pt_as_a_ts_library}
%
%In order to implement PT we need to be able to handle the following:
%
%\begin{itemize}
%    \item Defining templates
%    \item Instantiating templates
%    \item Renaming classes
%    \item Renaming class attributes
%    \item Merging classes
%\end{itemize}
%
%\subsection{Defining Templates}\label{sub:defining_templates}
%
%Templates could be defined as an ECMAScript class, where each member of the template is a static attribute.
%
%\subsection{Instantiating Templates}\label{sub:instantiating_templates}
%
%\subsection{Renaming Classes}\label{sub:renaming_classes}
%
%Since each class in a template is just a static member, we could create a new template where we use the new name for our class as the attribute name, and point to the class from the "old" template.
%
%TODO: Because of ES having open classes this could lead to unwanted side-effects.
%Might need to look into a different solution for this.
%
%\subsection{Renaming Class Attributes}\label{sub:renaming_class_attribtues}
%
%Maybe impossible?
%
%\subsection{Merging Classes}\label{sub:merging_classes}
%
%For merging of the types you would use the built-in decleration merging~\cite{declerationmerging}.
%Implementation merging is also possible because ECMAScript has open classes.
%For implementation merging you would create an empty class which has the type of the merged declarations, and then assign the fields and methods from the merging classes to this class.

% TODO Les over og se om dette er brukbart

\subsection{Implementing PT as a TS Library}\label{subsec:implementing-pt-as-a-ts-library}

One of the first approaches we need to check out is if we are able to achieve the functionality of PT, without having to create a compiler.
Creating a library would presumably be an easier task than having to create a compiler, and it would also make it easier to use, as the programmer would not have to install a compiler in order to get the PT functionality.
In order to implement PT we need to be able to handle the following:

\begin{itemize}
    \item Defining templates
    \item Renaming classes and class attributes
    \item Instantiating templates
    \item Merging classes
\end{itemize}

\subsubsection{Defining Templates}\label{subsubsec:defining-templates}

For defining templates we would like a construct that can wrap our template classes in a scope.
We will also need to be able to reference the template.
JavaScript has three options for this, an array, an object or a class.
It should however also be possible to inherit from classes in your own template, which rules out both arrays and objects, as there is no way of referencing other members during definition of the array/object.
Templates should therefore be defined as classes, where each member of the template is an attribute of the template class.
In listing~\vref{code:libraryimpl-template} we see an example of how this could be done.

%We are making the templates classes static in order to be able to rename them, see section~\vref{sub:renaming_classes}.

\codeinputfile{template-example-without-decorator.ts}{typescript}{Example of defining a template in a library implementation.}{code:libraryimpl-template}

\subsubsection{Renaming Classes and Class Attributes}\label{subsubsec:renaming-classes-and-class-attributes}

Renaming of classes is possible to an extent.
Since we made the classes static attributes of the template class we could easily just create a new static field on the template class and use the \codeword{delete}-op~\footnote{An operator in JavaScript for removing a property of an object. See~\url{https://developer.mozilla.org/en-US/docs/Web/JavaScript/Reference/Operators/delete}.} to remove the old field.
We can see an example of this in listing~\vref{code:libraryimpl-template-renaming}.


\codeinputfile{class-renaming-example.ts}{typescript}{Example of renaming a template class}{code:libraryimpl-template-renaming}

Even though we were able to give the class a "new name", this would still not actually rename the class.
Any reference to the old names would be left unchanged, and thus we are not able to achieve renaming in TypeScript.
Listing~\vref{lst:lib-rename-problem} shows how this can be a problem, where the function \codeword{f} of class \codeword{X} would fail at run-time due to it not being able to find class \codeword{A}.

\begin{code}{typescript}{Example showcasing the problems of renaming classes in a template in the library implementation.}{lst:lib-rename-problem}
    // Type-safe template declaration
    class T1 {
        static A = class {
            i = 0;
        }
        static X = class {
            f() {
                return new A();
            }
        }
    }

   // Renaming
    const classRef = T1.A;
    T1.B = classRef;
    delete T1.A;

    // Trying to use the template after renaming
    const x = new T1.X();
    x.f(); // ReferenceError: A is not defined
\end{code}

Attribute renaming would most likely be possible in a similar manner, where we could change the prototype~\footnote{The prototype of a class is an object which objects of the class inherit their methods from. See~\url{https://developer.mozilla.org/en-US/docs/Learn/JavaScript/Objects/Object_prototypes}.} of the class.
Seeing as we are not able to fully rename classes by doing this we will not be looking further into this.

\subsubsection{Instantiating Templates}\label{subsubsec:instantiating-templates}

As with renaming, we are also able to instantiate templates to an extent.
We are able to iterate over the attributes of the template class, and populate a package/template with references to the template.
An example of this can be seen in listing~\vref{code:libraryimpl-template-inst}.

\codeinputfile{inst-template-example.ts}{typescript}{Example of instantiating a template}{code:libraryimpl-template-inst}

The instantiation will only contain references to the instantiated templates classes, while PT instantiations make textual copies of the templates content.
Only having references to the original template could mean that if a template that has been instantiated is later renamed, then the instantiated template might lose some of its references.
We could possibly avoid circumvent this by getting the textual representation of the class, through the class' \codeword{toString}, and then use \codeword{eval} to evaluate the class declaration.

\subsubsection{Merging Classes}

For merging of types you would use the built-in declaration merging~\cite{declerationmerging}.
Implementation merging is also possible because JavaScript has open classes.
For implementation merging you would create an empty class which has the type of the merged declarations, and then assign the fields and methods from the merging classes to this class.
There are several libraries that supports class merging, such as mixin-js~\cite{mixinjs}.

\subsubsection{Conclusion}

Since we are not able to support renaming fully we will not be able to implement PT as a library for TypeScript.
Because of this we will have to find another approach for the project.


\subsection{Preprocessor for the TypeScript Compiler}\label{subsec:preprocessor-for-the-typescript-compiler}

Could we implement the PT specific features in a preprocessor?
In order to understand this we need to understand what a preprocessor is.
There are a lot of different definitions for preprocessors, but they are generally something that makes a source file ready for the compiler, through some simple transformations.
I will here define a preprocessor as a "dumb" compiler.
Where a compiler generally works on the source file as a tree, requiring knowledge of the underlying programming language, performing advanced tasks such as semantic analysis, a preprocessor works on the source file as a piece of text, without knowledge of the language, performing simple textual transformations such as removing comments, expanding macros (such as \codeword{\#include} in C), etc.

So the question becomes, can we transform a PTS program to TypeScript by just doing textual transformations, and not having to rely on performing more advanced tasks such as semantic analysis.
We would most likely be able to implement parts of PT with a preprocessor such as simple instantiation without renaming.
However, as we mentioned in section~\vref{sec:typescript-vs-javascript} we will need to do some type-checking in order to find the correct references when renaming, we can't just rename everything that is textually equal.
This means that we will need an understanding of the underlying programming language, something more advanced than a preprocessor to implement the features of PT\@.

\subsection{TypeScript Compiler Plugin/Transform}\label{subsec:typescript-compiler-plugin}

At the time of writing the official TypeScript compiler does not support compile time plugins.
The plugins for the TypeScript compiler is, as the TypeScript compiler wiki specifies, "for changing the editing experience only"~\cite{tscplugin}.
However, there are alternatives that do enable compile time plugins/transformers;

\begin{itemize}
    \item ts-loader~\cite{tsloadergithub}, for the webpack ecosystem
    \item Awesome Typescript Loader~\cite{awesometypescriptloadergithub}, for the webpack ecosystem.
    \item ts-node~\cite{tsnodegithub}, REPL/runtime
\end{itemize}

Unfortunately all of the above do not support adding custom syntax, as they only work on the AST produced by the TypeScript compiler.
Because of this they are not a viable option for our use-case and will therefore be discarded.

\subsection{Babel plugin}\label{subsec:babel-plugin}

Babel isn't strictly for TypeScript, but for JavaScript, however there does exist a plugin for TypeScript in babel, and we could write a plugin that depend on this TypeScript plugin.

Making a Babel plugin will make it very accessible as most web-projects use Babel, and the upkeep is cheap, as plugins are loosely coupled with the core.

In order for a Babel plugin to support custom syntax it has to provide a custom parser, a fork of the Babel parser.
Through this we can extend the TypeScript syntax with our syntax for PT\@.
This is all hidden away from the user, as this custom parser is a dependency of our Babel plugin.

Seeing as we have to make a fork of the parser in order to solve our problem, the upkeep will not be as cheap as first anticipated.
However, being able to have most of the logic loosely coupled with the compiler core it will still make it easier to keep updated than through a fork of the TypeScript compiler.

% TODO: Er det støttet å bruke flere plugins med forskjellige parsere?
% E.g. babel-plugin-typescript + vårt babel plugin?

\subsection{TypeScript Compiler Fork}\label{subsec:typescript-compiler-fork}

The TypeScript compiler is a monolith.
It has about 2.5 million lines of code, and therefore has a quite steep learning curve to get into.
If we were to go with this route it could prove a hard task to keep up with the TypeScript updates, as updates to the compiler \textit{might} break our implementation.
However, as we have seen, going the plugin/transform route also requires us to fork the underlying compiler and make changes to it, however with the majority of the implementation being loosely coupled it might presumably still make it easier to keep up-to-date.
That being said it will probably be a lot easier to do semantic analysis in a fork of the TypeScript compiler vs in a plugin/transform.

\subsection{Making a Custom Compiler}\label{subsec:making-a-custom-compiler}

Making a custom compiler for PTS might seem like a hard task, but let us dig deeper into what this entails.
Firstly we need to consider what the target should be.
Normally a compiler would output some sort of byte code, like Java byte code in the Java compiler.
Many compilers also produce native code.
Native code is pretty much out of the image for our implementation as we still want to stay in the same ecosystem, namely the browser.
We could possibly also produce WebAssembly byte code, however there are a lot of constructs in TypeScript/JavaScript that do not translate to WebAssembly, such as working with the DOM\@.
Since both of these are out of the picture we could either produce TypeScript or JavaScript.
Producing TypeScript is possibly the easiest way to go, as most of PTS is TypeScript.
And producing TypeScript also means that we could run the resulting program through the TypeScript compiler to produce JavaScript.

Having TypeScript as the target for our compiler also means that we can ignore most parts of the language and mainly focus on the PT specifics.
The rest of the language can be outputted pretty much as is, since our language will be a superset of TypeScript.


\section{Conclusion}\label{sec:planning-conclusion}


While it would be great to be able to implement Package Templates as an internal DSL in TypeScript, it would seem that this is not a suitable approach.
Even though we were able to modify the prototype of the classes in the templates, and effectively achieve some form of renaming, we were not able to rename the references.
This means that we won't be able to use the renaming to its fullest potential, and are thus not able to implement it as an internal DSL\@.
On top of this, while we were able to reproduce certain PT functionality such as simple instantiations and class merging, the fact that we are not able to change the syntax of the language, and having to define templates as classes of classes leads to a quite ugly DSL, which could also potentially be hard for the programmer to grasp.

Making a preprocessor tot the TypeScript compiler in order to implement the features of PT would presumably make the implementation time short.
However, as we learned, in order to safely rename classes and attributes we need something more powerful than a simple preprocessor.
If we were to look at the core of PT, without the renaming mechanism this would likely be the easiest approach.

Making a TypeScript compiler plugin would seemingly also be a good approach, in the future.
As we discussed the official TypeScript compiler, nor any of its alternatives, does not have proper support for plugins that would alter the syntax of the language.
Due to this we are not able to implement the features of PT, since these would require us to add extra syntax.
At the time this makes this approach not viable, however if in the future this would be supported it might prove a good approach for doing tasks such as these.

Implementing PT into the TypeScript compiler would likely lead to the most robust implementation, however the sheer size of the TypeScript compiler makes this approach undesirable.
I fear that this approach would be too time costly for this project, and might lead to an incomplete implementation as a result of this.
A similar project has been performed by Isene in~\cite{Isene2018}, where they implemented PT in C\# by extending the Roslyn compiler.
Here Isene suggested that a project of this size was better fit for a group of two.
To avoid re-discovering this we will therefore opt to go for another approach.
As well as this an implementation of PT in the TypeScript compiler would not achieve our desired trait of having a loosely coupled implementation.
This could result in a tedious process of dealing with merge conflicts when updates to the TypeScript compiler comes out.

Creating a plugin for babel might be a good approach, however since we have to implement our grammar as part of a fork of the babel parser, this makes the approach less desirable.
As with the approach of implementing PT in a fork of the TypeScript compiler this would also lead to a tightly coupled implementation, at least for the parser part of our compiler.
If we were able to write a plugin for the parsing step of Babel this might prove a viable option, however as of now there are no plans of supporting this.

Our last approach is creating a custom compiler.
As discussed previously, if we are able to % TODO

With the planning done we can jump into the implementation.

%! Author = petter
%! Date = 04.01.2021

\chapter{Implementation}\label{ch:implementation}

In this chapter we are going to look at the implementation for PT in TypeScript.

\section{Architecture / Parts of the compiler / PP}\label{sec:architecture}



\subsection{Lexer and Parser}\label{sec:lexer-and-parser}

tree-sitter grammar extending tree-sitter-typescript

\subsection{Instantiation and Renaming}\label{sec:instantiation-and-renaming}


\subsection{Verification of Templates}\label{sec:verification-of-templates}

ts api

\subsection{Code Generation}\label{sec:code-generation}

generate ts and compile ts to js through ts api.

\section{Notes on Performance}

Very slow compiler/PP because of the chosen implementation, with tree traverser for every step.



\part{Result}\label{part:Results}

%! Author = petter
%! Date = 02.05.2021

\chapter{Discussion}\label{ch:discussion}

\chapter{Does PTS Fulfill The Requirements of PT?}\label{ch:does-pts-fulfill-the-requirements-of-pt?}

This thesis is concerned about implementing Package Templates in TypeScript.
However, in order to determine to what degree we have actually implemented PT or just created something that looks like it, we have to understand what the requirements of PT are, and if we are meeting those requirements.
We will therefore in this chapter look at the requirements as described in~\cite{jot}.
After getting an understanding of the requirements we are going to look at how our implementation holds up to them.

% TODO: Kanskje bedre å ha denne inn i background til PT?
\section{The Requirements of PT}\label{sec:the-requirements-of-pt}

In~\cite{jot} the authors discuss requirements of a desired language mechanism for re-use and adaptation through collections of classes.
They then present a proposal for Package Templates, which to a large extent fulfills all the desired requirements.
% In order for us to be satisfied to call our implementation of the language mechanism Package Templates, we will also have to meet these requirements.
These requirements can therefore be used to evaluate our implementation and determine whether our implementation can be classified as a valid implementation of Package Templates.

The requirements presented in the paper were the following:

\begin{multicols}{2}
    \begin{itemize}
        \item Parallel extension
        \item Hierarchy preservation
        \item Renaming
        \item Multiple uses
        \item Type parameterization
        \item Class merging
        \item Collection-level type-checking
    \end{itemize}
\end{multicols}

In order to get a better understanding of what these requirements entail we will have to dive a bit deeper into each requirement.

\subsection{Parallel Extension}\label{subsec:parallel-extension}
% - \textbf{Parallel  extension:}  When  using  the  collection  C  in  a  certain  setting  we  can  add  attributes  to  A  and  B.
% These  additions  should  also  have  effect  for  the  code  of  C,  e.g.  so  that  we  by  means  of  an  A-variable  defined  in  C  can  directly  (without  casting) access the attributes added to A.
%
% - - In the graph example, assume that we have added the int variable length to Edge, and  that  n  is  a  Node-reference.
% With  this  property  we  can  conveniently  specify  directly: “n.firstEdge().length = 5;”, as firstEdge is typed with the extended Edge class.

The parallel extension requirement is about making additions to classes in a package/template, and being able to make use of them in the same collection.
What this means is that if we are making additions to a class, then we should be able to reference these additions in a declaration or in an addition to a separate class within the same package/template, without needing to cast it.
We can see an example of this in listing~\vref{lst:parallel-extension}, where an addition to class \codeword{B} is referencing the added method of class \codeword{A}.
The order of additions does not affect the parallel extension.
We could just as easily switched the positions of the additions around in this example.

\begin{code}{Java}{Example of parallel extension in PTj. Here we make additions to both \codeword{A} and \codeword{B} in our instantiation in package \codeword{P}, and we are able to reference the additions done to \codeword{A} in our addition to \codeword{B}. This is done without the need to cast \codeword{A}, as if the additions were present at the time of declaration.}{lst:parallel-extension}
    template T {
        class A {
            ...
        }

        class B {
            A a = new A();
            ...
        }
    }

    package P {
        inst T;

        addto A {
            void someMethod() {
                ...
            }
        }

        addto B {
            void someOtherMethod() {
                a.someMethod();
            }
        }
    }
\end{code}

\subsection{Hierarchy Preservation}\label{subsec:hierachy-preservation}
% - \textbf{Hierarchy  preservation:}  The  mechanism  should  allow  B  to  be  a  subclass  of  A,  and  if  additional  attributes  are  given  to  A  and  to  B,  then  the  B  with  additions  should be a subclass of the A with additions.
% Note that this will not be the case if we just use the collection C with the classes A and B and then define subclasses A’  and  B’  to  A  and  B,  respectively,  with  the  additions  we  want  in  these  subclasses.
% B’ will then not be a subclass of A’.
%
% - - In  the  compiler  example  this  is  exactly  what we need in order to be able to add attributes as explained in the example.
PT should never break the inheritance hierarchy of its contents.
If we have a template with classes \codeword{A} and \codeword{B}, and class \codeword{B} is a subclass of class \codeword{A}, then this relation should not be affected by any additions or merges done to either of the classes.
That is if we make additions to class \codeword{B} it should still be a subclass of class \codeword{A}, and any additions made to class \codeword{A} should be inherited to class \codeword{B}.
Even if we make additions to both class \codeword{A} and \codeword{B}, then \codeword{B} with additions should still be a subclass of class \codeword{A} with additions.

\subsection{Renaming}\label{subsec:renaming}

% - \textbf{Renaming:}  When  C  is  used,  we  should  be  able  to  change  the  name  of  A  and  B,  and of their attributes, so that they fit with the specific use situation.
%
% - - For  the  graph  example,  the  renaming  property  makes  it  possible  to  rename  the  nodes and edges to cities and roads.

The renaming requirement states that PT should enable us to rename the names of any class, and its attributes, so that they better fit their use case.

\subsection{Multiple Uses}\label{subsec:multiple-uses}
% - \textbf{Multiple  uses:}  It  should  be  possible  to  use  the  classes  of  C  for  different  and  independent  purposes  in  the  same  program,  and  so  that  each  purpose  have  different additions and renamings.
% The compiler should be able to check that each use implies a different set of classes as if they are defined in separate hierarchies.
%
% - - In  a  program  we  may  need  the  basic  graph  structure  for  different  purposes.
PT should be allow us to use packages/templates multiple times for different purposes in the same program, and any additions or renamings should not affect any of the other uses.
Each use should be kept independent of each other.
This is an important requirement of PT as when we create a package or a template it is often designed to be reused.
An example of this is the graph template we created in listing~\vref{code:rename}.
Here we bundled the minimal needed classes in order to have a working implementation for graphs.
We then used this graph implementation to model a road systems, however we might later also want to reuse the graph implementation for modelling the sewer systems of each city, and this should not be affected by any changes we made to the graph template for our road system.

\subsection{Type Parameterization}\label{subsec:type-parameterization}

% - \textbf{Type parameterization:} It should be possible to write a collection of classes that assumes the existence of classes that have some required attributes, but are not yet completely defined.
% In each use of this collection, one can provide specific classes that have at least the required attributes.
%
% - - In the compiler example we assume that the front end shall always produce Java Bytecode,  and  we  use  some  readymade  mechanism  for  packing  the  code  to  a  correctly  formatted  class  file.
% However,  a  number  of  such  packers  may  be  available,  and  we  do  not  want  to  choose  which  to  use  while  implementing  the  front end class collection.

%\red{Dette kan kanskje heller være en del av PT background. Skrev dette basert på JOT, men er det slik at PT nå bruker required types istedenfor? Da burde dette kanskje endres noe}

The requirement of type parameterization of templates works similar to how type parameterization for classes works in Java.
Type parameterization in Java enables the programmer to assume the existence of a class during declaration, and the class can later be given at the time of instantiation.
Type parameters in Java can have a constraint that it must extend another class.
Similarly, type parameterization in PT also enables the programmer to assume the existence of a class, however here the type parameter is accessible to the whole template.
PT type parameters can also be constrained, however in PT type constraining uses structural typing instead of nominal typing.
What this means is that instead of declaring what the type parameter must extend it instead declares what structure the type parameter must conform to.

In listing~\vref{lst:type-parameterization} we see an example of how type parameterization can be used to implement a list.
This example might not look too different from what you would do with type parameterization in Java, however having the type parameter at the template level does have some advantages.
One advantage of having the type parameter at template level is that you don't need to specify the actual parameter again after the instantiation.
At instantiation of the \codeword{ListsOf} template we can give i.e. a \codeword{Person} class, containing some information about a persons name, date of birth, etc., as the actual parameter, and then we would not have to keep specifying the actual parameter for every reference.
Another advantage of using type parameters at the template level is that the type parameter can be used by all classes in the template.
If we wanted to implement this in Java we would either have to have type parameters for both classes, or the \codeword{AuxElem} class would need to be a inner class of the \codeword{List} class.

\begin{code}{Java}{Example from~\cite{jot} where type parameterization is used to create a list implementation.}{lst:type-parameterization}
    template ListsOf<E> {
        class List {
            AuxElem first, last;
            void insertAsLast(E e) { ... }
            E removeFirst() { ... }
        }
        class AuxElem {
            AuxElem next;
            E e; // Reference to the real element
        }
    }
\end{code}

\subsection{Class Merging}\label{subsec:class-merging}

PT should allow for merging two or more classes.
When merging classes the result should be a union of their attributes.
If we merge two classes \codeword{A} and \codeword{B}, it should be possible to reach all of \codeword{B}'s attributes from an \codeword{A}-variable, and vice versa.

% - \textbf{Class  merging:}  Assume  we  have  the  two  collections  C  (with  classes  A  and  B)  and D (with class E).
% When they are both used in the same program, we should be able to merge e.g. the classes A and E so that the resulting class gets the union of the attributes, and so that we via an E-variable defined within D can also directly see the A attributes (and similarly for an A-variable in C).
%
% - - Assume that we in addition to the graph collection have a collection Persons with a class Person.
% In a program handling personal relations we then want to use both collections together so that we obtain a new class, say PersonNode, which has all the  attributes  of  Node  and  Person,  and  where  a  Person-variable  p  defined  in  Persons gives access directly to the Node attributes, e.g. “p.firstEdge”.

\subsection{Collection-Level Type-Checking}\label{subsec:collection-level-type-checking}

The final requirement is collection-level type-checking.
This requirement is there to ensure that each separate package/template can be independently type-checked.
By having the possibility to type-check each pacakge/template we can also verify that the produced program is also type-safe, as long as the instantiation is conflict-free.

% In  addition  to  these  properties,  it  is  important  that  such  collections  of  classes  can  be  separately type-checked.
% We also prefer a mechanism that allows only single inheritance, as  the  merge  property  described  above  to  a  large  extent  will  take  care  of  the  need  for  combining  code  from  different  sources  (for  which  purpose  multiple  inheritance  is  often  used).
% Finally, the type system should be as simple and intuitive as possible.


\section{PTS' Implementation of the Requirements}\label{sec:pts'-implementation-of-the-requirements}

With a proper understanding of the requirements of PT we can examine our implementation and see whether our implementation fulfills these requirements.
For each requirement we will be looking at a program in PTS which showcases the requirements, and the resulting program after compilation.


\subsection{Parallel Extension}\label{subsec:pts-parallel-extension}

To understand how this requirement can be fulfilled it is important to understand how the requirement could fail to be fulfilled.
A failure to fulfill the requirement would be that making additions in parallel would fail to compile, or create an otherwise incorrect program.
Failure to compile might of course not always be a bad thing, there are certain scenarios where we do want the compiler to throw an error.
There are mainly two scenarios where we would like the compilation to fail for additions, trying to make an addition to a non-existent class, and trying to reference non-existent attributes in a class.

The first scenario where our compiler should fail is when we are making additions to a non-existent class.
This will be caught in the class merging part of our compiler.
In the class merging part of the implementation the compiler will group all class declarations and additions by the class name.
If there is a group containing only additions then it will fail, as we have no class to make additions to.

The second scenario is when we are trying to reference non-existent attributes in a class.
An example of this can be seen in listing~\vref{lst:pts-parallel-extension-non-existent-attribute}.
This example will fail during the type-checking of our pacakges/templates, as discussed in~\vref{sec:type-checking-of-templates}.
Our approach for dealing with this is pretty much by not dealing with it, and instead assume that everything is okay at this stage of the compilation.
We will then instead discover any inconsistencies in the type-checking stage of the compiler.
In the aforementioned listing it is of course pretty easy to examine class \codeword{A} to see if it contains an attribute \codeword{h}, however it might not always be this easy.
In a more complicated example where we are in the process of merging several classes and additions it might prove a tougher task to see if the addition would result in a type-safe class.
So as long as we are able to perform the addition we can instead assume that it is working as intended and instead let the TypeScript compiler check if it is type-safe, after the addition has been performed.

\begin{code}{typescript}{An example showing a program that should fail during compilation, where we are trying to reference a non-existent attribute, \codeword{h}, in an addition to class \codeword{A}.}{lst:pts-parallel-extension-non-existent-attribute}
    template T {
        class A {
            function f() {
                return 1;
            }
        }
    }

    package P {
        inst T;
        addto A {
            function g() {
                this.h();
            }
        }
    }
\end{code}

Now that we understand when we want compilation to fail let us look at where we do not want it to fail, when we have a valid parallel extension.
One such way it could fail is if we tried to check if the addition contains any invalid references or type errors.
This could commonly happen if we are trying to check the addition's references to the declared class.
However, as discussed above, checking if a reference to an attribute is valid is quite tricky, and in our implementation instead leave this up to the TypeScript compiler.
By doing this we will not incorrectly throw any false-negatives when it comes to parallel extensions.
This approach does unfortunately come with some downsides.
By not addressing the issue at the addition stage it makes it harder to give informative error messages when invalid references do occur, however this was a tradeoff that was beneficial for this project.

\subsection{Hierarchy Preservation}\label{subsec:pts-hierarchy-preservation}

In order to fulfill the hierarchy preservation requirement we have to preserve all super-/subclass relations after additions and merges have been applied.
Listing~\vref{lst:pts-hierarchy-preservation} shows a program, and the resulting TypeScript program after compilation, which fulfills the requirement of hierarchy preservation.
This one example fulfills the requirement as class \codeword{B} is still a subclass to class \codeword{A} after both a merge and an addition is made to \codeword{B}.
As we talked about briefly in section~\vref{subsec:merging-class-declarations} when we merge classes we make sure to also merge their class heritage, combining the extending classes and implementing interfaces of the different classes.
This means that we might end up with instances where we are extending multiple different classes, however this will then be picked up in the type-checking stage of the compiler.
If we had not merged class heritage, we could have ended up breaking the inheritance hierarchy in the aforementioned listing, as we could have for example ended up with class \codeword{C}'s heritage, which does not have a superclass.
Because of the heritage merging we can with confidence say that we have fulfilled the requirement of hierarchy preservation, as we always preserve all super-/subclass relations.

\begin{code}{typescript}{Example showcasing a program where the super-/subclass relation between classes \codeword{A} and \codeword{B} is preserved after additions and class merging have been applied. We can see the resulting TypeScript program at the bottom of the listing, where the semantics are as expected.}{lst:pts-hierarchy-preservation}
    // PTS
    template T1 {
        class A {
            i = 0;
        }

        class B extends A {
            f() {
                return this.i;
            }
        }
    }

    template T2 {
        class C {
            j = 0;
        }
    }

    package P {
        inst T1;
        inst T2 { C -> B };
        addto B {
            k = 0;
        }
    }

    // Resulting program
    class A {
        i = 0;
    }

    class B extends A {
        f() {
            return this.i;
        }
        j = 0;
        k = 0;
    }
\end{code}

\subsection{Renaming}\label{subsec:pts-renaming}

In order to be able to fulfill the renaming requirement our implementation should be able to rename classes and their attributes.
This renaming should result in a program where not only the declarations have been renamed, but also all references.
Listing~\vref{lst:pts-renaming} shows an example program of renaming in PTS, where we are renaming a class, \codeword{A}, and the class' attribute, \codeword{i}.
We can see in the resulting program that the identifier in the declaration of both the class and the attribute has changed, but so has the references to these in the constructor of the class and references in another class, \codeword{B}\@.
The renaming has also not wrongly renamed other references that are similar in naming, such as the parameter of the constructor of class \codeword{A}\@.
This simple example works as expected, however there are also scenarios where the renaming does not work as expected.

\begin{code}{typescript}{Example of renaming in PTS.}{lst:pts-renaming}
    // PTS
    template T {
        class A {
            i = 0;
            constructor(i: number) {
                this.i = i;
            }
        }

        class B {
            a = new A();
            function f() {
                return a.i;
            }
        }
    }

    pack P {
        inst T { A -> X (i -> j) };
    }

    // Resulting program
    class X {
        j = 0;
        constructor(i: number) {
            this.j = i;
        }
    }

    class B {
        a = new X();
        function f() {
            return a.j;
        }
    }
\end{code}

% TODO diskuter rundt strukturell typing og potensiell usikkerhet rundt hvilken deklarasjon en bruksforekomst forholder seg til.
Since TypeScript is a structurally typed language we can run into scenarios where a rename could and should result in an invalid program.
Listing~\vref{lst:pts-renaming-problem} showcases this problem.
The problem arises in the \codeword{a} attribute of class \codeword{B} in template \codeword{T}.
This has been declared to be a variable expecting an object where there exists an attribute \codeword{i}.
\codeword{a} is initialised with an object of the class \codeword{A}.
This is fine in template \codeword{T}, however when \codeword{T} is instantiated in package \codeword{P}, and \codeword{A}'s attribute \codeword{i} is renamed to \codeword{j}, this is no longer the case.
Since an object of \codeword{A} no longer contains an attribute \codeword{i}, this is no longer valid.

\begin{code}{typescript}{Example showcasing the problem of having renaming in a structural language. In class \codeword{B} we have an attribute, \codeword{a}, that expects an object that contains an attribute \codeword{i}. The attribute is initialized with an \codeword{A} object. This is fine in template \codeword{T} as \codeword{A} contains an attribute \codeword{i}, however when class \codeword{A}'s attribute is renamed in the instantiation in package \codeword{P} then an object of \codeword{A} is no longer valid as a value, since it no longer contains an attribute \codeword{i}. This is an instance where we can't just rename the references to \codeword{i}, since this reference isn't explicitly related to \codeword{A}.}{lst:pts-renaming-problem}
    // PTS
    template T {
        class A {
            i = 0;
        }

        class B {
            a : { i : number } = new A();
            i = a.i;
        }
    }

    pack P {
        inst T { A -> A (i -> j) };
    }

    // Expected result
    class A {
        j = 0;
    }

    class B {
        a : { i : number } = new A();
        i = a.i;
    }
\end{code}

The aforementioned listing shows how we would like the compilation result to look like, however this is not the result the current implementation produces.
TypeScript's type system can be quite complicated, and due to a lack of time I chose to ignore most of the type declarations.
The current implementation would have treated the attribute \codeword{a} as an \codeword{A}-variable, since it is being initialized with an object of \codeword{A}, and therefore have renamed later references to \codeword{a.i} to \codeword{a.j}.
It was more important to get a working prototype, than support all scenarios with different type signatures.
This is something I would of course have liked to take into consideration if I had more time to spend on the implementation.
Deciding the type of variables is something that possibly would have come for cheaper if I had opted for a fork of the TypeScript compiler as my approach.
This is something we will come back to in~\vref{subsec:result-approach}.

\subsection{Multiple Uses}\label{subsec:pts-multiple-uses}

In order for this requirement to be fulfilled we should be able to re-use a template several times, with different renamings and additions while the different instantiations stay independent of each other.
This was something I paid extra attention to during implementation, not just to fulfill the requirement, but to avoid bugs.
I solved this by making sure that while transforming the AST this would be done in an immutable fashion.
In order to test this we will be creating a simple program where we instantiate the same template more than once and see if the resulting program is as expected.
The program can be seen in listing~\vref{lst:pts-multiple-uses}.
The program comprises a template \codeword{T} with a single class, \codeword{A}, with an attribute \codeword{i}.
This template will then be instantiated three times, where we first will be renaming the class and field, then instantiate without renaming, and finally instantiate it with just an attribute renaming.
The expected program should have two classes, one class \codeword{B}, with an attribute \codeword{j}, and a class \codeword{A} where the two bottom instantiations should have created a merged class with attributes \codeword{i} and \codeword{x}.
We can see from the resulting program after a successful compilation that this is as expected.

\begin{code}{typescript}{A program showcasing multiple uses in PTS, and the resulting program in TypeScript at the bottom.}{lst:pts-multiple-uses}
    // PTS
    template T {
        class A {
            i = 0;
        }
    }

    pack P {
        inst T { A -> B (i -> j) };
        inst T;
        inst T { A -> A (i -> x) };
    }

    // Resulting program
    class B {
        j = 0;
    }

    class A {
        i = 0;
        x = 0;
    }
\end{code}

\subsection{Type Parameterization}\label{subsec:pts-type-parameterization}

The type parameterization requirement is something the implementation does not fulfill.
This was not implemented due to it not being prioritized.
There is only so much time available during the span of a master thesis, and I chose to look at how the core of PT would fit into a structurally language like TypeScript, rather than on making sure it would be a fully fleshed out implementation of PT\@.
Another reason for avoiding this is that much of type parameterization can be achieved through merging and renaming.
Listing~\vref{lst:type-parameterization-without-rt} shows an example of how you can use an empty class as a generic type implementation of lists, similar to the list implementation with required types in listing~\vref{lst:type-parameterization}.
Required types do of course have a lot of advantages such as making it possible to constrain the type, and forcing the programmer to give an actual parameter for the type, which we are unable to do.

\begin{code}{java}{Example of a similar list implementation as in listing~\vref{lst:type-parameterization}, without the use of required types. Instead of giving a type for the required type we will have to merge the class \codeword{E} with the "actual parameter".}{lst:type-parameterization-without-rt}
    template ListsOf {
        class E { }
        class List {
            AuxElem first, last;
            void insertAsLast(E e) { ... }
            E removeFirst() { ... }
        }
        class AuxElem {
            AuxElem next;
            E e;
        }
    }
\end{code}

\subsection{Class Merging}

\subsection{Collection-level Type-checking}\label{subsec:implementation-collection-level-type-checking}

\section{Conclusion}\label{sec:requirements-conclusion}

While we do not fulfill every requirement, we do fulfill most of them.
The current implementation might not be a full implementation of PT, but we can confidently say we have at least made an implementation of the core of PT for TypeScript.
Not having a full implementation does mean that we might not be able to examine all the differences between our implementation and PTj, however we will be able to examine the common elements, which covers the most interesting parts.
This allows us to explore how a mechanism like PT fits with the TypeScript language, and its potential utility.

%! Author = petter
%! Date = 04.01.2021

\section{Nominal vs. Structural Typing in PT}\label{sec:difference-between-pts-and-ptj}

One of the most notable differences between PTS and PT are the underlying languages' type systems.
PTS, as an extension of TypeScript, has structural typing, while PT on the other hand, an extension of Java, has nominal typing.

Nominal and structural are two major categories of type systems.
Nominal is defined as "being something in name only, and not in reality" in the Oxford dictionary.
Nominal types are as the name suggest, types in name only, and not in the structure of the object.
They are the norm in mainstream programming languages, such as Java, C, and C++.
A type could be \codeword{A} or \codeword{Tree}, and checking whether an object conforms to a type restriction, is to check that the type restriction is referring to the same named type, or a subtype.
Structural types on the other hand are not tied to the name of the type, but to the structure of the object.
These are not as common in mainstream programming languages, but are very prominent in research literature.
However, in more recent (mainstream) programming languages, such as Go and TypeScript, structural typing is becoming more and more common.
A type in a structurally typed programming language is often defined as a record, and could for example be \codeword{\{ name:~string \}}.

In listing~\vref{lst:nominal-typing-example} we can see an example of a nominally typed program in a Java-like language.
Here \codeword{B} is a subtype of \codeword{A}, while \codeword{C} is not.
This is due to nominally typed programs having the requirement of explicitly naming its subtype relations, through e.g.\ a subclass-relation.
Because of this we can see that at the bottom of the listing the first two statements pass, since both \codeword{A} and \codeword{B} are of type \codeword{A}, while the last statement fails (typically at compile time), as \codeword{C} is not of type \codeword{A}.

In listing~\vref{lst:structural-typing-example} we see a structurally typed program.
This program also has the exact same declarations as in listing~\vref{lst:nominal-typing-example}, that is classes \codeword{A}, \codeword{B}, and \codeword{C} and the function \codeword{g}.
In this program both type \codeword{B} and type \codeword{C} are a subtype of type \codeword{A}, since they both contain all members of type \codeword{A}.
Not necessarily the same implementation as in class \codeword{A}, but the same types as in type \codeword{A}.
This is one of the major differences between nominal and structural typing, types can conform to other types without having to explicitly state that they should.
Type \codeword{C} is an example of this, while it does not have a subclass relation to class \codeword{A}, nor implement any common nominal interface, it still conforms to the type of \codeword{A}.
The result of this is that all three usages of function \codeword{g} are valid in a structural type system, while consuming \codeword{C} was illegal in the nominal example.


\begin{code}{Java}{Example of a nominally typed program in a Java-like language}{lst:nominal-typing-example}
    // Given the following class definitions for A, B and C:
    class A {
        void f() {
            ...
        }
    }

    class B extends A {
        ...
    }

    class C {
        void f() {
            ...
        }
    }

    // And a consumer with the following type:
    void g(A a) { ... }

    // Would result in the following
    g(new A()); // Ok
    g(new B()); // Ok
    g(new C()); // Error, C not of type A
\end{code}

\begin{code}{Java}{Example of a structurally typed program in a Java-like language}{lst:structural-typing-example}
    // Given the same class definitions and
    // the same consumer as in the previous listing.
    // Would result in the following
    g(new A()); // Ok
    g(new B()); // Ok
    g(new C()); // Ok, because C is structurally equal to A
\end{code}

\subsection{Advantages of Nominal Type Systems}\label{subsec:advantages-of-nominal-types}

\subsubsection{Subtypes}\label{subsubsec:subtypes}

In nominal type systems it is trivial to check if a type is a subtype of another, as this has to be explicitly stated, while in structural type systems this has to be structurally checked, by checking that all members of the super type, are also present in the subtype.
Because of this each subtype relation only has to be checked once for each type, which makes it easier to make a more performant type checker for nominal type systems.
However, it is also possible to achieve similar performance in structurally typed languages through some clever representation techniques~\cite{tapl}. % TODO: Maybe find an article describing this
We can see an example of subtype relations in both nominal and structural type systems, in a Java-like language, in listing~\vref{lst:subtype}.
It is important to note that even though \codeword{C} is a \emph{subtype} of \codeword{A} in a structural language, it is not a \emph{subclass} of \codeword{A}.

\begin{code}{Java}{Example of subtype relations in nominal and structural typing, in a Java-like language. In the example of the nominal subtype we have to explicitly state the subtype relation, while in the structural subtype example the subtype relation is inferred from the common attributes.}{lst:subtype}
    // Given class A
    class A {
        void f() { ... }
     }

    // A subtype, B, in nominal typing
    class B extends A { ... }

    // A subtype, C, in structural typing
    class C {
        void f() { ... }
        int g() { ... }
    }
\end{code}

% TODO: Vurder om recursive types er verdt å ta med
% \subsubsection{Recursive types}\label{subsubsec:recursive-types}
%
% Recursive types are types that mention itself in its definition, and are widely used in datastructures, such as lists and trees.
% An advantage in nominal typing is how natural and intuitive recursive types are in the type system.
% Referring to itself in a type definition is as easy as referring to any other type.
% It is however just as easy to do this in structural type systems as well, however for calculi such as type safety proofs, recursive types come for free in nominal type systems, while it is a bit more cumbersome in structurally typed systems, especially with mutually recursive types~\cite{tapl}.
% Listings~\vref{lst:ts-recursive-type} and~\vref{lst:java-recursive-type} shows the use of recursive types in TypeScript (structurally typed) and Java (nominally typed), respectively.
%
% \begin{code}{TypeScript}{Usage of a recursive type, \codeword{Tree}, in TypeScript}{lst:ts-recursive-type}
%     interface Tree<T> {
%         getData(): T;
%         getChildren(): Tree<T>[];
%     }
% \end{code}
%
% \begin{code}{Java}{Usage of a recursive type, \codeword{Tree}, in Java}{lst:java-recursive-type}
%     interface Tree<T> {
%         T getData();
%         List<Tree<T>> getChildren();
%     }
% \end{code}

\subsubsection{Runtime Type Checking}\label{subsubsec:runtime-type-checking}

Often runtime-objects in nominally typed languages are tagged with the types (a pointer to the "type") of the object.
This makes it cheap and easy to do runtime type checks, like in upcasting or doing a \codeword{instanceof} check in Java.
It is also easier to check sub-type relations in nominal type systems, even though you might still have to do a structural comparison, you only have to perform this once per type~\cite{tapl}.
% TODO: Skriv mere

\subsection{Advantages of Structural Type Systems}\label{subsec:advantages-of-structural-types}

\subsubsection{Tidier and More Elegant}\label{subsubsec:tidier-and-more-elegant}

% TODO Eyvind synes dette er litt for subjektivt og preachy, bør finne noe annet.

Structural types carry with it all the information needed to understand its meaning.
This is often seen as an advantage over nominal typing as the programmer arguably only has to look at the type to understand its meaning, while in nominal typing you would often have to look at the implementation or documentation to understand the type, as the type itself is part of a global collection of names~\cite{tapl}.

\subsubsection{More General Functions/Classes}\label{subsubsec:more-general-functions}

Malayeri and Aldrich performed a study (see~\cite{malayeri}) on the usefulness of structural subtyping.
The study was mainly focused around two characteristics of nominally-typed programs that would indicate that they would benefit from a structurally typed program.
The first characteristic was that a program is systematically making use of a subset of methods of a type, in which there is no nominal type corresponding to the subset.
The second characteristic was that two different classes might have methods which are equal in name and perform the same operation, but are not contained in a common nominal supertype.
29 open-source Java projects were examined for these characteristics.

For the first characteristic the authors ran structural type inference over the projects and found that on average the inferred structural type consisted of 3.5 methods, while the nominal types consisted of 37.8 methods.
While for the second characteristic the authors looked for types with more than one common methods and found that every 2.9 classes would have a common method without a common nominal supertype.
We can see that from both of these characteristics that the projects could have benefited from a structural type system, as this would make the programs more generalized, and could therefore support easier re-use of code.

\subsection{Disadvantage of Structural Type Systems}\label{subsec:disadvantage-of-structural-type-systems}

It is worth noting that the advantage of types conforming to each other without explicitly stating it in structural type systems can also be a disadvantage.
Structurally written programs can be prone to \textit{spurious subsumption}, that is consuming a structurally equal type where it should not be consumed.
An example of this can be seen in listing~\vref{lst:spurious-subsumption}.

Here the function \codeword{double} will consume an object that has a \codeword{calculate} attribute.
The intended use is to consume something that does a calculation on the object and returns a number which will be doubled, while the unintended use example in this case does some unexpected side effect and returns a number as a status code.
The unintended object can be consumed by the \codeword{double} function as it is conforms to the signature of the function, while in a nominally typed system this can be avoided to a much larger extent.

\begin{code}{typescript}{Example of spurious subsumption in TypeScript}{lst:spurious-subsumption}
    function double(o: {calculate: () => number}) {
        return o.calculate() * 2
    }

    const vector = {
        x: 2,
        y: 3,
        calculate: () => 4
    }

    // function calculate also returns number, but as a status code
    const unintended = {
        calculate: () => {
            doSomeSideEffect();
            return 1;
        }
    }

    double(vector); // Ok, intended

    double(unintended); // Not intended use,
                        // but it is type-safe to do so.

\end{code}

%\subsection{What Difference Does This Make For PT?}\label{subsec:what-difference-does-this-make-for-pt?}
%
%We are not going to go further into comparing nominal and structural type systems and "crown a winner", as there are a lot useful scenarios for both nominal and structural type systems.
%We will instead look more closely into how a structural type system fits into PT, and what differences this makes to the features, and constraints, of this language mechanism.


\subsection{Which Better Fits PT?}\label{subsec:which-better-fits-pt?}

With the addition of required types in PT the language mechanism now has to utilize structural typing, independent of the underlying language's type system.
Using structural typing was seen as a necessity for required types as this would give the mechanism its required flexibility.
One could therefore argue that a structural type system is a better fit for Package Templates as it would remove the confusion of dealing with two different styles of typing in a single program, and make the language mechanism feel more like a first class citizen of the host language.

Another advantage for having structural typing for PT is that it can help strengthen one of the main concepts of Package Templates, namely re-use.
As we learned from the study of Malayeri and Aldrich, structural typing can make programs more general which makes them more prone to re-use.

However, there are also some quite significant problems with having PT in a structural language.
Renaming is especially something that might not fit nicely into a structurally typed language.
In listing~\vref{lst:problems-with-structural-pt} we see an example of a program that breaks after renaming an attribute.
The renaming resulted in class \codeword{Consumable} no longer conforming to the signature of function \codeword{f} in class \codeword{Consumer}.
PT does not support changing the signature of functions so there is no way for us to be able to make the \codeword{Consumable} class conform.
In order to avoid running into this problem we might consider disallowing inline type declaration.
This would force us to give an interface as the type for the formal parameter, \codeword{consumable}.
We could then also rename the members of the interface in order to once again make the \codeword{Consumable} class conform to the signature of function \codeword{f}.

It is worth noting that the problem of renaming causing programs to break is not something unique to structural typing, this can also occur in nominally typed programs.
Listing~\vref{lst:problems-with-nominal-pt} showcases a program that breaks after renaming.
In this listing we see that after renaming method \codeword{f} of class \codeword{A} the class no longer fulfill the requirements of the implementing interfaces \codeword{I}, as \codeword{I} expects a method \codeword{f} to be present, which it no longer is.
We could resolve this by also performing an equal renaming to interface \codeword{I}.
Although it is a problem in nominal PT as well, it is less so than with structural PT, since it can be resolved with just an additional renaming, and due to the relation being explicitly stated we can also give an error message during compilation notifying the programmer of this inconsistency.
In structural PT we would have to disallow inline-types in order to make the problem more solvable for the programmer, but due to the relation between the class and interface not necessarily being explicitly stated it would be harder to give a sensible warning for the programmer.

With the discussed general advantages and disadvantages of structural and nominal type systems, and the points brought forward in this section we can see that both styles of typing have strong use cases with PT\@.
A nominal type system in PT would seemingly lead to less problematic renaming scenarios, while a structural type system in PT would arguably fit better with the overall theme of PT, flexible re-use.


\begin{code}{typescript}{Example of how using renaming in PTS might break a program. After renaming the field \codeword{i} to \codeword{j} the class \codeword{Consumable} is no longer consumable by function \codeword{f} in class \codeword{Consumer}.}{lst:problems-with-structural-pt}
    template T {
        class Consumable {
            i = 0;
        }

        class Consumer {
            function f(consumable : {i: number}) {
                ...
            }
        }
    }

    pack P {
        inst T { Consumable -> Consumable (i -> j) };
    }
\end{code}

\begin{code}{Java}{Example of how using renaming in PT might break a program. After renaming the method \codeword{f} to \codeword{g} the class, \codeword{A}, no longer conform to the implementing interface \codeword{I}.}{lst:problems-with-nominal-pt}
    template T {
        interface I {
            void f();
        }

        class A implements I {
            void f() { ... }
        }
    }

    package P {
        inst T with A => A (f() -> g());
    }
\end{code}

\chapter{Results}\label{ch:results}

In this thesis we have discussed the approach to the project and the implementation of the compiler for the PTS language, as well as evaluated the implementation and discussed the consequences of having a structurally typed language as the host of PT were.
We will conclude this work by looking at how our implementation of the PTS programming language can be used in a real world example, as well as returning to our initial research questions and trying to answer these with the knowledge we have accumulated through the span of this thesis.
After this we will look at what could have been done differently in retrospect, and finish off with proposing future works within the field.

\section{How Can PTS Be Used?}\label{sec:how-can-this-be-used?}

There are mainly two ways of using the PTS compiler:

\begin{itemize}
    \item Installing it globally, or
    \item Creating a PTS project
\end{itemize}

In the following sections we will look at how you can use install and use it for both approaches.

The PTS compiler requires you to have Node and NPM installed on your computer.

\subsection{Installing and Using PTS Globally}\label{subsec:installing-and-using-pts-globally}

Installing PTS globally will enable you to use PTS anywhere, and might be favorable if you are planning to create several smaller projects to test it out, or if you are not too experienced with the node ecosystem.
If you want to install the compiler globally you can do the following:

\begin{minted}{bash}
    $ npm install -g pts-lang
\end{minted}

This will give you access to the PTS compiler CLI through the command \codeword{pts-lang}.
By giving the \codeword{-{}-help} flag you will get some useful information for how to use the compiler.

\begin{minted}{text}
$ pts-lang --help
Options:
      --help                      Show help
      --version                   Show version number
  -i, --input                     Name of the input file
  -o, --output                    Name of the output file
  -v, --verbose                   Show extra information during
                                  transpilation
  -t, --targetLanguage, --target  Target language for
                                  transpilation
  -r, --run
\end{minted}

\subsection{Creating a PTS Project}\label{subsec:creating-a-pts-project}

If you are using PTS for a specific project it might be better to set it up as a project dependency in npm.
When installed in a npm project the CLI is available to use through npm scripts or through accessing it directly from the \codeword{node\_modules} folder in your project.
The compiler can also be access through the API by importing it as with any other npm package.

Installing it inside a npm project will not require you to install it globally, as it will stay contained in the project.
This also means that any contributors of the project will not have to worry about installing PTS, as it will be installed when the project is set up.

To initialize an NPM project you can do the following:

\begin{minted}{bash}
    $ mkdir <project name>
    $ cd <project name>
    $ npm init -y
\end{minted}

With a project set up you can install the PTS compiler as following:

\begin{minted}{bash}
    $ npm install pts-lang
\end{minted}

With the PTS compiler installed in the project you can then set up some scripts to start and/or build the project.
This can be done by adding scripts to the project's \codeword{package.json}.
Below you can see an example of a section of a \codeword{package.json} file with scripts for running and building a file:

\begin{minted}{json}
    {
      "scripts": {
        "start": "pts-lang -i src/index.pts --run",
        "build": "pts-lang -i src/index.pts -o build/index"
      }
    }
\end{minted}

The start script only runs the program, and does not emit any files, while the build script transpiles the \codeword{src/index.pts} file to JavaScript.
If you would rather have TypeScript output you can use the \codeword{-t} flag to specify this:

\begin{minted}{bash}
    pts-lang -i src/index.pts -o build/index -t ts
\end{minted}

\subsection{A Real World Example}\label{subsec:a-real-world-example}

Now that we understand how to get PTS set up, let us look at how it can be used in a real world example, and how this enables the programmer to modularize the code base even further giving great flexibility.

The most common use of TypeScript is to create web applications.
Let us look at how PTS can help make this task easier for the programmer.
We will try to create a simple web application for displaying a Pokémon.
To do this we will use one of the most popular web frameworks, React.

React is a web framework developed by Facebook.
It makes creating scalable web projects easier to handle, through enabling the programmer to modularize collections of elements into "components".
These components are often created to make reuse of common elements easier, such as creating a styled button with certain features, or we could create a component to represent the entire web application.

Displaying one specific Pokémon can be pretty simple, however we would like to create a React component that can display information and a picture of any Pokémon.
We do not want to have to write down information about all Pokémon, so we will fetch this information from an API, more specifically the PokéAPI (\url{https://pokeapi.co/}).
This API lets us fetch data from all Pokémon.

We will start of with the task of fetching data.
As this is something you commonly want to do in web applications it might be a good idea to separate this logic into a separate template:

\begin{minted}{typescript}
    template FetchJSON {
        class FetchJSON extends Component {
            componentDidMount() {
                fetch(this.props.url)
                    .then(response => response.json())
                    .then(data =>
                        this.setState(state => ({...state, data}))
                    ).catch(error =>
                        this.setState(state => ({...state, error}))
                    );
            }
        }
    }
\end{minted}

The component we see above will fetch whatever URL we pass to it in its props and update the state with the results of the fetch.
If we for some reason should fail to fetch the data we will instead update the state with the error message we got.

In addition to fetching data, it might be useful to have a logger, which will log all state changes to the console.
This is often useful when working with React components as we are able to see when they update, and what the state was at the time of the update.
Such a logger could then also be separated into its own template, like the following:

\begin{minted}{typescript}
    template StateLogger {
        class StateLogger extends Component {
            componentDidUpdate() {
                console.log("State updated!", this.state);
            }
        }
    }
\end{minted}

Finally we would like to combine these into our Pokémon component, and add some logic for displaying the information.
We will do this inside of a package, so that this will produce an output:

\begin{minted}{typescript}
    pack Pokemon {
        inst FetchJson { FetchJson -> Pokemon };
        inst StateLogger { StateLogger -> Pokemon };
        addto Pokemon {
            render() {
                if(this.state.error) {
                    return (
                        <div>
                            <h1>An error occurred</h1>
                            <p>{this.state.error.message}</p>
                        </div>
                    );
                }

                if(this.state.data === undefined) {
                    return 'Loading...';
                }

                const name = this.state.data.name;
                const pokemonTypes = this.state.data.types;
                const image = this.state.data.sprites.front_default;
                return (
                    <div>
                        <img src={image} />
                        <h1>{name}</h1>

                        <h2>Types</h2>
                        <ul>
                            {pokemonTypes.map(pokemonType => (
                                <li>{pokemonType.name}</li>
                            ))}
                        </ul>
                    </div>
                )
            }
        }

    }
\end{minted}

We could then use our Pokémon component in our application by supplying a URL for the Pokémon to display, as seen below:

\begin{minted}{typescript}
    class App extends Component {
        render() {
            <Pokemon
                url="https://pokeapi.co/api/v2/pokemon/ditto" />
        }
    }
\end{minted}


\section{Addressing Research Questions}\label{sec:adressing-research-questions}


\subsubsection{RQ1: How does the language mechanism, Package Templates, fit into TypeScript and the web ecosystem?}

\red{Tenker at dette spørsmålet blir litt for bredt, og at svaret fort blir litt for subjektivt, så tenker nok å fjerne dette spørsmålet. Web bruker jo for tiden også særdeles lite klasser, da de fleste heller velger å uttrykke seg gjennom den funksjonelle delen av JS/TS, så det passer kanskje egentlig ikke så bra inn heller.}


\subsubsection{RQ2: Does structural typing change how the core of Package Templates works?}

Most of the functionality of Package Template stay the same in a structurally typed language as they do in a nominally typed language.
The most notable deviation is the renaming mechanism.
While we still rename all valid references, it is worth noting that what a reference is will differ in a structurally typed language.

In nominal PT, all variables that are instantiated with an object of a class, will also have some explicit relation to the class.
A variable in nominal PT will likely have either the class (or a superclass) as the type, or an interface.
For the case of the variable having the class as the type the relation to itself is obviously there.
For interface or superclass as the type the relation must explicitly be declared in the class' heritage.
If the class' attributes are renamed we will either be able to rename the variables references to these attributes if the renaming was applied to the class or any of its superclasses, or at least give a detailed error message for when the renaming was applied to the interface.

In structural PT, variables instantiated with an object of a class might not have the same explicit relation to the constructed class.
A variable in a structurally typed language might be typed with a structure that the class conforms to.
However, this conformity might break after a renaming has been applied to the class.
In these cases we will not be able to rename the variables type, nor any of the variables references to the possibly renamed attributes, since these do not have the same explicit relation to the class.

\subsubsection{RQ3: Will having PT in a structurally typed language have any notable benefits over having it in a nominally typed language?}

As we discussed in section~\vref{subsec:which-better-fits-pt?} there are some benefits of hosting PT in a structurally typed language.
Structural typing does fit nicely in with the theme of Package Templates, namely re-use.
A structural type system gives the programmer a flexibility that nominal type systems can not offer to the same extent.
One of the strongest points for structural typing is how it enables us to easily use third-party libraries without necessarily having to alter our classes.
Say we have implemented a graph library in PT and later on wanted to use some third-party graph utility library.
This library might take a graph as input and do some calculations such as finding the shortest path between nodes in the graph.
In structural typing, as long as our graph implementation is structurally equal to the type of the input we can pass it with no problems.
In a nominally typed language we might have to alter our implementation to explicitly declare that our classes implement some interfaces from the library.
It might at worst not even be possible if the graph utility library has typed their input with classes instead of interfaces.
This is of course one of the strengths of PT in the first place, being able to add implementing interfaces to a class without altering the original implementation or merging classes if possible, however the fact that we would not have to perform this step in a structurally typed language could arguably be seen as a benefit.

\section{In Retrospect}\label{sec:in-retrospect}

\subsection{Approach}\label{subsec:result-approach}

While we were able to implement most of PT with our approach, I fear that further development to reach a complete implementation might be hindered by our chosen approach.
This is largely because we might have to implement much of the type system in order to properly identify references, as we briefly discussed in section~\vref{subsec:supporting-all-references}.
Instead of re-inventing the wheel, we might be better off by implementing PT in a fork of the TypeScript compiler.
While this \textit{might} make updates harder, than with our implementation, we will most often likely be able to merge the changes to the TypeScript compiler into our fork with the help of Git, conflict free.
Greater changes to the language might of course still give us some merge conflicts, however these larger changes could as easily make our currently used approach break.

\section{Future Work}\label{sec:future-work}

\subsection{Finishing the PTS Compiler}\label{subsec:finishing-the-pts-compiler}

As detailed in section~\vref{sec:completing-the-implementation} the implementation of the compiler was not completed, and requires some further work to complete.
The majority of the work will presumably lie in performing more advanced semantic analysis in order to correctly identify all references.
This could as we have pointed out either be attempted in a fork of the TypeScript compiler, or as a continuation of this compiler.
If one is to continue on with this implementation it could be worth looking into if it is possible to use the TypeScript compiler API in order to identify references.

\subsection{Improve the Compilers Error Messages}\label{subsec:compiler-with-focus-on-error-messages}

As I mentioned shortly in subsection~\vref{subsec:the-ast-nodes} tree-sitter does have support for giving position of a syntax node, and this could be utilized to produce better error messages.
In addition to giving the position of where the error occurred, it would also be helpful to give more informative error messages than we currently do.
In our implementation errors are usually first found during the type-check phase, where we might have already instantiated templates, and renamed classes and attributes.
We should try to make more effort to spot errors earlier rather than later.

\subsection{Making Syntax Highlighting for the PTS Language}\label{subsec:making-syntax-highligthing-for-the-pts-language}

By utilizing tree-sitter as our lexer/parser we could presumably pretty easily also utilize it to get syntax highlighting.
Most modern editors and IDEs have recently been switching to tree-sitter for syntax highlighting, opposed to the traditional method of using regex to highlight code files.
This should be as simple as writing query files for identifying keywords, operators, etc.
Similar to how we extended the TypeScript grammar in order to create our PTS grammar, it should also be possible to extend the TypeScript highlight query files to create syntax highlighting for PTS\@.
In the GitHub repository for the tree-sitter grammar for PTS there has been done some initial work for setting this up, but has been abandoned as implementing features for the compiler was more urgent.



\backmatter{}
\printbibliography
\end{document}
